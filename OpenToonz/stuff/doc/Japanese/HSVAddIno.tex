\documentclass[a4paper,12pt]{article}
\usepackage[a4paper, total={180mm, 272mm}]{geometry}

\usepackage{fontspec}
\setmainfont{ipaexm.ttf}

\setlength\parindent{3.5em}
\setlength\parskip{0em}
\renewcommand{\baselinestretch}{1.247}

\begin{document}

\thispagestyle{empty}

\Large
\noindent \\
HSV Add Ino\medskip
\par
\normalsize
参照画像(Noise)の指定チャンネルの値で肌合いを加えます。\par
色相、彩度、明度、あるいは Alpha が変化します。\\
\par
セル調の画像に、色相、彩度、明度、あるいは Alpha でノイズを加え、\par
背景の絵に馴染ませる等々に使います。\par
同様な事をノイズツールで行うと、1ドット毎のノイズなので絵が1本調子になります。\par
それとは違い、ノイズを参照画像としてユーザーが用意し、\par
その絵をノイズ成分として計算し、画像の肌合いを決めます。\\
\par
あくまで、2次元上の静止画像に対する処理なので、立体を感じさ\par
せる場合や、動画の場合は、別の考察、処理(?)が必要です。\\
\par
Alpha チャンネルによって効果の強さが決まります。よって、\par
Alpha によるマスクエッジが滑らかであればその状態を保持します。\\
\\
-{-}- \ 入力 \ -{-}-\\
Source\par
処理をする画像を接続します。\\
Noise\par
Pixel 毎に HSV 値によってノイズを加えるための画像を接続します。\\
Reference\par
Pixel 毎に効果の強弱をつけるための参照画像を接続します。\\
\\
-{-}- \ 設定 \ -{-}-\\
From RGBA\par
参照画像(Noise)のどの channel を使用するかを指定します。\par
\textquotedbl Red\textquotedbl \ \textquotedbl Green\textquotedbl \ \textquotedbl Blue\textquotedbl \ \textquotedbl Alpha\textquotedbl のどれかを選びます。\par
初期値は\textquotedbl Red\textquotedbl です。\\
\\
Offset\par
参照画像(Noise)の Pixel 値に対する offset 値です。\par
Pixel 値(8 or 16bits)をゼロから1の値として、-1.0から1.0の間\par
の値で指定します。\par
参照 Pixel 値がここでの指定と同じ値のときは変化しません。\par
この値に対する参照 Pixel 値の大小によって、変化がおきます。\par
初期値は0.5です。\\
\\
Hue\par
色相(Hue)変化の強さを指定します。\par
offset 値を中心としてスケールがかります。

\newpage

\thispagestyle{empty}

\ \vspace{-0.2em}
\par
-1.0から1.0の間の値で指定します。\par
ゼロだとなにも変化しません。\par
初期値は0.0です。\\
\\
Saturation\par
彩度(Saturation)変化の強さを指定します。\par
初期値は0.0です。\par
他は\textquotedbl Hue\textquotedbl と同様です。\\
\\
Value\par
明度(brightness Value)変化の強さを指定します。\par
初期値は0.25です。\par
他は\textquotedbl Hue\textquotedbl と同様です。\\
\\
Alpha\par
不透明度(Alpha)変化の強さを指定します。\par
初期値は0.0です。\par
他は\textquotedbl Hue\textquotedbl と同様です。\\
\\
Premultiply\par
ON なら、RGB に対して Premultiply 済の\par
(Alpha チャンネルの値があらかじめ RGB チャンネルに乗算されている)\par
画像として処理します。\par
そのとき、Alpha にも処理を加えてしまうと、正しい画像にならない場合があります。\par
初期値は ON です。\\
\\
Reference\par
Pixel 毎に効果の強弱をつけるための参照画像の値の取り方を選択します。\par
入力の\textquotedbl Reference\textquotedbl に画像を接続し、\par
Red/Green/Blue/Alpha/Luminance/Nothing から選びます。\par
この効果をつけたくないときは Nothing を選ぶか、接続を切ります。\par
初期値は Red です。

\end{document}