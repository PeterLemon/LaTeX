\documentclass[a4paper,12pt]{article}
\usepackage[a4paper, total={180mm, 272mm}]{geometry}

\usepackage{fontspec}
\setmainfont{ipaexm.ttf}

\setlength\parindent{4.2em}
\setlength\parskip{0em}
\renewcommand{\baselinestretch}{1.247}

\begin{document}

\thispagestyle{empty}

\Large
\noindent \\
Level Master Ino\medskip
\par
\normalsize
レベル補正をします。\par
RGB 別に補正したい時は"Level RGBA"を使用してください。\\
\\
--- \ 入力 \ ---\\
Source\par
処理をする画像を接続します。\\
Reference\par
Pixel 毎に効果の強弱をつけるための参照画像を接続します。\\
\\
--- \ 設定 \ ---\\
In\par
入力 Pixel 値の Min(最小)値と Max(最大)値を指定します。\par
Min より小さい値は Min に、Max より大きい値は Max に制限します。\par
最小は0、最大は1です。\par
初期値は、Min が0、Max が1、です。\par
小数点以下4桁までの入力になります。\\
\\
Out\par
"In"で指定した範囲を、\par
ここで指定する Min(最小)値と Max(最大)値の範囲に当てはめます。\par
最小は0、最大は1です。\par
初期値は、Min が0、Max が1、です。\par
小数点以下4桁までの入力になります。\\
\\
Gamma\par
"Out Min"と"Out Max"の間で gamma 補正します。\par
0.1から1.0の間だと、画像が暗くなります。\par
1.0を指定すると補正しません。\par
1.0から10.0の間では、明るくなります。\par
初期値は1です。\\
\\
Alpha Rendering\par
ON で Alpha にも処理をします。\par
OFF のときは、Alpha に処理しません。\par
初期値は ON です。\\
\\
Premultiplied\par
ON なら、RGB に対して Premultiply 済の\par
(Alpha チャンネルの値があらかじめ RGB チャンネルに乗算されている)\par
画像として処理します。

\newpage

\thispagestyle{empty}

\ \vspace{-0.2em}
\par
そのとき、Alpha にも処理を加えてしまうと、正しい画像にならない場合があります。\par
初期値は ON です。\\
\\
Reference\par
Pixel 毎に効果の強弱をつけるための参照画像の値の取り方を選択します。\par
入力の"Reference"に画像を接続し、\par
Red/Green/Blue/Alpha/Luminance/Nothing から選びます。\par
この効果をつけたくないときは Nothing を選ぶか、接続を切ります。\par
初期値は Red です。

\end{document}