\documentclass[a4paper,12pt]{article}
\usepackage[a4paper, total={180mm, 272mm}]{geometry}

\usepackage{fontspec}
\setmainfont{ipaexm.ttf}

\setlength\parindent{4.2em}
\setlength\parskip{0em}
\renewcommand{\baselinestretch}{1.247}

\begin{document}

\thispagestyle{empty}

\Large
\noindent \\
HLS Noise Ino\medskip
\par
\normalsize
色相、明度、彩度、Alpha にドットノイズを加えます。\par
セル画調の絵にノイズを加え、背景の絵に馴染ませることを目的と\par
して開発しました。\\
\par
Alpha チャンネルによってノイズの強さが決まります。よって、滑\par
らかなエッジは滑らかなままです。\par
Alpha チャンネル自身のノイズの強さも、自身の値を見ます。\\
\par
結果をチェックするときは、サブカメラを使わないでください。\par
サブカメラは入力画像の範囲が違うため、ノイズパターンが変わっ\par
てしまいます。\\
\\
--- \ 入力 \ ---\\
Source\par
処理をする画像を接続します。\\
Reference\par
Pixel 毎に効果の強弱をつけるための参照画像を接続します。\\
\\
--- \ 設定 \ ---\\
Hue\par
色相(hue)についてノイズの強さを指定します。\par
Pixel 値(8 or 16bits)をゼロから1の値として指定します。\par
最小は0、最大は1です。\par
0のときは色相(hue)に対するノイズがかかりません。\par
初期値は0.025です。\\
\\
Lightness\par
明度(lightness)についてノイズの強さを指定します。\par
Pixel 値(8 or 16bits)をゼロから1の値として指定します。\par
最小は0、最大は1です。\par
0のときは明度(lightness)に対するノイズがかかりません。\par
初期値は0.035です。\\
\\
Saturation\par
彩度(saturation)についてノイズの強さを指定します。\par
Pixel 値(8 or 16bits)をゼロから1の値として指定します。\par
最小は0、最大は1です。\par
0のときは彩度(saturation)に対するノイズがかかりません。\par
初期値は0.0です。

\newpage

\thispagestyle{empty}

\ \vspace{-0.2em}
\par
\noindent Alpha\par
Alpha チャンネルについてノイズの強さを指定します。\par
Pixel 値(8 or 16bits)をゼロから1の値として指定します。\par
最小は0、最大は1です。\par
0のときは Alpha チャンネルに対するノイズがかかりません。\par
初期値は0.0です。\\
\\
Seed\par
画像のノイズパターンを決定するための値です。\par
ゼロ以上の整数値を指定します。\par
この値が同じであれば、パターンを再現します。\par
違う値でノイズを加えれば違うパターンになります。\par
初期値は1です。\\
\\
NBlur\par
ノイズ成分をぼかして、ドット感を減らします。\par
最小は0、最大は1です。\par
ドットに隣接するピクセルだけで計算しているので、非\par
常に軽いボカシの感じになります。\par
ゼロだとボカシはかからず、1.0で隣接するピクセルとの\par
平均をとります。\par
初期値は1です。\\
\\
Limits\par
明度(Lightness), 彩度(Saturation), 不透明度(Alpha)の、\par
端値(0 or 1付近)に対する効果調整をします。\par
ゼロや1付近でノイズをかけると、ゼロ以下あるいは1以上の値が\par
現われますが、表現できないので、それぞれゼロあるいは1に切り\par
詰められます。その切り詰めを補填する効果です。\par
--> "端値における Noise 効果調整 \ \ 図1 \ \ 比較"参照\par
--> "端値における Noise 効果調整 \ \ 図2 \ \ 説明"参照\\
\par
\noindent \ \ \, Effective\par
この効果(Limits)の強さを決めます。\par
ゼロならなにも効果はありません。ゼロより大きい値で効果が表\par
われます。1が最強です。\par
初期値はゼロです。\\
\par
\noindent \ \ \, Center\par
効果の中心です。\par
ノイズ範囲のずれ幅、あるいは、ノイズ幅の減少の効果は、ゼロ\par
あるいは1の端値の部分で最も強く、中心では効果がありません。

\newpage

\thispagestyle{empty}

\ \vspace{-0.2em}
\par
この効果のない中心値の位置をきめます。\par
ゼロから1の間で指定します。\par
ゼロだと、値がゼロの Pixel には効果がでなくなります。\par
1だと、値が1の Pixel には効果がでなくなります。\par
初期値は0.5が中心です。\\
\par
\noindent \ \ \, Type\par
効果のタイプを撰択します。\par
"Keep Noise"を選ぶと、(全体的に)ノイズ範囲をずらしノイズ幅\par
を維持、画像全体のコントラストは縮小します。\par
"Keep Contrast"を選ぶと、端のみでノイズ幅を減らしコントラス\par
トを維持します。\par
初期値は"Keep Noise"です。\\
\\
Premultiply\par
ON なら、RGB に対して Premultiply 済の\par
(Alpha チャンネルの値があらかじめ RGB チャンネルに乗算されている)\par
画像として処理します。\par
そのとき、Alpha にも処理を加えてしまうと、正しい画像にならない場合があります。\par
初期値は ON です。\\
\\
Reference\par
Pixel 毎に効果の強弱をつけるための参照画像の値の取り方を選択します。\par
入力の"Reference"に画像を接続し、\par
Red/Green/Blue/Alpha/Luminance/Nothing から選びます。\par
この効果をつけたくないときは Nothing を選ぶか、接続を切ります。\par
初期値は Red です。

\end{document}