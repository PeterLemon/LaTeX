\documentclass[a4paper,12pt]{article}
\usepackage[a4paper, total={147.8mm, 245mm}, left=34mm]{geometry}

\usepackage[colorlinks,linkcolor=blue,bookmarks,bookmarksopen,pdfauthor=fsanches]{hyperref}

\usepackage[T1]{fontenc}
\usepackage[portuguese,brazil]{babel}

\usepackage{soul}

\usepackage[normalem]{ulem}
\renewcommand{\ULthickness}{0.04em}

\usepackage[compact]{titlesec}
\titlespacing{\section}{0em}{*0}{0em}
\titlespacing{\subsection}{0em}{*0}{*0}
\titleformat{\section}{\ttfamily}{\thesection}{1em}{}

\setlength\parindent{5.8em}
\renewcommand{\baselinestretch}{1.2}

\usepackage{enumitem}

\usepackage{fancyhdr}
\pagestyle{fancy}
\fancyhf{}
\renewcommand\headrulewidth{0pt}

\frenchspacing

\begin{document}

\ttfamily
\sodef\an{}{0.05em}{0.6em}{0em}

\noindent \\

\vspace{2em}
\hspace{5em} \an{JOÃO JOSÉ NETO}\par
\hspace{7.2em} \an{``ASPECTOS DO PROJETO DE SOFTWARE DE UM MINICOMPUTADOR''}\par

\vspace{2em}
\hspace{5em} \an{``Dissertação de Mestrado'' apresentada à Escola Politécnica da Universidade de São Paulo, para obtenção do título de Mestre em Engenharia.}\par
\hspace{7.2em} \an{Área de Concentração - Engenharia de Eletricidade.}\par

\vspace{19em}
\hspace{4.45em} \an{Orientador: Prof. Dr. Antonio Marcos de Aguirra Massola}\par

\vspace{7em}
\hspace{10em} \an{São Paulo}\par
\hspace{10.3em} \an{-1975-}

\newpage

\noindent \\

\vspace{21em}
\hspace{6.7em} \an{A meus pais e irmã.}\par

\newpage

\setcounter{page}{1}
\fancyhead[R]{\ttfamily {I\hskip 0.05em .\hskip 0.05em \thepage}}

\renewcommand\contentsname{\ttfamily \hskip 14.5em \uline{I\hskip 0.05em N\hskip 0.05em D\hskip 0.05em I\hskip 0.05em C\hskip 0.05em E}\\
\\
\uline{A\hskip 0.05em s\hskip 0.05em s\hskip 0.05em u\hskip 0.05em n\hskip 0.05em t\hskip 0.05em o} \hskip 26em \uline{P\hskip 0.05em á\hskip 0.05em g\hskip 0.05em i\hskip 0.05em n\hskip 0.05em a}}
\tableofcontents

\newpage

\setcounter{page}{1}
\fancyhead[R]{\ttfamily {1\hskip 0.05em .\hskip 0.05em \thepage}}

\phantomsection
\section*{\uline{R\hskip 0.05em E\hskip 0.05em S\hskip 0.05em U\hskip 0.05em M\hskip 0.05em O}}

\noindent \\
\par
\an{O presente trabalho descreve os métodos utilizados no desenvolvimento de alguns dos módulos do ``software'' básico do Patinho Feio, o primeiro minicomputador desenvolvido no Laboratório de Sistemas Digitais do Departamento de Engenharia de Eletricidade da Escola Politécnica da Universidade de São Paulo.}\\

\par
\an{Para cada módulo apresentado, são discutidos os seus obsetivos, sendo, quando conveniente, apresentados alguns dos problemas enfrentados durante seu desenvolvimento específico ou então problemas mais gerais, referentes ao projeto de programas semelhantes ao mesmo. Sempre que possível, são apresentadas alternativas ao mesmo. Sempre que possível, são apresentadas alternativas de solução dos referidos problemas, descrevendo-se finalmente alguns detalhes de um exemplo de implementação.}\\

\par
\an{No Capítulo 2 são discutidos mais profundamente os problemas enfrentados durante o projeto e a implementação de um montador, desde a sua concepção até a sua implantação definitiva, enfatizando-se a metodologia empregada na implementação e os critérios adotados nas decisões mais importantes.}\\

\par
\an{No Capítulo 3 é descrito um simulador-interpretador, implementado em outro computador com a intenção de auxiliar o desenvolvimento do ``software'' básico do Patinho Feio. Algumas técnicas de simulação são discutidas neste Capítulo, ao lado dos algorítmos utilizados pelo interpretador.}\\

\par
\an{Nos demais Capítulos, são descritos mais alguns programas do ``software'' básico desenvolvidos para o Patinho Feio, tais como um desmontador, um programa para auxiliar a depuraçãode outros programas e um editor simbólico.}\\

\newpage

\par
\an{Nos apêndices, são apresentados tópicos julgados convenientes para a complementação de algumas idéias, bem como alguns exemplos de utilização de programas apresentados neste trabalho.}\\

\newpage

\par
\an{A B S T R A C T}\\

\par
\an{The present work describes the methods that had been employed during the development of some of the basic software modules of the ``Patinho Feio'', the first minicomputer of the ``Laboratório de Sistemas Digitais da Escola Politécnica da Universidade de São Paulo''.}\\

\par
\an{For each module, after a brief discussion of their objectives, some of the main problems which had to be solved during their development are presented, as well as more general ones, which arrive when designing programs of the same class. Whenever possible, alternative solutions of these problems are presented, and, at last, an example of implementation is described.}\\

\par
\an{Chapter 2 discusses in detail the problems arrived during the design and implementation of an assembler, from the phase of its conception until that of its final installation, emphasizing the methods employed in the implementation and the criteria which were used when the most important decisions had to be taken.}\\

\par
\an{Chapter 3 describes a simulator-interpreter, which had been implemented in an auxiliary computer, and was intended to help the development of the basic software of the ``Patinho Feio''. This chapter discusses also some simulation techniques and the algorithms employed in the interpreter.}\\

\par
\an{The remaining chapters describe some additional programs of the basic software developed for the ``Patinho Feio'', like a disassembler, a debugging routine and a symbolic editor.}\\

\newpage

\par
\an{The appendices present some subjects considered helpful for completing some ideas, and some execution examples of the programs presented in this work.}\\

\newpage





\end{document}
