\documentclass[a4paper,12pt]{article}
\usepackage[a4paper, total={180mm, 272mm}]{geometry}

\usepackage{fontspec}
\setmainfont[Path=fonts/, Extension=.ttf]{ipaexm}

\setlength\parindent{3.5em}
\setlength\parskip{0em}
\renewcommand{\baselinestretch}{1.247}

\begin{document}

\thispagestyle{empty}

\Large
\noindent \\
Density Ino\medskip
\par
\normalsize
半透明画像の濃度を調節します。\par
流線タッチ効果線を強調するために作成しました。\par
Alphaチャンネルに半透明値が必要です。\\
\\
-{-}- \ 入力 \ -{-}-\\
Source\par
処理をする画像を接続します。\\
Reference\par
Pixel 毎に効果の強弱をつけるための参照画像を接続します。\\
\\
-{-}- \ 設定 \ -{-}-\\
Density\par
濃度を指定します。\par
画像の半透明 Pixel に対して濃度が変化します。\par
ゼロ以上の値で指定します。最大は10です。\par
1より大きい値のとき、画像は濃くなります。\par
1のときは等濃度で変化しません。\par
1より小さくゼロより大きい値のとき、画像は薄くなります。\par
ゼロのときは濃度ゼロなので画像は消えます。\par
初期値は1です。\\
\\
Reference\par
Pixel 毎に効果の強弱をつけるための参照画像の値の取り方を選択します。\par
入力の\textquotedbl Reference\textquotedbl に画像を接続し、\par
Red/Green/Blue/Alpha/Luminance/Nothingから選びます。\par
この効果をつけたくないときは Nothingを選ぶか、接続を切ります。\par
初期値は Red です。

\end{document}