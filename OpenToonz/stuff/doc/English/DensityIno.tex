\documentclass[a4paper,12pt]{article}
\usepackage[a4paper, total={180mm, 272mm}]{geometry}

\usepackage{fontspec}
\setmainfont{ipaexm.ttf}

\setlength\parindent{4.2em}
\setlength\parskip{0em}
\renewcommand{\baselinestretch}{1.247}

\begin{document}

\thispagestyle{empty}

\Large
\noindent \\
Density Ino\medskip
\par
\normalsize
Adjusts the concentration of a semi-transparent image.\par
It was created in order to emphasize the streamlined touch effect for lines.\par
The Alpha channel requires semi-transparent values.\\
\\
--- \ Inputs \ ---\\
Source\par
Connect the image to process.\\
Reference\par
Connect the reference image to put the strength of the effect into each Pixel.\\
\\
--- \ Settings \ ---\\
Density\par
Specifies the concentration.\par
Concentration for the semi-transparent Pixel of the image will change.\par
Specify a value greater than or equal to 0. The maximum is 10.\par
When using a value greater than 1, the image will be darker.\par
It does not change at an equal concentration when above 1.\par
When using a value greater than 0 and smaller than 1, the image will be thinner.\par
When using a concentration value of 0 the image will disappear.\par
The default value is 1.\\
\\
Reference\par
Choose how Reference image values put the strength of the effect into each Pixel.\par
An image is connected to the "Reference" of the input,\par
Choose from Red/Green/Blue/Alpha/Luminance/Nothing.\par
Choose Nothing when you do not want this effect, it will turn off the connection.\par
The default setting is Red.

\end{document}