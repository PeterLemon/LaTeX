\documentclass[a4paper,12pt]{article}
\usepackage[a4paper, total={134.5mm, 245mm}, left=34mm]{geometry}

\usepackage[colorlinks,linkcolor=blue,bookmarks,bookmarksopen,pdfauthor=krom]{hyperref}

\usepackage[T1]{fontenc}
\usepackage[portuguese,brazil]{babel}

\usepackage[normalem]{ulem}
\renewcommand{\ULthickness}{0.04em}

\usepackage[compact]{titlesec}
\titlespacing{\section}{0em}{0em}{0em}
\titlespacing{\subsection}{2em}{0em}{1.5em}
\titleformat{\section}{\ttfamily}{\thesection}{0em}{}
\titleformat{\subsection}{\ttfamily}{\thesection}{0em}{}

\setlength\parindent{5.27em}
\setlength{\parskip}{0em}
\renewcommand{\baselinestretch}{1.2}

\usepackage{enumitem}

\usepackage{fancyhdr}
\pagestyle{fancy}
\fancyhf{}
\renewcommand\headrulewidth{0pt}

\usepackage{epic}
\usepackage{curves}

\frenchspacing

\begin{document}

\ttfamily

\noindent \\

\vspace{2em}
\hspace{5em} UNIVERSIDADE DE SÃO PAULO\par
\hspace{7em} ESCOLA POLITÉCNICA\par
\hspace{0.5em} DEPARTAMENTO DE ENGENHARIA DE ELETRICIDADE\par
\hspace{3em} LABORATÕRIO DE SISTEMAS DIGITAIS\par

\vspace{19em}
\hspace{4.5em} MONTADOR DO ``PATINHO FEIO''\par

\vspace{7em}
\hspace{7.5em} Antonio Marcos de Aguirra Massola\par
\hspace{7.5em} João José Neto\par
\hspace{7.5em} Moshe Bain\par

\vspace{7em}
\hspace{9.5em} Julho\par
\hspace{9.75em} 1977

\newpage

\noindent \\

\vspace{21em}
\hspace{6.5em} Em memória de\par
\hspace{6.5em} Laís Costa Ortenzi\par

\newpage

\setcounter{page}{1}
\fancyhead[R]{\ttfamily {I.\thepage}}

\renewcommand\contentsname{\ttfamily \hskip 13em \uline{INDICE}\\
\\
\uline{Assunto} \hskip 22.5em \uline{Página}}
\tableofcontents

\newpage

\setcounter{page}{1}
\setcounter{section}{1}
\fancyhead[R]{\ttfamily {\thesection \hskip 0.05em .\hskip 0.05em \thepage}}

\phantomsection
\section*{1 - \uline{INTRODUÇÃO}}
\addcontentsline{toc}{section}{CAPÍTULO 1 - INTRODUÇÃO}

\noindent \\
\par
O ante-projeto do minicomputador Patinho Feio nasceu\\
de um curso de pós-graduação dado pelo Professor Glen \hfill George\\
Langdon Jr., em 1972. A seguir, os engenheiros e estagiários do\\
Laboratório de Sistemas Digitais (LSD) da EPUSP terminaram \hfill o\\
projeto e montaram o Patinho Feio que, dessa forma, se \hfill tornou\\
o primeiro computador projetado e construído no Brasil.\\
\par
Os circuitos do Patinho Feio são totalmente \hfill consti-\\
tuídos por circuitos integrados da família TTL (``transistor \hfill -\\
transistor logic''), apresentando uma memória de núcleos de fe\uline r\\
rite, e tendo um ciclo de máquina de dois microsegundos.\\
\par
O Patinho Feio foi destinado a pesquisas no LSD, ta\uline n\\
to na área de programação (``software'') como dos circuitos ele-\\
trônicos (``hardware'').\\
\par
Cuidou-se do desenvolvimento de um ``software'' que pe\uline r\\
mitisse um uso mais eficiente do minicomputador, já que, \hfill de\\
início só se podia programá-lo em linguagem de máquina, manua\uline l\\
mente, através do seu painel. Em particular, foi definida \hfill uma\\
linguagem de montador (``assembly language''), que associa a ca-\\
da instrução de máquina um mnemônico, e um programa \hfill montador\\
(``assembler''), cuja função é traduzir programas escritos \hfill em\\
linguagem de montador para linguagem de máquina, os quais \hfill são\\
os assuntos tratados neste manual.\\
\par
Este manual foi escrito de forma a tratar cada tópi-\\
co de forma mais ou menos extensa, na suposição de que o \hfill lei-\\
tor tenha tido previamente apenas um pequeno contato com \hfill a\\
área de computação, e pouco ou nenhum conhecimento de \hfill lingu\uline a\\
gens de baixo nível, como um montador. Por causa disso,tentou-\\
se fazer com que o manual fosse o mais auto-explicativo e ind\uline e\\
pendente possível de outros textos. Naturalmente é \hfill impossível

\newpage

\noindent que um texto seja completamente independente de outros; por i\uline s\\
so, recomenda-se consultar outros textos, tais como manuais de\\
operação do Patinho Feio e de seus equipamentos periféricos(de\\
entrada/saída), textos sobre números binários, etc.\\
\par
Foi feito um bom esforço para apresentar os \hfill concei-\\
tos com clareza e para padronizar as notações, com o \hfill objetivo\\
de tornar o manual realmente útil. Contudo, certamente \hfill muitas\\
falhas subsistem, de forma que são bem recebidas quaisquer su-\\
gestões e críticas de modo a melhorar o manual em futuras edi-\\
ções.\\
\\
\\[-0.5em]
\uline{Observações}:

\begin{enumerate}[label=\alph*), align=left, leftmargin=1.5em, labelsep=-0.5em, itemsep=1em, topsep=1em]
\item As informações contidas neste manual são as melhores que se\\
pôde obter na época em que o manual foi escrito (setembro de\\
1975). Contudo, devido ao constante desenvolvimento de \hfill no-\\
vos projetos de ``hardware'' e ``software'' para o Patinho Feio,\\
alguns detalhes podem ter sofrido alterações até a presente\\
data.
\item Os programas e trechos de programa existentes no manual fo-\\
ram aí colocados por estarem sintaticamente corretos, \hfill mas\\
não representam necessariamente exemplos de boa técnica \hfill de\\
programação.
\end{enumerate}

\newpage

\setcounter{page}{1}
\setcounter{section}{2}

\phantomsection
\section*{2 - \uline{ARITMÉTICA BINÃRÍA E HEXADECIMAL}}
\addcontentsline{toc}{section}{CAPÍTULO 2 - ARITMÉTICA BINÃRÍA E HEXADECIMAL}

\hskip -3em (Com números inteiros)\\
\\[-0.5em]

\phantomsection
\subsection*{1. \uline{Bases de Numeração}}
\addcontentsline{toc}{subsection}{\hskip 5em Bases de Numeração}

Utiliza-se, na vida diária, a base decimal de numer\uline a\\
ção para representar os números. Isto significa duas coisas:

\begin{enumerate}[label=\alph*), align=left, leftmargin=1.5em, labelsep=-0.5em, itemsep=1em, topsep=1.5em]
\item existem dez algarismos com os quais todos os números são r\uline e\\
presentados (pois a base de numeração é dez), a saber: 0, 1,\\
2, 3, 4, 5, 6, 7, 8, 9.
\item emprega-se uma \uline{notação posicional} onde está subentendido que,\\
quando um algarismo é deslocado de uma posição para a esque\uline r\\
da, seu valor é multiplicado por dez. Por exemplo:\\[-0.5em]
\phantom \ \ \ \ \ \ \ \ \ \ \ \ 2 \ \ \ \ \ \ \ \ 1 \ \ \ \ \ \ \ \ 0\\[-1em]
295 = 2 x 10 \ + 9 x 10 \ + 5 x 10\\[-1.5em]
\end{enumerate}

Generalizando, quando se escreve o número N = d \ d \ \ ...\\[-1em]
\phantom \ \ \ \ \ \ \ \ \ \ \ \ \ \ \ \ \ \ \ \ \ \ \ \ \ \ \ \ \ \ \ \ \ \ \ \ \ \ \ \ \ \ \ \ \ \ \ \ \ \ \ \ \ \ \ \ \ n \ n-1\\[-0.5em]
d \ d \ d \ (sem sinal), onde os d \ (i= 0, 1, 2,...,n) são os seus\\[-1em]
\phantom \ 2 \ 1 \ 0 \ \ \ \ \ \ \ \ \ \ \ \ \ \ \ \ \ \ \ \ \ \ i\\[-0.5em]
algarismos (ou dígitos), está-se querendo dizer que: N = d \hfill x\\[-1em]
\phantom \ \ \ \ \ \ \ \ \ \ \ \ \ \ \ \ \ \ \ \ \ \ \ \ \ \ \ \ \ \ \ \ \ \ \ \ \ \ \ \ \ \ \ \ \ \ \ \ \ \ \ \ \ \ \ \ \ \ n \\[-1em]
\phantom \ \ n \ \ \ \ \ \ \ \ \ \ n-1 \ \ \ \ \ \ \ \ \ \ \ \ \ \ \ 2 \ \ \ \ \ \ \ \ \ 1 \ \ \ \ \ \ \ \ \ 0\\[-1em]
10 \ + d \ \ x 10 \ \ \ + ....+ d \ x 10 \ + d \ x 10 \ + d \ x 10 \ .\\[-1em]
\phantom \ \ \ \ \ \ \ n-1 \ \ \ \ \ \ \ \ \ \ \ \ \ \ \ \ 2 \ \ \ \ \ \ \ \ \ 1 \ \ \ \ \ \ \ \ \ 0\\
\par
Nada obriga a que se use apenas a base dez. Na verda-\\
de, qualquer base \uline b(inteira) pode ser escolhida para \ represen-\\
tar um número. Para tanto, escolhem-se \uline b símbolos distintos (os\\
algarismos da base) que representam os números de zero a (b - 1).\\
Escrevendo-se agora n + 1 algarismos adjacentes d d \ \ ...d d \ e\\[-1em]
\phantom \ \ \ \ \ \ \ \ \ \ \ \ \ \ \ \ \ \ \ \ \ \ \ \ \ \ \ \ \ \ \ \ \ \ \ \ \ \ \ \ \ \ \ \ \ \ \ \ \ n n-1 \ \ \ 1 0\\[-0.5em]
subentendida a notação posicional descrita acima, tem-se o núm\uline e\\
ro N representado por essa notação:\par
\ \ \ \ \ \ \ \ \ \ \ \ \ n \ \ \ \ \ \ \ \ n-1 \ \ \ \ \ \ \ \ \ \ 1 \ \ \ \ \ \ 0\\[-1em]
\phantom \ \ \ \ \ \ \ \ \ \ \ \ \ N = d \ . b \ + d \ \ \ b \ \ + ...+ d \ b \ + d \ b\\[-1em]
\phantom \ \ \ \ \ \ \ \ \ \ \ \ \ \ \ \ \ \ n \ \ \ \ \ \ \ \ n-1 \ \ \ \ \ \ \ \ \ \ \ \ 1 \ \ \ \ \ \ 0\\[-0.5em]
\par
Inversamente, pode-se provar que cada número N tem uma\\
única representação, numa dada base \uline b, que satisfaz as condições\\
mencionadas acima.\\
\par
Exemplo: Escolhendo b = 3, têm-se três algarismos;

\newpage

\noindent convencionalmente usa-se 0, 1, 2. Então tem-se:\\
\\[-1em]
\phantom \ \ \ \ \ \ \ \ \ \ \ \ \ \ \ \ \ \ \ \ \ 3 \ \ \ \ \ \ \ 2 \ \ \ \ \ \ \ 1 \ \ \ \ \ \ \ 0\\[-1em]
\phantom \ \ \ \ \ \ (1202) \ = 1 x 3 \ + 2 x 3 \ + 0 x 3 \ + 2 x 3 \ = (47)\\[-1em]
\phantom \ \ \ \ \ \ \ \ \ \ \ \ 3 \ \ \ \ \ \ \ \ \ \ \ \ \ \ \ \ \ \ \ \ \ \ \ \ \ \ \ \ \ \ \ \ \ \ \ \ \ \ \ \ \ \ 10\\
\par
Pode-se começar a perceber a importância do que \hfill foi\\
dito acima quando se considera que os computadores modermos tr\uline a\\
balham sempre, em última análise, com a base dois.\\
\\[0.5em]

\phantomsection
\subsection*{2. \uline{Bases mais empregadas em computação}}
\addcontentsline{toc}{subsection}{\hskip 5em Bases mais empregadas em computação}

Além da base dez, que é de uso geral, empregam-se co-\\
mumente as seguintes bases:

\begin{enumerate}[label=\alph*), align=left, leftmargin=1.5em, labelsep=-0.5em, itemsep=1em, topsep=1.5em]
\item base dois (binária) - necessita dois algarismos distintos p\uline a\\
ra representar os números zero e um. Por convenção utilizam-\\
se os símbolos 0 e 1, Um algarismo binário é também \ chamado\\
``bit'' (do inglês ``binary digit'').\\
\\
A base dois é extremamente importante pois, como já foi cit\uline a\\
do, os computadores só entendem sequências de zeros e uns \ ,\\ 
que são usadas tanto para representar as instruções dadas \ à\\
máquina quanto números prop iamente ditos.
\item base oito (octal) - utiliza os algarismos de 0 a 7. Não será\\
aqui tratada com mais detalhes porque não é utilizada no Pa-
tinho Feio, embora o seja em vários outros computadores.
\item base dezesseis (hexadecimal) - os dígitos hexadecimais \ são:\\
0, 1, 2, 3, 4, 5, 6, 7, 8, 9, A, B, C, D, E, F; usados \ para\\
representar os números de zero a quinze.
\end{enumerate}

\noindent \\[-3em]
\phantom \ \ \ \ \ \ \ \ \ \ \ \ \ \ \ \ \ \ \ \ \ \ \ \ \ \ \ \ 1 \ \ \ \ \ \ \ \ \ 0\\[-1em] 
\phantom \ \ \ Exemplo: (AB) \ \ = 10 x 16 \ + 11 x 16 \ = (171)\\[-1em]
\phantom \ \ \ \ \ \ \ \ \ \ \ \ \ \ \ \ 16 \ \ \ \ \ \ \ \ \ \ \ \ \ \ \ \ \ \ \ \ \ \ \ \ \ \ \ \ \ 10\\[-0.5em]
\par
A correspondência entre os valores binários, decimais\\
e hexadecimais é apresentada na tabela seguinte (note-se que são\\
necessários quatro bits para representar todos os dígitos \ hex\uline a\\
decimais na base dois).

\newpage

\ \ \ \uline{Decimal} \ \ \ \uline{Hexadecimal} \ \ \ \uline{Binário}\\[-1em]

\ \ \ \ \ \ 0 \ \ \ \ \ \ \ \ \ \ 0 \ \ \ \ \ \ \ \ \ \ 0000\par
\ \ \ \ \ \ 1 \ \ \ \ \ \ \ \ \ \ 1 \ \ \ \ \ \ \ \ \ \ 0001\par
\ \ \ \ \ \ 2 \ \ \ \ \ \ \ \ \ \ 2 \ \ \ \ \ \ \ \ \ \ 0010\par
\ \ \ \ \ \ 3 \ \ \ \ \ \ \ \ \ \ 3 \ \ \ \ \ \ \ \ \ \ 0011\par
\ \ \ \ \ \ 4 \ \ \ \ \ \ \ \ \ \ 4 \ \ \ \ \ \ \ \ \ \ 0100\par
\ \ \ \ \ \ 5 \ \ \ \ \ \ \ \ \ \ 5 \ \ \ \ \ \ \ \ \ \ 0101\par
\ \ \ \ \ \ 6 \ \ \ \ \ \ \ \ \ \ 6 \ \ \ \ \ \ \ \ \ \ 0110\par
\ \ \ \ \ \ 7 \ \ \ \ \ \ \ \ \ \ 7 \ \ \ \ \ \ \ \ \ \ 0111\par
\ \ \ \ \ \ 8 \ \ \ \ \ \ \ \ \ \ 8 \ \ \ \ \ \ \ \ \ \ 1000\par
\ \ \ \ \ \ 9 \ \ \ \ \ \ \ \ \ \ 9 \ \ \ \ \ \ \ \ \ \ 1001\par
\ \ \ \ \ 10 \ \ \ \ \ \ \ \ \ \ A \ \ \ \ \ \ \ \ \ \ 1010\par
\ \ \ \ \ 11 \ \ \ \ \ \ \ \ \ \ B \ \ \ \ \ \ \ \ \ \ 1011\par
\ \ \ \ \ 12 \ \ \ \ \ \ \ \ \ \ C \ \ \ \ \ \ \ \ \ \ 1100\par
\ \ \ \ \ 13 \ \ \ \ \ \ \ \ \ \ D \ \ \ \ \ \ \ \ \ \ 1101\par
\ \ \ \ \ 14 \ \ \ \ \ \ \ \ \ \ E \ \ \ \ \ \ \ \ \ \ 1110\par
\ \ \ \ \ 15 \ \ \ \ \ \ \ \ \ \ F \ \ \ \ \ \ \ \ \ \ 1111\\

\phantomsection
\subsection*{3. \uline{Conversão de números entre as bases dois, dez e \ \ dezes-}\\
\phantom \ \ \ \uline{seis}}
\addcontentsline{toc}{subsection}{\hskip 5em Conversão entre as bases dois, dez \ e\\ dezes seis}

Conforme já se deve ter percebido, surge frequente -\\
mente a necessidade de converter números escritos em uma \ base\\
psta outra. Para isso existem algoritmos gerais, dos quais são\\
apresentados abaixo alguns casos particulares:

\begin{enumerate}[label=\alph*), align=left, leftmargin=1.5em, labelsep=-0.5em, itemsep=1em, topsep=1.5em]
\item Conversão para a base dez de números escritos em outra base.\\[-0.5em]
\phantom \ \ \ \ \ \ \ \ \ \ \ \ \ \ \ \ \ \ \ \ \ \ \ \ \ \ \ \ \ \ \ \ \ \ \ \ \ \ \ n\\[-1em]
Basta escrever o número na forma d \ . b \ + ...+ d \ e efetuar\\[-1em]
\phantom \ \ \ \ \ \ \ \ \ \ \ \ \ \ \ \ \ \ \ \ \ \ \ \ \ \ \ \ \ \ \ \ \ \ n \ \ \ \ \ \ \ \ \ \ \ \ \ 0\\[-0.5em]
as operações indicadas.\\[0.5em]
\phantom \ \ \ \ \ \ \ \ \ \ \ \d{o}\\[-1.2em]
Exemplos: 1 ) \ (101111100001) \ para a base 10\\[-1em]
\phantom \ \ \ \ \ \ \ \ \ \ \ \ \ \ \ \ \ \ \ \ \ \ \ \ \ \ \ \ \ 2\\[-0.5em]
\phantom \ \ \ \ \ \ \ \ \ \ \ \ \ \ \ \ \ \ \ \ 11 \ \ \ \ \ 10 \ \ \ \ \ \ \ \ \ \ 1\\[-1em]
\phantom \ \ \ \ \ \ \ \ \ \ \ \ \ \ \ = 1.2 \ \ + 0.2 \ \ + ...+ 0.2 \ + 1 = \ (3041)\\[-1em]
\phantom \ \ \ \ \ \ \ \ \ \ \ \ \ \ \ \ \ \ \ \ \ \ \ \ \ \ \ \ \ \ \ \ \ \ \ \ \ \ \ \ \ \ \ \ \ \ \ \ \ \ \ \ \ \ \ \ 10\\[1em]
\phantom \ \ \ \ \ \ \ \ \ \ \ \d{o}\\[-1.2em]
\phantom \ \ \ \ \ \ \ \ \ \ 2 ) \ (BE1) \ \ para a base 10\\[-1em]
\phantom \ \ \ \ \ \ \ \ \ \ \ \ \ \ \ \ \ \ \ \ 16\\
\phantom \ \ \ \ \ \ \ \ \ \ \ \ \ \ \ \ \ \ \ \ \ \ 2\\[-1em]
\phantom \ \ \ \ \ \ \ \ \ \ \ \ \ = 11 x 16 \ + 14 x 16 + 1 = \ (3041)\\[-1em]
\phantom \ \ \ \ \ \ \ \ \ \ \ \ \ \ \ \ \ \ \ \ \ \ \ \ \ \ \ \ \ \ \ \ \ \ \ \ \ \ \ \ \ \ \ \ \ \ \ 10\\[1em]

\newpage

\noindent Uma forma conveniente de fazer isso é dada \ nos \ diagramas \\
abaixo:\\\\[-1em]
\phantom \ \ \ \ \ \ \ \ \ \ \ \ \ \ (BE1) \ \ para a base 10\\[-1em]
\phantom \ \ \ \ \ \ \ \ \ \ \ \ \ \ \ \ \ \ \ 16\\[-0.5em]

\noindent \ \ \ \ \ \ \ \ \uline{\ \ \ \ \ \ 11 \ \ 14 \ \ \ 1 \ \ \ \ \ }\\[-0.5em]
base\\[-1em]
\phantom \ \ \ \ \ \ \ \ \ \ \ \ x\\[-1em]
original=16 \ \ 11 \ \ 190 \ (3041)\\[-1em]
\phantom \ \ \ \ \ \ \ \ \ \ \ \ \ \ \ \ \ \ \ \ \ \ \ \ \ \ \ \ \ \ 10

\begin{picture}(0,0)
\put(92.5,46){\vector(0,-1){10}}
\put(84.5,30){\vector(-1,0){15}}
\end{picture}\\[-2em]

\noindent Método usado: 11 x 16 + 14 = 190\\[1em]
\phantom \ \ \ \ \ \ \ \ \ \ \ \ \ \ \ \ \ \ \ \ \ \ \ \ \ \ \ \ \ 190 x 16 + 1 = 3041

\begin{picture}(0,0)
\put(188,48){\vector(0,-1){15}}
\end{picture}\\[-3em]

\phantom \ \ \ \ \ \ \ \ \ \ \ \ \ \ \ (10110) \ para a base 10\\[-1em]
\phantom \ \ \ \ \ \ \ \ \ \ \ \ \ \ \ \ \ \ \ \ \ 2\\[-1em]

\noindent \ \ \ \ \ \ \ \ \uline{\ \ \ \ \ \ 1 \ \ 0 \ \ 1 \ \ 1 \ \ 0 \ \ }\\
base\\[-0.5em]
original= 2 \ \ 1 \ \ 2 \ \ 5 \ 11 (22)\\[-1em]
\phantom \ \ \ \ \ \ \ \ \ \ \ \ \ \ \ \ \ \ \ \ \ \ \ \ \ \ \ \ \ \ \ \ 10

\begin{picture}(0,0)
\put(89.25,53){\vector(0,-1){15}}
\end{picture}\\[-2.5em]

Método usado: 1\hskip 0.25em x\hskip 0.25em 2\hskip 0.25em +\hskip 0.25em 0 = 2 \ \ \ \ \ \ \ 5\hskip 0.25em x\hskip 0.25em 2\hskip 0.25em +\hskip 0.25em 1\hskip 0.25em =\hskip 0.25em 11\\
\\[-0.5em]
\phantom \ \ \ \ \ \ \ \ \ \ \ \ \ \ \ \ \ \ \ \ \ \ \ \ 2\hskip 0.25em x\hskip 0.25em 2\hskip 0.25em +\hskip 0.25em 1\hskip 0.25em =\hskip 0.25em 5 \ \ \ \ \ \ \ 11\hskip 0.25em x\hskip 0.25em 2\hskip 0.25em +\hskip 0.25em 0 = 22

\begin{picture}(0,0)
\put(151,48){\vector(0,-1){15}}
\put(206.5,33){\vector(0,1){15}}
\put(265.5,48){\vector(0,-1){15}}
\end{picture}\\[-2.5em]

\item Conversão de números escritos na base dez para uma outra b\uline a\\
se. Divide-se \ repetidamente o número dado pela base de de\uline s\\
tino até que o quociente seja zero. Os restos obtidos \hfill são\\
a representação desejada, em ordem invertida. Ver os esque-\\
mas abaixo:\\[-2.5em]

\phantom \ \ \ \ \ \ \ \ \ \ \ \ (3041) \ \ \ \ \ \ base 16\\[-1em]
\phantom \ \ \ \ \ \ \ \ \ \ \ \ \ \ \ \ \ 10\\
\phantom \ \ \ \ \ \ \ \ \ \ \ \ \ \ \ \ \ \ \ \ \ \ \ \ \ \ \ \ \ \ \ \ resto de 3041 : 16\\
\phantom \ \ \ \ \ \ \ \ \ \ \ \ \ \ \ \ \ \ \ \ \ \ \ \ \ \ \ \ \ \ \ \ resto de \ 190 : 16\\[-1em]
\phantom \ \ \ \ \ \ \ \ \ \ \ \ \ \ \ \ \ 3041 \ \ \ 1\\[-0.5em]
\phantom \ \ \ \ \ \ \ \ \ \ \ \ \ \ \ \ \ \ \ \ \ \ \ \ \ \ \ \ \ \ \ \ resto de \ \ 11 : 16\\[-0.5em]
\phantom \ \ 3041 : 16 \ \ \ \ \ \ 190 \ \ 14\\[0.5em]
\phantom \ \ \ 190 : 16 \ \ \ \ \ \ \ 11 \ \ 11 \ 14 \ 1\\[1em]
\phantom \ \ \ \ 11 : 16 \ \ \ \ \ \ \ \ 0 \ \ \ (BE1)\\[-1em]
\phantom \ \ \ \ \ \ \ \ \ \ \ \ \ \ \ \ \ \ \ \ \ \ \ \ \ \ \ \ \ \ 16\\

\begin{picture}(0,0)
\put(143,28){\line(0,1){110}}
\put(75,101){\vector(1,0){30}}
\put(75,78){\vector(1,0){30}}
\put(75,48){\vector(1,0){30}}

\dashline{4}(163,125)(194,142)
\put(163,125){\vector(-2,-1){2}}

\dashline{4}(163,103)(194,124)
\put(163,103){\vector(-2,-1){2}}

\dashline{4}(163,80)(194,108)
\put(163,80){\vector(-1,-1){2}}

\put(162,120){\vector(1,-1){35}}
\put(162,98){\vector(1,-1){13}}
\put(155,72){\vector(1,-3){6}}
\put(179,72){\vector(-1,-2){9}}
\put(196,72){\vector(-1,-1){18}}
\end{picture}

\newpage

\ \ \ \ \ \ \ \ \ \ \ \ \ \ (3041) \ \ \ \ \ base 2\\[-1em]
\phantom \ \ \ \ \ \ \ \ \ \ \ \ \ \ \ \ \ \ \ \ 10\\[-0.5em]

\ \ \ \ \ 3041 \ \ 1\\[-1.8em]
\par
\ \ \ \ \ 1524 \ \ 0\\[-1.8em]
\par
\ \ \ \ \ \ 760 \ \ 0\\[-1.8em]
\par
\ \ \ \ \ \ 380 \ \ 0\\[-1.8em]
\par
\ \ \ \ \ \ 190 \ \ 0\\[-1.8em]
\par
\ \ \ \ \ \ \ 95 \ \ 1\\[-1.8em]
\par
\ \ \ \ \ \ \ 47 \ \ 1\\[-1.8em]
\par
\ \ \ \ \ \ \ 23 \ \ 1\\[-1.8em]
\par
\ \ \ \ \ \ \ 11 \ \ 1\\[-1.8em]
\par
\ \ \ \ \ \ \ \ 5 \ \ 1\\[-1.8em]
\par
\ \ \ \ \ \ \ \ 2 \ \ 0\\[-1.8em]
\par
\ \ \ \ \ \ \ \ 1 \ (1 0 1 1 1 1 1 0 0 0 0 1)\\[-1em]
\phantom \ \ \ \ \ \ \ \ \ \ \ \ \ \ \ \ \ \ \ \ \ \ \ \ \ \ \ \ \ \ \ \ \ \ \ \ 2\\[-2.2em]
\par
\ \ \ \ \ \ \ \ 0

\begin{picture}(0,0)
\put(66,8){\line(0,1){250}}
\put(80,228){\vector(3,-4){133}}
\put(80,210){\vector(3,-4){120}}
\put(80,193){\vector(3,-4){107}}
\put(80,175){\vector(3,-4){93}}
\put(80,158){\vector(3,-4){80}}
\put(80,141){\vector(3,-4){68}}
\put(80,124){\vector(3,-4){55}}
\put(80,107){\vector(3,-4){42}}
\put(80,91){\vector(3,-4){30}}
\put(80,74){\vector(3,-4){17}}
\put(80,59){\vector(3,-4){8}}
\end{picture}\\[-2em]

\item Conversão entre as bases dois e dezesseis.

\begin{enumerate}[label=c-\arabic*), align=left, leftmargin=2.5em, labelsep=-0.1em, itemsep=1em, topsep=0em]
\item Da base dois para a base dezesseis. Basta agrupar \hfill os\\
dígitos binários de quatro em quatro (a partir da \hfill di-\\
reita) e substituí-los pelo respectivo dígito hexadec\uline i\\
mal, conforme a tabela apresentada mais atrás (item 2.\\
c).\\[0.5em]
Exemplo:\\[-0.5em]
\par
\ \ \ \ \ \ \ \ \ \ \ \ \ \ \ \ (1011110101) \ para base 16\\[-1em]
\phantom \ \ \ \ \ \ \ \ \ \ \ \ \ \ \ \ \ \ \ \ \ \ \ \ \ \ \ \ 2\\[-0.5em]
\phantom \ \ \ zeros\\[-0.5em]
\phantom \ \ \ adicionados\\[-0.5em]
\phantom \ \ \ \ \ \ \ \ \ \ \ \ \ \ \ \ \uline{0010} \uline{1111} \uline{0101}\\[-0.5em]
\par
\ \ \ \ \ \ \ \ \ \ \ \ \ \ \ \ \ \ \ (2F5)\\[-1em]
\phantom \ \ \ \ \ \ \ \ \ \ \ \ \ \ \ \ \ \ \ \ \ \ \ \ 16

\begin{picture}(0,0)
\put(87,76){\line(1,0){21}}
\put(87,62){\line(0,1){28}}
\put(80,62){\line(1,0){7}}
\put(80,90){\line(1,0){7}}
\put(101.5,76){\vector(0,-1){10}}
\put(108,76){\vector(0,-1){10}}
\put(111,53){\vector(2,-3){12}}
\put(142,53){\vector(-1,-2){9}}
\put(173,53){\vector(-3,-2){28}}
\end{picture}\\[-2em]

\item Da base dezesseis para a base dois. Basta substituir c\uline a\\
da dígito hexadecimal pelo seu código de quatro \hfill bits.\\[0.5em]
Exemplo:

\newpage

\ \ \ \ \ \ \ \ \ (2F5) \ \ para base 2\\[-1em]
\phantom \ \ \ \ \ \ \ \ \ \ \ \ \ \ 16\\
\\
\\[-0.5em]
\phantom \ \ \ \ (0010 1111 0101)\\[-1em]
\phantom \ \ \ \ \ \ \ \ \ \ \ \ \ \ \ \ \ \ \ \ 2\\

\begin{picture}(0,0)
\put(62,90){\vector(-1,-2){18}}
\put(71,90){\vector(0,-1){36}}
\put(80,90){\vector(2,-3){24}}
\end{picture}\\[-5em]

\phantom \ \ \ \ \ \ \ \ \d{o}\\[-1.2em]
0bs.: 1 ) Para a base oito, como é fácil perceber,o m\uline é\\
\phantom \ \ \ \ \ \ \ \ \ \ todo é inteiramente análogo, dividindo-se \hfill o\\
\phantom \ \ \ \ \ \ \ \ \ \ número binário em grupos de três bits.\\[1.2em]
\phantom \ \ \ \ \ \ \ \d{o}\\[-1.2em]
\phantom \ \ \ \ \ \ 2 ) Pode-se agora notar porque são tão usadas as\\
\phantom \ \ \ \ \ \ \ \ \ \ bases oito e dezesseis em computação: \hfill elas\\[-0.5em]
\phantom \ \ \ \ \ \ \ \ \ \ \ \ \ \ \ \ \ \ \ \ \ \ \ \ \ \ \ \ \ \ \ \ \ \ \ \ \ \ \ \ \ \ \ \ \ \ \ 3\\[-1em]
\phantom \ \ \ \ \ \ \ \ \ \ permitem dividir por três (pois 8 = 2 )e por\\[-0.5em]
\phantom \ \ \ \ \ \ \ \ \ \ \ \ \ \ \ \ \ \ \ \ \ \ \ \ \ \ \ \ \ 4\\[-1em]
\phantom \ \ \ \ \ \ \ \ \ \ quatro (pois 16 = 2 ), respectivamente, \hfill o\\
\phantom \ \ \ \ \ \ \ \ \ \ comprimento em algarismos do número \hfill escrito\\
\phantom \ \ \ \ \ \ \ \ \ \ na base dois, que costuma ser inconveniente-\\
\phantom \ \ \ \ \ \ \ \ \ \ mente longo.\\[1.2em]
\phantom \ \ \ \ \ \ \ \d{o}\\[-1.2em]
\phantom \ \ \ \ \ \ 3 ) Estã-se dando mais ênfase à base hexadecimal\\
\phantom \ \ \ \ \ \ \ \ \ \ porque no Patinho Feio os números têm ou oi-\\
\phantom \ \ \ \ \ \ \ \ \ \ to ou doze bits de comprimento, podendo \hfill en-\\
\phantom \ \ \ \ \ \ \ \ \ \ tão, ser representados com dois ou três \hfill dí-\\
\phantom \ \ \ \ \ \ \ \ \ \ gitos hexadecimais, enquanto que, por \hfill exem-\\
\phantom \ \ \ \ \ \ \ \ \ \ plo, para transformar um número de oito bits\\
\phantom \ \ \ \ \ \ \ \ \ \ (também chamado ``byte'') em um número octal ,\\
\phantom \ \ \ \ \ \ \ \ \ \ tem-se que adicionar um zero à frente do nú-\\
\phantom \ \ \ \ \ \ \ \ \ \ mero, para dividí-lo em três grupos de \hfill três\\
\phantom \ \ \ \ \ \ \ \ \ \ bits cada.\\
\end{enumerate}
\end{enumerate}

\phantomsection
\subsection*{4. \uline{Soma de números binários positivos}}
\addcontentsline{toc}{subsection}{\hskip 5em Soma de números binários positivos}

Realiza-se de forma inteiramente análoga à soma \ so-\\
mum, de números decimais. Para ver isso, examine-se detalhada-\\
mente uma soma decimal, por exemplo, de 1672 com 729.

\newpage

\hskip 2.5em 0 \ \ \ 1 \ \ 1 \ \ 1\\
\\[-0.5em]
\phantom \ \ \ \ \ \ \ \ \ \ \ \ \ \ \ \ \ \ 1 \ \ 6 \ \ 7 \ \ 2\\[-0.5em]
\phantom \ \ \ \ \ \ \ \ \ \ \ \ \ \ \ \ \ + \ \ + \ \ + \ \ +\\[-0.5em]
\phantom \ \ \ \ \ \ \ \ \ \ \ \ \ \ \ \ \ \ \ \ \ \ \ \ \ \ \ \ \ \ \ \ \ +\\[-0.5em]
\phantom \ \ \ \ \ \ \ \ \ \ \ \ \ \ \ \ \ \ \ \ \ \ 7 \ \ 2 \ \ 9\\
\\[-0.5em]
\phantom \ \ \ \ \ \ \ \ \ \ \ \ \ 0 \ \ \ 2 \ \ 4 \ \ 0 \ \ 1\\

\begin{picture}(0,0)
\put(30,122){\line(-1,-2){9}}

\put(68,122){\line(1,-2){6}}
\put(93,122){\line(1,-2){6}}
\put(118,122){\line(1,-2){6}}

\put(68,122){\line(-1,0){4}}
\put(93,122){\line(-1,0){4}}
\put(118,122){\line(-1,0){4}}

\put(61,122){\vector(-1,-2){6}}
\put(86,122){\vector(-1,-2){6}}
\put(111,122){\vector(-1,-2){6}}

\put(21,104){\vector(0,-1){49}}
\put(52.5,92){\vector(0,-1){37}}
\put(76.5,92){\vector(0,-1){16}}
\put(101.5,92){\vector(0,-1){16}}
\put(126.5,92){\vector(0,-1){16}}

\put(18,52){\line(1,0){112}}
\end{picture}\\[-3em]

Começando a partir da direita, foram realizadas \hfill as\\
seguintes operações:\\[0.5em]
\phantom \ \ \ \ \ \ \ \ \ \ \ \ \ \ 2 + 9 = 1 \ e \ vai-um \ (= 11)\\
\phantom \ \ \ \ \ \ \ \ \ \ 1 + 7 + 2 = 0 \ e \ vai-um \ (= 10)\\
\phantom \ \ \ \ \ \ \ \ \ \ 1 + 6 + 7 = 4 \ e \ vai-um \ (= 14)\\
\phantom \ \ \ \ \ \ \ \ \ \ \ \ \ \ 1 + 1 = 2 \ e \ vai-zero(= 02)\\
\phantom \ \ \ \ \ \ \ \ \ \ \ \ \ \ \ \ \ \ 0 = 0\\
\par
Com os números binários procede-se da mesma forma,s\uline e 
gundo as seguintes regras:\\
\\
\phantom \ \ \ \ \ \ \ \ \ \ 0 + 0 + 0 \ = 00 \ \ \ \ \ \ \ \ 0 \ e \ vai-um\\
\phantom \ \ \ \ \ \ \ \ \ \ 0 + 0 + 1 \ = 01 \ \ \ \ \ \ \ \ 1 \ e \ vai-zero\\
\phantom \ \ \ \ \ \ \ \ \ \ 0 + 1 + 1 \ = 10 \ \ \ \ \ \ \ \ 0 \ e \ vai-um\\
\phantom \ \ \ \ \ \ \ \ \ \ 1 + 1 + 1 \ = 11 \ \ \ \ \ \ \ \ 1 \ e \ vai-um\\

\begin{picture}(0,0)
\put(106,91){\vector(1,0){36}}
\put(106,73){\vector(1,0){36}}
\put(106,55){\vector(1,0){36}}
\put(106,38){\vector(1,0){36}}
\end{picture}\\[-3em]

Exemplo:\\
\phantom \ \ \ \ \ \ \ \ \ \ \ \d{o}\\[-1.2em]
\phantom \ \ \ \ \ \ \ \ \ \ 1 ) Seja somar \ 10111010 com 10011. Tem-se:\\
\\
\\
\phantom \ \ \ \ \ \ \ \ \ \ \ \ \ \ \ 0 \ \ 0 \ \ 1 \ \ 1 \ \ 0 \ \ 0 \ \ 1 \ \ 0\\[0.5em]
\phantom \ \ \ \ \ \ \ \ \ \ \ \ \ \ \ \ \ 1 \ \ 0 \ \ 1 \ \ 1 \ \ 1 \ \ 0 \ \ 1 \ \ 0\\
\phantom \ \ \ \ \ \ \ \ \ \ \ \ \ \ \ \ \ \ \ \ \ \ \ \ \ \ \ \ \ \ \ \ \ \ \ \ \ \ \ \ \ \ \ \ \ \ \ \ +\\[-1em]
\phantom \ \ \ \ \ \ \ \ \ \ \ \ \ \ \ \ \ \ \ \ \ \ \ \ \ \ \ \ \ 1 \ \ 0 \ \ 0 \ \ 1 \ \ 1\\[-0.5em]
\phantom \ \ \ \ \ \ \ \ \ \ \ \ \ \ \ \ \ \ \ \ \ \ \ \ \ \ .\\[-0.5em]
\phantom \ \ \ \ \ \ \ \ \ \ \ \ \ \ \ 0 1 \ \ 1 \ \ 0 \ \ 0 \ \ 1 \ \ 1 \ \ 0 \ \ 1

\begin{picture}(0,0)
\dashline{4}(33,75)(33,38)
\put(33,38){\vector(0,-1){0}}

\curve(33,75, 38,77.5, 43,75)
\curve(46,75, 57,80, 68,75)
\curve(71,75, 82,80, 93,75)
\curve(96,75, 107,80, 118,75)
\curve(121,75, 132,80, 143,75)
\curve(146,75, 157,80, 168,75)
\curve(171,75, 182,80, 193,75)
\curve(196,75, 207,80, 218,75)

\put(46,75){\vector(-3,-2){0}}
\put(71,75){\vector(-3,-2){0}}
\put(96,75){\vector(-3,-2){0}}
\put(121,75){\vector(-3,-2){0}}
\put(146,75){\vector(-3,-2){0}}
\put(171,75){\vector(-3,-2){0}}
\put(196,75){\vector(-3,-2){0}}

\put(30,34){\line(1,0){193}}
\end{picture}\\

\par
Começa-se a partir da direita, realizando as seguin-\par
tes operações:

\newpage

\end{document}