\documentclass[a4paper,12pt]{article}
\usepackage[a4paper, total={180mm, 272mm}]{geometry}

\usepackage{fontspec}
\setmainfont{ipaexm.ttf}

\setlength\parindent{3.5em}
\setlength\parskip{0em}
\renewcommand{\baselinestretch}{1.247}

\begin{document}

\thispagestyle{empty}

\Large
\noindent \\
HSV Adjust Ino\medskip
\par
\normalsize
Hue (H), Saturation (S), Brightness (V), is multiplied by the Scale, then Shifted.\\
\\
-{-}- \ Inputs \ -{-}-\\
Source\par
Connect the image to process.\\
Reference\par
Connect the reference image to put the strength of the effect into each Pixel.\\
\\
-{-}- \ Settings \ -{-}-\\
Pivot\par
\noindent \ \ \, Specify the central value of where to apply the Scale.\\
\par
\noindent \ \ \, Hue\par
Specify the central value of the Scale for the color (Hue).\par
Minimum value is 0.0, maximum value is 360.0.\par
The default value is 0.0.\par
\noindent \ \ \, Saturation\par
Specify the central value of the Scale for the chroma (Saturation).\par
Minimum value is 0.0, maximum value is 1.0.\par
The default value is 0.0.\par
\noindent \ \ \, Value\par
Specify the central value of the Scale for the brightness (Value).\par
Minimum value is 0.0, maximum value is 1.0.\par
The default value is 0.0.\\
\\
Scale\par
\noindent \ \ \, Multiplies the Scale to enlarge or reduce the range of HSV around the Pivot value.\par
\noindent \ \ \, Hue value will be applied recursively around the top of the circle,\par
\noindent \ \ \, Saturation, Value value is greater than or equal to 0, and can go up to 1.\\
\par
\noindent \ \ \, Hue\par
Multiplied by the Scale for the color (Hue).\par
The minimum value is 0.0.\par
The default value is 1.0.\par
\noindent \ \ \, Saturation\par
Multiplied by the Scale for the chroma (Saturation).\par
The minimum value is 0.0.\par
The default value is 1.0.\\
\noindent \ \ \, Value\par
Multiplied by the Scale for the brightness (Value).

\newpage

\thispagestyle{empty}

\ \vspace{-0.2em}
\par
The minimum value is 0.0.\par
The default value is 1.0.\\
\\
Shift\par
\noindent \ \ \, This will Shift the values of the HLS.\par
\noindent \ \ \, Hue value will be multiplied recursively around the top of the circle,\par
\noindent \ \ \, Saturation, Value value is greater than or equal to 0, and can go up to 1.\\
\par
\noindent \ \ \, Hue\par
Shift for the color (Hue).\par
The default value is 0.0.\par
\noindent \ \ \, Saturation\par
Shift for the chroma (Saturation).\par
The default value is 0.0.\par
\noindent \ \ \, Value\par
Shift for the brightness (Value).\par
The default value is 0.0.\\
\\
Premultiplied\par
When ON, RGB is already set to Premultiply\par
(The value of Alpha channel is multiplied in advance RGB channels)\par
and processes as an image.\par
The default setting is ON.\\
\\
Reference\par
Choose how Reference image values put the strength of the effect into each Pixel.\par
An image is connected to the \textquotedbl Reference\textquotedbl \ of the input,\par
Choose from Red/Green/Blue/Alpha/Luminance/Nothing.\par
Choose Nothing when you do not want this effect, it will turn off the connection.\par
The default setting is Red.

\end{document}