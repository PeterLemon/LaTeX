\documentclass[a4paper,12pt]{article}
\usepackage[a4paper, total={134.5mm, 245mm}, left=34mm]{geometry}

\usepackage[colorlinks,linkcolor=blue,bookmarks,bookmarksopen,pdfauthor=krom]{hyperref}

\usepackage[T1]{fontenc}
\usepackage[portuguese,brazil]{babel}

\usepackage[normalem]{ulem}
\renewcommand{\ULthickness}{0.04em}

\usepackage[compact]{titlesec}
\titlespacing{\section}{0em}{0em}{0em}
\titlespacing{\subsection}{2em}{0em}{1.5em}
\titleformat{\section}{\ttfamily}{\thesection}{0em}{}
\titleformat{\subsection}{\ttfamily}{\thesection}{0em}{}

\setlength\parindent{5.27em}
\setlength{\parskip}{0em}
\renewcommand{\baselinestretch}{1.2}

\usepackage{setspace}

\usepackage{enumitem}

\usepackage{fancyhdr}
\pagestyle{fancy}
\fancyhf{}
\renewcommand\headrulewidth{0pt}

\usepackage{epic}
\usepackage{curves}

\usepackage{pifont}

\frenchspacing

\begin{document}

\ttfamily

\noindent \\

\vspace{2em}
\hspace{5em} UNIVERSIDADE DE SÃO PAULO\par
\hspace{7em} ESCOLA POLITÉCNICA\par
\hspace{0.5em} DEPARTAMENTO DE ENGENHARIA DE ELETRICIDADE\par
\hspace{3em} LABORATÕRIO DE SISTEMAS DIGITAIS\par

\vspace{19em}
\hspace{4.5em} MONTADOR DO ``PATINHO FEIO''\par

\vspace{7em}
\hspace{7.5em} Antonio Marcos de Aguirra Massola\par
\hspace{7.5em} João José Neto\par
\hspace{7.5em} Moshe Bain\par

\vspace{7em}
\hspace{9.5em} Julho\par
\hspace{9.75em} 1977

\newpage

\noindent \\

\vspace{21em}
\hspace{6.5em} Em memória de\par
\hspace{6.5em} Laís Costa Ortenzi\par

\newpage

\setcounter{page}{1}
\fancyhead[R]{\ttfamily {I.\thepage}}

\renewcommand\contentsname{\ttfamily \hskip 13em \uline{INDICE}\\
\\
\uline{Assunto} \hskip 22.5em \uline{Página}}
\tableofcontents

\newpage

\setcounter{page}{1}
\setcounter{section}{1}
\fancyhead[R]{\ttfamily {\thesection \hskip 0.05em .\hskip 0.05em \thepage}}

\phantomsection
\section*{1 - \uline{INTRODUÇÃO}}
\addcontentsline{toc}{section}{CAPÍTULO 1 - INTRODUÇÃO}

\noindent \\
\par
O ante-projeto do minicomputador Patinho Feio nasceu\\
de um curso de pós-graduação dado pelo Professor Glen \hfill George\\
Langdon Jr., em 1972. A seguir, os engenheiros e estagiários do\\
Laboratório de Sistemas Digitais (LSD) da EPUSP terminaram \hfill o\\
projeto e montaram o Patinho Feio que, dessa forma, se \hfill tornou\\
o primeiro computador projetado e construído no Brasil.\\
\par
Os circuitos do Patinho Feio são totalmente \hfill consti-\\
tuídos por circuitos integrados da família TTL (``transistor \hfill -\\
transistor logic''), apresentando uma memória de núcleos de fe\uline r\\
rite, e tendo um ciclo de máquina de dois microsegundos.\\
\par
O Patinho Feio foi destinado a pesquisas no LSD, ta\uline n\\
to na área de programação (``software'') como dos circuitos ele-\\
trônicos (``hardware'').\\
\par
Cuidou-se do desenvolvimento de um ``software'' que pe\uline r\\
mitisse um uso mais eficiente do minicomputador, já que, \hfill de\\
início só se podia programá-lo em linguagem de máquina, manua\uline l\\
mente, através do seu painel. Em particular, foi definida \hfill uma\\
linguagem de montador (``assembly language''), que associa a ca-\\
da instrução de máquina um mnemônico, e um programa \hfill montador\\
(``assembler''), cuja função é traduzir programas escritos \hfill em\\
linguagem de montador para linguagem de máquina, os quais \hfill são\\
os assuntos tratados neste manual.\\
\par
Este manual foi escrito de forma a tratar cada tópi-\\
co de forma mais ou menos extensa, na suposição de que o \hfill lei-\\
tor tenha tido previamente apenas um pequeno contato com \hfill a\\
área de computação, e pouco ou nenhum conhecimento de \hfill lingu\uline a\\
gens de baixo nível, como um montador. Por causa disso,tentou-\\
se fazer com que o manual fosse o mais auto-explicativo e ind\uline e\\
pendente possível de outros textos. Naturalmente é \hfill impossível

\newpage

\noindent que um texto seja completamente independente de outros; por i\uline s\\
so, recomenda-se consultar outros textos, tais como manuais de\\
operação do Patinho Feio e de seus equipamentos periféricos(de\\
entrada/saída), textos sobre números binários, etc.\\
\par
Foi feito um bom esforço para apresentar os \hfill concei-\\
tos com clareza e para padronizar as notações, com o \hfill objetivo\\
de tornar o manual realmente útil. Contudo, certamente \hfill muitas\\
falhas subsistem, de forma que são bem recebidas quaisquer su-\\
gestões e críticas de modo a melhorar o manual em futuras edi-\\
ções.\\
\\
\\[-0.5em]
\uline{Observações}:

\begin{enumerate}[label=\alph*), align=left, leftmargin=1.5em, labelsep=-0.5em, itemsep=1em, topsep=1em]
\item As informações contidas neste manual são as melhores que se\\
pôde obter na época em que o manual foi escrito (setembro de\\
1975). Contudo, devido ao constante desenvolvimento de \hfill no-\\
vos projetos de ``hardware'' e ``software'' para o Patinho Feio,\\
alguns detalhes podem ter sofrido alterações até a presente\\
data.
\item Os programas e trechos de programa existentes no manual fo-\\
ram aí colocados por estarem sintaticamente corretos, \hfill mas\\
não representam necessariamente exemplos de boa técnica \hfill de\\
programação.
\end{enumerate}

\newpage

\setcounter{page}{1}
\setcounter{section}{2}

\phantomsection
\section*{2 - \uline{ARITMÉTICA BINÃRÍA E HEXADECIMAL}}
\addcontentsline{toc}{section}{CAPÍTULO 2 - ARITMÉTICA BINÃRÍA E HEXADECIMAL}

\hskip -3em (Com números inteiros)\\
\\[-0.5em]

\phantomsection
\subsection*{1. \uline{Bases de Numeração}}
\addcontentsline{toc}{subsection}{\hskip 5em Bases de Numeração}

Utiliza-se, na vida diária, a base decimal de numer\uline a\\
ção para representar os números. Isto significa duas coisas:

\begin{enumerate}[label=\alph*), align=left, leftmargin=1.5em, labelsep=-0.5em, itemsep=1em, topsep=1.5em]
\item existem dez algarismos com os quais todos os números são r\uline e\\
presentados (pois a base de numeração é dez), a saber: 0, 1,\\
2, 3, 4, 5, 6, 7, 8, 9.
\item emprega-se uma \uline{notação posicional} onde está subentendido que,\\
quando um algarismo é deslocado de uma posição para a esque\uline r\\
da, seu valor é multiplicado por dez. Por exemplo:\\[-0.5em]
\phantom \ \ \ \ \ \ \ \ \ \ \ \ 2 \ \ \ \ \ \ \ \ 1 \ \ \ \ \ \ \ \ 0\\[-1em]
295 = 2 x 10 \ + 9 x 10 \ + 5 x 10\\[-1.5em]
\end{enumerate}

Generalizando, quando se escreve o número N = d \ d \ \ ...\\[-1em]
\phantom \ \ \ \ \ \ \ \ \ \ \ \ \ \ \ \ \ \ \ \ \ \ \ \ \ \ \ \ \ \ \ \ \ \ \ \ \ \ \ \ \ \ \ \ \ \ \ \ \ \ \ \ \ \ \ \ \ n \ n-1\\[-0.5em]
d \ d \ d \ (sem sinal), onde os d \ (i= 0, 1, 2,...,n) são os seus\\[-1em]
\phantom \ 2 \ 1 \ 0 \ \ \ \ \ \ \ \ \ \ \ \ \ \ \ \ \ \ \ \ \ \ i\\[-0.5em]
algarismos (ou dígitos), está-se querendo dizer que: N = d \hfill x\\[-1em]
\phantom \ \ \ \ \ \ \ \ \ \ \ \ \ \ \ \ \ \ \ \ \ \ \ \ \ \ \ \ \ \ \ \ \ \ \ \ \ \ \ \ \ \ \ \ \ \ \ \ \ \ \ \ \ \ \ \ \ \ n \\[-1em]
\phantom \ \ n \ \ \ \ \ \ \ \ \ \ n-1 \ \ \ \ \ \ \ \ \ \ \ \ \ \ \ 2 \ \ \ \ \ \ \ \ \ 1 \ \ \ \ \ \ \ \ \ 0\\[-1em]
10 \ + d \ \ x 10 \ \ \ + ....+ d \ x 10 \ + d \ x 10 \ + d \ x 10 \ .\\[-1em]
\phantom \ \ \ \ \ \ \ n-1 \ \ \ \ \ \ \ \ \ \ \ \ \ \ \ \ 2 \ \ \ \ \ \ \ \ \ 1 \ \ \ \ \ \ \ \ \ 0\\
\par
Nada obriga a que se use apenas a base dez. Na verda-\\
de, qualquer base \uline b(inteira) pode ser escolhida para \ represen-\\
tar um número. Para tanto, escolhem-se \uline b símbolos distintos (os\\
algarismos da base) que representam os números de zero a (b - 1).\\
Escrevendo-se agora n + 1 algarismos adjacentes d d \ \ ...d d \ e\\[-1em]
\phantom \ \ \ \ \ \ \ \ \ \ \ \ \ \ \ \ \ \ \ \ \ \ \ \ \ \ \ \ \ \ \ \ \ \ \ \ \ \ \ \ \ \ \ \ \ \ \ \ \ n n-1 \ \ \ 1 0\\[-0.5em]
subentendida a notação posicional descrita acima, tem-se o núm\uline e\\
ro N representado por essa notação:\par
\ \ \ \ \ \ \ \ \ \ \ \ \ n \ \ \ \ \ \ \ \ n-1 \ \ \ \ \ \ \ \ \ \ 1 \ \ \ \ \ \ 0\\[-1em]
\phantom \ \ \ \ \ \ \ \ \ \ \ \ \ N = d \ . b \ + d \ \ \ b \ \ + ...+ d \ b \ + d \ b\\[-1em]
\phantom \ \ \ \ \ \ \ \ \ \ \ \ \ \ \ \ \ \ n \ \ \ \ \ \ \ \ n-1 \ \ \ \ \ \ \ \ \ \ \ \ 1 \ \ \ \ \ \ 0\\[-0.5em]
\par
Inversamente, pode-se provar que cada número N tem uma\\
única representação, numa dada base \uline b, que satisfaz as condições\\
mencionadas acima.\\
\par
Exemplo: Escolhendo b = 3, têm-se três algarismos;

\newpage

\noindent convencionalmente usa-se 0, 1, 2. Então tem-se:\\
\\[-1em]
\phantom \ \ \ \ \ \ \ \ \ \ \ \ \ \ \ \ \ \ \ \ \ 3 \ \ \ \ \ \ \ 2 \ \ \ \ \ \ \ 1 \ \ \ \ \ \ \ 0\\[-1em]
\phantom \ \ \ \ \ \ (1202) \ = 1 x 3 \ + 2 x 3 \ + 0 x 3 \ + 2 x 3 \ = (47)\\[-1em]
\phantom \ \ \ \ \ \ \ \ \ \ \ \ 3 \ \ \ \ \ \ \ \ \ \ \ \ \ \ \ \ \ \ \ \ \ \ \ \ \ \ \ \ \ \ \ \ \ \ \ \ \ \ \ \ \ \ 10\\
\par
Pode-se começar a perceber a importância do que \hfill foi\\
dito acima quando se considera que os computadores modermos tr\uline a\\
balham sempre, em última análise, com a base dois.\\
\\[0.5em]

\phantomsection
\subsection*{2. \uline{Bases mais empregadas em computação}}
\addcontentsline{toc}{subsection}{\hskip 5em Bases mais empregadas em computação}

Além da base dez, que é de uso geral, empregam-se co-\\
mumente as seguintes bases:

\begin{enumerate}[label=\alph*), align=left, leftmargin=1.5em, labelsep=-0.5em, itemsep=1em, topsep=1.5em]
\item base dois (binária) - necessita dois algarismos distintos p\uline a\\
ra representar os números zero e um. Por convenção utilizam-\\
se os símbolos 0 e 1, Um algarismo binário é também \ chamado\\
``bit'' (do inglês ``binary digit'').\\
\\
A base dois é extremamente importante pois, como já foi cit\uline a\\
do, os computadores só entendem sequências de zeros e uns \ ,\\ 
que são usadas tanto para representar as instruções dadas \ à\\
máquina quanto números prop iamente ditos.
\item base oito (octal) - utiliza os algarismos de 0 a 7. Não será\\
aqui tratada com mais detalhes porque não é utilizada no Pa-
tinho Feio, embora o seja em vários outros computadores.
\item base dezesseis (hexadecimal) - os dígitos hexadecimais \ são:\\
0, 1, 2, 3, 4, 5, 6, 7, 8, 9, A, B, C, D, E, F; usados \ para\\
representar os números de zero a quinze.
\end{enumerate}

\noindent \\[-3em]
\phantom \ \ \ \ \ \ \ \ \ \ \ \ \ \ \ \ \ \ \ \ \ \ \ \ \ \ \ \ 1 \ \ \ \ \ \ \ \ \ 0\\[-1em] 
\phantom \ \ \ Exemplo: (AB) \ \ = 10 x 16 \ + 11 x 16 \ = (171)\\[-1em]
\phantom \ \ \ \ \ \ \ \ \ \ \ \ \ \ \ \ 16 \ \ \ \ \ \ \ \ \ \ \ \ \ \ \ \ \ \ \ \ \ \ \ \ \ \ \ \ \ 10\\[-0.5em]
\par
A correspondência entre os valores binários, decimais\\
e hexadecimais é apresentada na tabela seguinte (note-se que são\\
necessários quatro bits para representar todos os dígitos \ hex\uline a\\
decimais na base dois).

\newpage

\ \ \ \uline{Decimal} \ \ \ \uline{Hexadecimal} \ \ \ \uline{Binário}\\[-1em]

\ \ \ \ \ \ 0 \ \ \ \ \ \ \ \ \ \ 0 \ \ \ \ \ \ \ \ \ \ 0000\par
\ \ \ \ \ \ 1 \ \ \ \ \ \ \ \ \ \ 1 \ \ \ \ \ \ \ \ \ \ 0001\par
\ \ \ \ \ \ 2 \ \ \ \ \ \ \ \ \ \ 2 \ \ \ \ \ \ \ \ \ \ 0010\par
\ \ \ \ \ \ 3 \ \ \ \ \ \ \ \ \ \ 3 \ \ \ \ \ \ \ \ \ \ 0011\par
\ \ \ \ \ \ 4 \ \ \ \ \ \ \ \ \ \ 4 \ \ \ \ \ \ \ \ \ \ 0100\par
\ \ \ \ \ \ 5 \ \ \ \ \ \ \ \ \ \ 5 \ \ \ \ \ \ \ \ \ \ 0101\par
\ \ \ \ \ \ 6 \ \ \ \ \ \ \ \ \ \ 6 \ \ \ \ \ \ \ \ \ \ 0110\par
\ \ \ \ \ \ 7 \ \ \ \ \ \ \ \ \ \ 7 \ \ \ \ \ \ \ \ \ \ 0111\par
\ \ \ \ \ \ 8 \ \ \ \ \ \ \ \ \ \ 8 \ \ \ \ \ \ \ \ \ \ 1000\par
\ \ \ \ \ \ 9 \ \ \ \ \ \ \ \ \ \ 9 \ \ \ \ \ \ \ \ \ \ 1001\par
\ \ \ \ \ 10 \ \ \ \ \ \ \ \ \ \ A \ \ \ \ \ \ \ \ \ \ 1010\par
\ \ \ \ \ 11 \ \ \ \ \ \ \ \ \ \ B \ \ \ \ \ \ \ \ \ \ 1011\par
\ \ \ \ \ 12 \ \ \ \ \ \ \ \ \ \ C \ \ \ \ \ \ \ \ \ \ 1100\par
\ \ \ \ \ 13 \ \ \ \ \ \ \ \ \ \ D \ \ \ \ \ \ \ \ \ \ 1101\par
\ \ \ \ \ 14 \ \ \ \ \ \ \ \ \ \ E \ \ \ \ \ \ \ \ \ \ 1110\par
\ \ \ \ \ 15 \ \ \ \ \ \ \ \ \ \ F \ \ \ \ \ \ \ \ \ \ 1111\\

\phantomsection
\subsection*{3. \uline{Conversão de números entre as bases dois, dez e \ \ dezes-}\\
\phantom \ \ \ \uline{seis}}
\addcontentsline{toc}{subsection}{\hskip 5em Conversão entre as bases dois, dez \ e\\ dezesseis}

Conforme já se deve ter percebido, surge frequente -\\
mente a necessidade de converter números escritos em uma \ base\\
psta outra. Para isso existem algoritmos gerais, dos quais são\\
apresentados abaixo alguns casos particulares:

\begin{enumerate}[label=\alph*), align=left, leftmargin=1.5em, labelsep=-0.5em, itemsep=1em, topsep=1.5em]
\item Conversão para a base dez de números escritos em outra base.\\[-0.5em]
\phantom \ \ \ \ \ \ \ \ \ \ \ \ \ \ \ \ \ \ \ \ \ \ \ \ \ \ \ \ \ \ \ \ \ \ \ \ \ \ \ n\\[-1em]
Basta escrever o número na forma d \ . b \ + ...+ d \ e efetuar\\[-1em]
\phantom \ \ \ \ \ \ \ \ \ \ \ \ \ \ \ \ \ \ \ \ \ \ \ \ \ \ \ \ \ \ \ \ \ \ n \ \ \ \ \ \ \ \ \ \ \ \ \ 0\\[-0.5em]
as operações indicadas.\\[0.5em]
\phantom \ \ \ \ \ \ \ \ \ \ \ \d{o}\\[-1.2em]
Exemplos: 1 ) \ (101111100001) \ para a base 10\\[-1em]
\phantom \ \ \ \ \ \ \ \ \ \ \ \ \ \ \ \ \ \ \ \ \ \ \ \ \ \ \ \ \ 2\\[-0.5em]
\phantom \ \ \ \ \ \ \ \ \ \ \ \ \ \ \ \ \ \ \ \ 11 \ \ \ \ \ 10 \ \ \ \ \ \ \ \ \ \ 1\\[-1em]
\phantom \ \ \ \ \ \ \ \ \ \ \ \ \ \ \ = 1.2 \ \ + 0.2 \ \ + ...+ 0.2 \ + 1 = \ (3041)\\[-1em]
\phantom \ \ \ \ \ \ \ \ \ \ \ \ \ \ \ \ \ \ \ \ \ \ \ \ \ \ \ \ \ \ \ \ \ \ \ \ \ \ \ \ \ \ \ \ \ \ \ \ \ \ \ \ \ \ \ \ 10\\[1em]
\phantom \ \ \ \ \ \ \ \ \ \ \ \d{o}\\[-1.2em]
\phantom \ \ \ \ \ \ \ \ \ \ 2 ) \ (BE1) \ \ para a base 10\\[-1em]
\phantom \ \ \ \ \ \ \ \ \ \ \ \ \ \ \ \ \ \ \ \ 16\\
\phantom \ \ \ \ \ \ \ \ \ \ \ \ \ \ \ \ \ \ \ \ \ \ 2\\[-1em]
\phantom \ \ \ \ \ \ \ \ \ \ \ \ \ = 11 x 16 \ + 14 x 16 + 1 = \ (3041)\\[-1em]
\phantom \ \ \ \ \ \ \ \ \ \ \ \ \ \ \ \ \ \ \ \ \ \ \ \ \ \ \ \ \ \ \ \ \ \ \ \ \ \ \ \ \ \ \ \ \ \ \ 10\\[1em]

\newpage

\noindent Uma forma conveniente de fazer isso é dada \ nos \ diagramas \\
abaixo:\\\\[-1em]
\phantom \ \ \ \ \ \ \ \ \ \ \ \ \ \ (BE1) \ \ para a base 10\\[-1em]
\phantom \ \ \ \ \ \ \ \ \ \ \ \ \ \ \ \ \ \ \ 16\\[-0.5em]

\noindent \ \ \ \ \ \ \ \ \uline{\ \ \ \ \ \ 11 \ \ 14 \ \ \ 1 \ \ \ \ \ }\\[-0.5em]
base\\[-1em]
\phantom \ \ \ \ \ \ \ \ \ \ \ \ x\\[-1em]
original=16 \ \ 11 \ \ 190 \ (3041)\\[-1em]
\phantom \ \ \ \ \ \ \ \ \ \ \ \ \ \ \ \ \ \ \ \ \ \ \ \ \ \ \ \ \ \ 10

\begin{picture}(0,0)
\put(92.5,46){\vector(0,-1){10}}
\put(84.5,30){\vector(-1,0){15}}
\end{picture}\\[-2em]

\noindent Método usado: 11 x 16 + 14 = 190\\[1em]
\phantom \ \ \ \ \ \ \ \ \ \ \ \ \ \ \ \ \ \ \ \ \ \ \ \ \ \ \ \ \ 190 x 16 + 1 = 3041

\begin{picture}(0,0)
\put(188,48){\vector(0,-1){15}}
\end{picture}\\[-3em]

\phantom \ \ \ \ \ \ \ \ \ \ \ \ \ \ \ (10110) \ para a base 10\\[-1em]
\phantom \ \ \ \ \ \ \ \ \ \ \ \ \ \ \ \ \ \ \ \ \ 2\\[-1em]

\noindent \ \ \ \ \ \ \ \ \uline{\ \ \ \ \ \ 1 \ \ 0 \ \ 1 \ \ 1 \ \ 0 \ \ }\\
base\\[-0.5em]
original= 2 \ \ 1 \ \ 2 \ \ 5 \ 11 (22)\\[-1em]
\phantom \ \ \ \ \ \ \ \ \ \ \ \ \ \ \ \ \ \ \ \ \ \ \ \ \ \ \ \ \ \ \ \ 10

\begin{picture}(0,0)
\put(89.25,53){\vector(0,-1){15}}
\end{picture}\\[-2.5em]

Método usado: 1\hskip 0.25em x\hskip 0.25em 2\hskip 0.25em +\hskip 0.25em 0 = 2 \ \ \ \ \ \ \ 5\hskip 0.25em x\hskip 0.25em 2\hskip 0.25em +\hskip 0.25em 1\hskip 0.25em =\hskip 0.25em 11\\
\\[-0.5em]
\phantom \ \ \ \ \ \ \ \ \ \ \ \ \ \ \ \ \ \ \ \ \ \ \ \ 2\hskip 0.25em x\hskip 0.25em 2\hskip 0.25em +\hskip 0.25em 1\hskip 0.25em =\hskip 0.25em 5 \ \ \ \ \ \ \ 11\hskip 0.25em x\hskip 0.25em 2\hskip 0.25em +\hskip 0.25em 0 = 22

\begin{picture}(0,0)
\put(151,48){\vector(0,-1){15}}
\put(206.5,33){\vector(0,1){15}}
\put(265.5,48){\vector(0,-1){15}}
\end{picture}\\[-2.5em]

\item Conversão de números escritos na base dez para uma outra b\uline a\\
se. Divide-se \ repetidamente o número dado pela base de de\uline s\\
tino até que o quociente seja zero. Os restos obtidos \hfill são\\
a representação desejada, em ordem invertida. Ver os esque-\\
mas abaixo:\\[-2.5em]

\phantom \ \ \ \ \ \ \ \ \ \ \ \ (3041) \ \ \ \ \ \ base 16\\[-1em]
\phantom \ \ \ \ \ \ \ \ \ \ \ \ \ \ \ \ \ 10\\
\phantom \ \ \ \ \ \ \ \ \ \ \ \ \ \ \ \ \ \ \ \ \ \ \ \ \ \ \ \ \ \ \ \ resto de 3041 : 16\\
\phantom \ \ \ \ \ \ \ \ \ \ \ \ \ \ \ \ \ \ \ \ \ \ \ \ \ \ \ \ \ \ \ \ resto de \ 190 : 16\\[-1em]
\phantom \ \ \ \ \ \ \ \ \ \ \ \ \ \ \ \ \ 3041 \ \ \ 1\\[-0.5em]
\phantom \ \ \ \ \ \ \ \ \ \ \ \ \ \ \ \ \ \ \ \ \ \ \ \ \ \ \ \ \ \ \ \ resto de \ \ 11 : 16\\[-0.5em]
\phantom \ \ 3041 : 16 \ \ \ \ \ \ 190 \ \ 14\\[0.5em]
\phantom \ \ \ 190 : 16 \ \ \ \ \ \ \ 11 \ \ 11 \ 14 \ 1\\[1em]
\phantom \ \ \ \ 11 : 16 \ \ \ \ \ \ \ \ 0 \ \ \ (BE1)\\[-1em]
\phantom \ \ \ \ \ \ \ \ \ \ \ \ \ \ \ \ \ \ \ \ \ \ \ \ \ \ \ \ \ \ 16\\

\begin{picture}(0,0)
\put(143,28){\line(0,1){110}}
\put(75,101){\vector(1,0){30}}
\put(75,78){\vector(1,0){30}}
\put(75,48){\vector(1,0){30}}

\dashline{4}(163,125)(194,142)
\put(163,125){\vector(-2,-1){2}}

\dashline{4}(163,103)(194,124)
\put(163,103){\vector(-2,-1){2}}

\dashline{4}(163,80)(194,108)
\put(163,80){\vector(-1,-1){2}}

\put(162,120){\vector(1,-1){35}}
\put(162,98){\vector(1,-1){13}}
\put(155,72){\vector(1,-3){6}}
\put(179,72){\vector(-1,-2){9}}
\put(196,72){\vector(-1,-1){18}}
\end{picture}

\newpage

\ \ \ \ \ \ \ \ \ \ \ \ \ \ (3041) \ \ \ \ \ base 2\\[-1em]
\phantom \ \ \ \ \ \ \ \ \ \ \ \ \ \ \ \ \ \ \ \ 10\\[-0.5em]

\ \ \ \ \ 3041 \ \ 1\\[-1.8em]
\par
\ \ \ \ \ 1524 \ \ 0\\[-1.8em]
\par
\ \ \ \ \ \ 760 \ \ 0\\[-1.8em]
\par
\ \ \ \ \ \ 380 \ \ 0\\[-1.8em]
\par
\ \ \ \ \ \ 190 \ \ 0\\[-1.8em]
\par
\ \ \ \ \ \ \ 95 \ \ 1\\[-1.8em]
\par
\ \ \ \ \ \ \ 47 \ \ 1\\[-1.8em]
\par
\ \ \ \ \ \ \ 23 \ \ 1\\[-1.8em]
\par
\ \ \ \ \ \ \ 11 \ \ 1\\[-1.8em]
\par
\ \ \ \ \ \ \ \ 5 \ \ 1\\[-1.8em]
\par
\ \ \ \ \ \ \ \ 2 \ \ 0\\[-1.8em]
\par
\ \ \ \ \ \ \ \ 1 \ (1 0 1 1 1 1 1 0 0 0 0 1)\\[-1em]
\phantom \ \ \ \ \ \ \ \ \ \ \ \ \ \ \ \ \ \ \ \ \ \ \ \ \ \ \ \ \ \ \ \ \ \ \ \ 2\\[-2.2em]
\par
\ \ \ \ \ \ \ \ 0

\begin{picture}(0,0)
\put(66,8){\line(0,1){250}}
\put(80,228){\vector(3,-4){133}}
\put(80,210){\vector(3,-4){120}}
\put(80,193){\vector(3,-4){107}}
\put(80,175){\vector(3,-4){93}}
\put(80,158){\vector(3,-4){80}}
\put(80,141){\vector(3,-4){68}}
\put(80,124){\vector(3,-4){55}}
\put(80,107){\vector(3,-4){42}}
\put(80,91){\vector(3,-4){30}}
\put(80,74){\vector(3,-4){17}}
\put(80,59){\vector(3,-4){8}}
\end{picture}\\[-2em]

\item Conversão entre as bases dois e dezesseis.

\begin{enumerate}[label=c-\arabic*), align=left, leftmargin=2.5em, labelsep=-0.1em, itemsep=1em, topsep=0em]
\item Da base dois para a base dezesseis. Basta agrupar \hfill os\\
dígitos binários de quatro em quatro (a partir da \hfill di-\\
reita) e substituí-los pelo respectivo dígito hexadec\uline i\\
mal, conforme a tabela apresentada mais atrás (item 2.\\
c).\\[0.5em]
Exemplo:\\[-0.5em]
\par
\ \ \ \ \ \ \ \ \ \ \ \ \ \ \ \ (1011110101) \ para base 16\\[-1em]
\phantom \ \ \ \ \ \ \ \ \ \ \ \ \ \ \ \ \ \ \ \ \ \ \ \ \ \ \ \ 2\\[-0.5em]
\phantom \ \ \ zeros\\[-0.5em]
\phantom \ \ \ adicionados\\[-0.5em]
\phantom \ \ \ \ \ \ \ \ \ \ \ \ \ \ \ \ \uline{0010} \uline{1111} \uline{0101}\\[-0.5em]
\par
\ \ \ \ \ \ \ \ \ \ \ \ \ \ \ \ \ \ \ (2F5)\\[-1em]
\phantom \ \ \ \ \ \ \ \ \ \ \ \ \ \ \ \ \ \ \ \ \ \ \ \ 16

\begin{picture}(0,0)
\put(87,76){\line(1,0){21}}
\put(87,62){\line(0,1){28}}
\put(80,62){\line(1,0){7}}
\put(80,90){\line(1,0){7}}
\put(101.5,76){\vector(0,-1){10}}
\put(108,76){\vector(0,-1){10}}
\put(111,53){\vector(2,-3){12}}
\put(142,53){\vector(-1,-2){9}}
\put(173,53){\vector(-3,-2){28}}
\end{picture}\\[-2em]

\item Da base dezesseis para a base dois. Basta substituir c\uline a\\
da dígito hexadecimal pelo seu código de quatro \hfill bits.\\[0.5em]
Exemplo:

\newpage

\ \ \ \ \ \ \ \ \ (2F5) \ \ para base 2\\[-1em]
\phantom \ \ \ \ \ \ \ \ \ \ \ \ \ \ 16\\
\\
\\[-0.5em]
\phantom \ \ \ \ (0010 1111 0101)\\[-1em]
\phantom \ \ \ \ \ \ \ \ \ \ \ \ \ \ \ \ \ \ \ \ 2\\

\begin{picture}(0,0)
\put(62,90){\vector(-1,-2){18}}
\put(71,90){\vector(0,-1){36}}
\put(80,90){\vector(2,-3){24}}
\end{picture}\\[-5em]

\phantom \ \ \ \ \ \ \ \ \d{o}\\[-1.2em]
0bs.: 1 ) Para a base oito, como é fácil perceber,o m\uline é\\
\phantom \ \ \ \ \ \ \ \ \ \ todo é inteiramente análogo, dividindo-se \hfill o\\
\phantom \ \ \ \ \ \ \ \ \ \ número binário em grupos de três bits.\\[1.2em]
\phantom \ \ \ \ \ \ \ \d{o}\\[-1.2em]
\phantom \ \ \ \ \ \ 2 ) Pode-se agora notar porque são tão usadas as\\
\phantom \ \ \ \ \ \ \ \ \ \ bases oito e dezesseis em computação: \hfill elas\\[-0.5em]
\phantom \ \ \ \ \ \ \ \ \ \ \ \ \ \ \ \ \ \ \ \ \ \ \ \ \ \ \ \ \ \ \ \ \ \ \ \ \ \ \ \ \ \ \ \ \ \ \ 3\\[-1em]
\phantom \ \ \ \ \ \ \ \ \ \ permitem dividir por três (pois 8 = 2 )e por\\[-0.5em]
\phantom \ \ \ \ \ \ \ \ \ \ \ \ \ \ \ \ \ \ \ \ \ \ \ \ \ \ \ \ \ 4\\[-1em]
\phantom \ \ \ \ \ \ \ \ \ \ quatro (pois 16 = 2 ), respectivamente, \hfill o\\
\phantom \ \ \ \ \ \ \ \ \ \ comprimento em algarismos do número \hfill escrito\\
\phantom \ \ \ \ \ \ \ \ \ \ na base dois, que costuma ser inconveniente-\\
\phantom \ \ \ \ \ \ \ \ \ \ mente longo.\\[1.2em]
\phantom \ \ \ \ \ \ \ \d{o}\\[-1.2em]
\phantom \ \ \ \ \ \ 3 ) Estã-se dando mais ênfase à base hexadecimal\\
\phantom \ \ \ \ \ \ \ \ \ \ porque no Patinho Feio os números têm ou oi-\\
\phantom \ \ \ \ \ \ \ \ \ \ to ou doze bits de comprimento, podendo \hfill en-\\
\phantom \ \ \ \ \ \ \ \ \ \ tão, ser representados com dois ou três \hfill dí-\\
\phantom \ \ \ \ \ \ \ \ \ \ gitos hexadecimais, enquanto que, por \hfill exem-\\
\phantom \ \ \ \ \ \ \ \ \ \ plo, para transformar um número de oito bits\\
\phantom \ \ \ \ \ \ \ \ \ \ (também chamado ``byte'') em um número octal ,\\
\phantom \ \ \ \ \ \ \ \ \ \ tem-se que adicionar um zero à frente do nú-\\
\phantom \ \ \ \ \ \ \ \ \ \ mero, para dividí-lo em três grupos de \hfill três\\
\phantom \ \ \ \ \ \ \ \ \ \ bits cada.\\
\end{enumerate}
\end{enumerate}

\phantomsection
\subsection*{4. \uline{Soma de números binários positivos}}
\addcontentsline{toc}{subsection}{\hskip 5em Soma de números binários positivos}

Realiza-se de forma inteiramente análoga à soma \ so-\\
mum, de números decimais. Para ver isso, examine-se detalhada-\\
mente uma soma decimal, por exemplo, de 1672 com 729.

\newpage

\hskip 2.5em 0 \ \ \ 1 \ \ 1 \ \ 1\\
\\[-0.5em]
\phantom \ \ \ \ \ \ \ \ \ \ \ \ \ \ \ \ \ \ 1 \ \ 6 \ \ 7 \ \ 2\\[-0.5em]
\phantom \ \ \ \ \ \ \ \ \ \ \ \ \ \ \ \ \ + \ \ + \ \ + \ \ +\\[-0.5em]
\phantom \ \ \ \ \ \ \ \ \ \ \ \ \ \ \ \ \ \ \ \ \ \ \ \ \ \ \ \ \ \ \ \ \ +\\[-0.5em]
\phantom \ \ \ \ \ \ \ \ \ \ \ \ \ \ \ \ \ \ \ \ \ \ 7 \ \ 2 \ \ 9\\
\\[-0.5em]
\phantom \ \ \ \ \ \ \ \ \ \ \ \ \ 0 \ \ \ 2 \ \ 4 \ \ 0 \ \ 1\\

\begin{picture}(0,0)
\put(30,122){\line(-1,-2){9}}

\put(68,122){\line(1,-2){6}}
\put(93,122){\line(1,-2){6}}
\put(118,122){\line(1,-2){6}}

\put(68,122){\line(-1,0){4}}
\put(93,122){\line(-1,0){4}}
\put(118,122){\line(-1,0){4}}

\put(61,122){\vector(-1,-2){6}}
\put(86,122){\vector(-1,-2){6}}
\put(111,122){\vector(-1,-2){6}}

\put(21,104){\vector(0,-1){49}}
\put(52.5,92){\vector(0,-1){37}}
\put(76.5,92){\vector(0,-1){16}}
\put(101.5,92){\vector(0,-1){16}}
\put(126.5,92){\vector(0,-1){16}}

\put(18,52){\line(1,0){112}}
\end{picture}\\[-3em]

Começando a partir da direita, foram realizadas \hfill as\\
seguintes operações:\\[0.5em]
\phantom \ \ \ \ \ \ \ \ \ \ \ \ \ \ 2 + 9 = 1 \ e \ vai-um \ (= 11)\\
\phantom \ \ \ \ \ \ \ \ \ \ 1 + 7 + 2 = 0 \ e \ vai-um \ (= 10)\\
\phantom \ \ \ \ \ \ \ \ \ \ 1 + 6 + 7 = 4 \ e \ vai-um \ (= 14)\\
\phantom \ \ \ \ \ \ \ \ \ \ \ \ \ \ 1 + 1 = 2 \ e \ vai-zero(= 02)\\
\phantom \ \ \ \ \ \ \ \ \ \ \ \ \ \ \ \ \ \ 0 = 0\\
\par
Com os números binários procede-se da mesma forma,s\uline e 
gundo as seguintes regras:\\
\\
\phantom \ \ \ \ \ \ \ \ \ \ 0 + 0 + 0 \ = 00 \ \ \ \ \ \ \ \ 0 \ e \ vai-um\\
\phantom \ \ \ \ \ \ \ \ \ \ 0 + 0 + 1 \ = 01 \ \ \ \ \ \ \ \ 1 \ e \ vai-zero\\
\phantom \ \ \ \ \ \ \ \ \ \ 0 + 1 + 1 \ = 10 \ \ \ \ \ \ \ \ 0 \ e \ vai-um\\
\phantom \ \ \ \ \ \ \ \ \ \ 1 + 1 + 1 \ = 11 \ \ \ \ \ \ \ \ 1 \ e \ vai-um\\

\begin{picture}(0,0)
\put(106,91){\vector(1,0){36}}
\put(106,73){\vector(1,0){36}}
\put(106,55){\vector(1,0){36}}
\put(106,38){\vector(1,0){36}}
\end{picture}\\[-3em]

Exemplo:\\
\phantom \ \ \ \ \ \ \ \ \ \ \ \d{o}\\[-1.2em]
\phantom \ \ \ \ \ \ \ \ \ \ 1 ) Seja somar \ 10111010 com 10011. Tem-se:\\
\\
\\
\phantom \ \ \ \ \ \ \ \ \ \ \ \ \ \ \ 0 \ \ 0 \ \ 1 \ \ 1 \ \ 0 \ \ 0 \ \ 1 \ \ 0\\[0.5em]
\phantom \ \ \ \ \ \ \ \ \ \ \ \ \ \ \ \ \ 1 \ \ 0 \ \ 1 \ \ 1 \ \ 1 \ \ 0 \ \ 1 \ \ 0\\
\phantom \ \ \ \ \ \ \ \ \ \ \ \ \ \ \ \ \ \ \ \ \ \ \ \ \ \ \ \ \ \ \ \ \ \ \ \ \ \ \ \ \ \ \ \ \ \ \ \ +\\[-1em]
\phantom \ \ \ \ \ \ \ \ \ \ \ \ \ \ \ \ \ \ \ \ \ \ \ \ \ \ \ \ \ 1 \ \ 0 \ \ 0 \ \ 1 \ \ 1\\[-0.5em]
\phantom \ \ \ \ \ \ \ \ \ \ \ \ \ \ \ \ \ \ \ \ \ \ \ \ \ \ .\\[-0.5em]
\phantom \ \ \ \ \ \ \ \ \ \ \ \ \ \ \ 0 1 \ \ 1 \ \ 0 \ \ 0 \ \ 1 \ \ 1 \ \ 0 \ \ 1

\begin{picture}(0,0)
\dashline{4}(33,75)(33,38)
\put(33,38){\vector(0,-1){0}}

\curve(33,75, 38,77.5, 43,75)
\curve(46,75, 57,80, 68,75)
\curve(71,75, 82,80, 93,75)
\curve(96,75, 107,80, 118,75)
\curve(121,75, 132,80, 143,75)
\curve(146,75, 157,80, 168,75)
\curve(171,75, 182,80, 193,75)
\curve(196,75, 207,80, 218,75)

\put(46,75){\vector(-3,-2){0}}
\put(71,75){\vector(-3,-2){0}}
\put(96,75){\vector(-3,-2){0}}
\put(121,75){\vector(-3,-2){0}}
\put(146,75){\vector(-3,-2){0}}
\put(171,75){\vector(-3,-2){0}}
\put(196,75){\vector(-3,-2){0}}

\put(30,34){\line(1,0){193}}
\end{picture}\\

\par
Começa-se a partir da direita, realizando as seguin-\par
tes operações:

\newpage

\ \ \ \ \ \ \ \ \ \ \ \ 0 + 1 \ = \ 1 \ e \ vai-zero\\
\phantom \ \ \ \ \ \ \ \ \ \ \ \ \ \ \ \ \ \ 0 + 1 + 1 \ = \ 0 \ e \ vai-um\\
\phantom \ \ \ \ \ \ \ \ \ \ \ \ \ \ \ \ \ \ 1 + 0 + 0 \ = \ 1 \ e \ vai-zero\\
\phantom \ \ \ \ \ \ \ \ \ \ \ \ \ \ \ \ \ \ 0 + 1 + 0 \ = \ 1 \ e \ vai-zero\\
\phantom \ \ \ \ \ \ \ \ \ \ \ \ \ \ \ \ \ \ 0 + 1 + 1 \ = \ 0 \ e \ vai-um\\
\phantom \ \ \ \ \ \ \ \ \ \ \ \ \ \ \ \ \ \ 1 + 1 \ \ \ \ \ = \ 0 \ e \ vai-um\\
\phantom \ \ \ \ \ \ \ \ \ \ \ \ \ \ \ \ \ \ 1 + 0 \ \ \ \ \ = \ 1 \ e \ vai-zero\\
\phantom \ \ \ \ \ \ \ \ \ \ \ \ \ \ \ \ \ \ 0 + 1 \ \ \ \ \ = \ 1 \ e \ vai-zero\\
\phantom \ \ \ \ \ \ \ \ \ \ \ \ \ \ \ \ \ \ 0 \ \ \ \ \ \ \ \ \ = \ 0\\
\\
\phantom \ \ \ \ \ \ coluna dos \ \ \ \ \ \ \ \ \ \ \ \ \ \ \ \ bits do segundo número\\[-0.5em]
\phantom \ \ \ \ \ \ vai-um ou\\[-0.5em]
\phantom \ \ \ \ \ \ vai-zero \ \ \ \ \ \ \ \ \ \ \ \ \ \ \ \ \ \ bits do primeiro número

\begin{picture}(0,0)
\put(102,43){\line(1,0){33}}
\put(102,43){\vector(0,1){95}}

\put(77,21){\line(1,0){58}}
\put(77,21){\vector(0,1){64}}

\put(52,31){\line(-1,0){16}}
\put(52,31){\vector(0,1){37}}
\end{picture}\\[-2em]

\phantom \ \ \d{o}\\[-1.2em]
\phantom \ \ \ \ \ \ \ \ \ \ 2 ) Somar \ 11101 com 110.\\
\\
\phantom \ \ \ \ \ \ \ \ \ \ \ \ \ \ \ 1 \ \ \ 1 \ \ 1 \ \ 0 \ \ 0\\[0.5em]
\phantom \ \ \ \ \ \ \ \ \ \ \ \ \ \ \ \ \ \ 1 \ \ 1 \ \ 1 \ \ 0 \ \ 1\\
\phantom \ \ \ \ \ \ \ \ \ \ \ \ \ \ \ \ \ \ \ \ \ \ \ \ \ \ \ \ \ \ \ \ \ \ \ \ +\\[-1em]
\phantom \ \ \ \ \ \ \ \ \ \ \ \ \ \ \ \ \ \ \ \ \ \ \ \ \ \ 1 \ \ 1 \ \ 0\\[0.5em]
\phantom \ \ \ \ \ \ \ \ \ \ \ \ \ \ \ 1 \ 0 \ \ 0 \ \ 0 \ \ 1 \ \ 1

\begin{picture}(0,0)
\put(33,75){\vector(0,-1){38}}

\curve(33,75, 41.5,79, 50,75)
\curve(53,75, 64,80, 75,75)
\curve(78,75, 89,80, 100,75)
\curve(103,75, 114,80, 125,75)
\curve(128,75, 139,80, 150,75)

\put(53,75){\vector(-3,-2){0}}
\put(78,75){\vector(-3,-2){0}}
\put(103,75){\vector(-3,-2){0}}
\put(128,75){\vector(-3,-2){0}}

\put(30,34){\line(1,0){125}}
\end{picture}\\[1em]

\phantomsection
\subsection*{5. \uline{Representação de números negativos}}
\addcontentsline{toc}{subsection}{\hskip 5em Representação de números negativos}

Obs.: Nos itens seguintes assume-se sempre que um n\uline ú\par
\ \ \ \ \ \ mero tem oito bits de comprimento, quando \hfill for\par
\ \ \ \ \ \ binário.\\
\par
Até agora, só foram tratados os números positivos \hfill .\\
Contudo, é óbvia a necessidade de se manipular números \hfill negat\uline i\\
vos, de modo que é preciso uma representação adequada para \hfill os\\
mesmos. Especialmente, é necessária essa representação para n\uline ú\\
meros binários, de modo que o computador possa reconhecer \hfill os\\
números que sejam negativos como tais.

\newpage

Existem três modos de representar números \hfill negativos\\
em notação binária, chamados de: sinal e amplitude, complemen-\\
to de um e complemento de dois.

\begin{enumerate}[label=\alph*), align=left, leftmargin=1.5em, labelsep=-0.5em, itemsep=1em, topsep=1em]
\item Representação de sinal e amplitude.\\
Usualmente, quando se quer denotar um número como \hfill negativo\\
(em qualquer base), coloca-se à sua frente um sinal de menos\\
(-), e quando positivo,às vezes, o sinal de mais (+). Mas,c\uline o\\
mo um computador não reconhece os sinais + e -, mas \hfill apenas\\
zeros e uns, vê-se que é necessário reservar um bit do núm\uline e\\
ro (geralmente o primeiro)para indicar o seu sinal \hfill (usa-se\\
\uline{zero} para indicar um número positivo e \uline{um} para indicar \hfill um\\
negativo). Supondo um número de oito bits, tem-se, por exe\uline m\\
plo:\\[0.5em]
\phantom \ \ \ \ (+6) \ \ \ \ \ \ \ 0000 0110\\[-1em]
\phantom \ \ \ \ \ \ \ \ 10\\[0.5em]
\phantom \ \ \ \ (-6) \ \ \ \ \ \ \ 1000 0100\\[-1em]
\phantom \ \ \ \ \ \ \ \ 10\\
\\
\phantom \ \ \ \ \ \ \ \ \ \ \ \ \ sinal \ amplitude

\begin{picture}(0,0)
\dashline{4}(104.5,102)(104.5,45)
\dashline{4}(104.5,45)(117,45)
\dashline{4}(117,45)(117,18)
\end{picture}\\[-1.5em]

Desta forma pode-se representar os números inteiros de -127\\
a +127. Note-se que existem duas representações \ do \hfill número\\
zero, a saber: 0000 0000 \ \ e \ \ 1000 0000.\\[-0.5em]
\item Representação em complemento de um.\\
Nesta representação, para indicar um número negativo troca-\\
se os seus zeros por uns e vice-versa. Como sempre, o \hfill pri-\\
meiro bit indicará o sinal do número. Exemplo:\\
\\
\phantom \ \ \ \ (+6) \ \ \ \ \ \ \ 0000 0110\\[-1em]
\phantom \ \ \ \ \ \ \ \ 10\\[0.5em]
\phantom \ \ \ \ (-6) \ \ \ \ \ \ \ 1111 1001\\[-1em]
\phantom \ \ \ \ \ \ \ \ 10\\
\\
\phantom \ \ \ \ \ \ \ \ \ \ \ \ \ \ sinal

\begin{picture}(0,0)
\dashline{4}(101.5,88)(101.5,33)
\end{picture}

\newpage

Deste modo, analogamente ao anterior, pode-se representar os\\
números de -127 a +127 e o zero continua com duas represen-\\
tações, a saber: \ 0000 0000 \ \ e \ \ 1111 1111.\\[-1em]
\item Representação \ em complemento de dois.\\
Para se obter a representação em complemento de dois, soma-\\
se um (em binário) à representação em complemento de um, r\uline e\\
tendo-se apenas os oito bits mais à direita. Exemplo:\\[1em]
(+6) \ = 0000 0110 \ \ \ \ \ ida \ \ \ \ \ \ \ 1111 1001\\[-1em]
\phantom \ \ \ \ 10 \ \ \ \ \ \ \ \ \ \ \ \ \ \ \ \ \ \ \ \ \ \ \ \ \ \ \ \ \ \ \ \ \ \ \ \ +\\[-1em]
\phantom \ \ \ \ \ \ \ \ \ \ \ \ \ \ \ \ \ \ \ \ \ \ \ \ \ \ \ \ \ \ \ \ \ \ \ \ \ \ \ \ \ \ 1\\[-0.5em]
\phantom \ \ \ \ \ \ \ \ \ \ \ \ \ \ \ \ \ \ \ complemento\\[-0.5em]
\phantom \ \ \ \ \ \ \ \ \ \ \ \ \ \ \ \ \ \ \ \ \ \ de um\\[-1em]
\phantom \ \ \ \ \ \ sinal\\[-1em]
\phantom \ \ \ \ \ \ \ \ \ \ \ \ \ \ \ \ 1 \ \ \ \ \ \ \ \ \ \ \ \ \ \ \ \ 1111 1010 \ = (-6) \ \ em co\uline m\\[-1em]
\phantom \ \ \ \ \ \ \ \ \ \ \ \ \ \ \ \ \ + \ \ \ \ \ \ \ \ \ \ \ \ \ \ \ \ \ \ \ \ \ \ \ \ \ \ \ \ \ \ \ \ 10\\[-1em]
\phantom \ \ \ \ \ \ \ \ 0000 0101 \ \ \ \ \ volta\\[-1em]
\phantom \ \ \ \ \ \ \ \ \ \ \ \ \ \ \ \ \ \ \ \ \ \ \ \ \ \ \ \ \ \ \ \ \ \ \ \ \ \ \ \ \ \ \ \ \ \ \ \ \ plemento de\\[-0.5em]
\phantom \ \ \ \ \ \ \ \ \ \ \ \ \ \ \ \ \ \ \ \ \ \ \ \ \ \ \ \ \ \ \ \ \ sinal\\[-1em]
\phantom \ \ \ \ \ \ \ \ \ \ \ \ \ \ \ \ \ \ \ \ \ \ \ \ \ \ \ \ \ \ \ \ \ \ \ \ \ \ \ \ \ \ \ \ \ \ \ \ \ dois.

\begin{picture}(0,0)
\put(111,100){\line(1,0){24}}
\put(167,100){\vector(1,0){37}}
\put(52,69){\vector(0,1){24}}
\put(56,69){\line(1,0){49}}
\put(209,69){\line(1,0){57}}
\put(136,47){\vector(-1,0){24}}
\put(177,47){\line(3,1){29}}
\put(218,36){\vector(-1,3){5}}
\end{picture}\\[-2em]

A representação em complemento de dois tem as seguintes pr\uline o\\
priedades:

\begin{enumerate}[label=\arabic*a.), align=left, leftmargin=2.5em, labelsep=0em, itemsep=1em, topsep=0.5em]
\item O primeiro bit do número indica o seu sinal: positivo se\\
zero e negativo se um.
\item São representáveis os números de -128, cuja representa-\\
ção é \ 1000 0000; a +127, cuja representação é 0111 1111.\\
Desta forma, o número -128 não tem complemento de dois.\\
De fato, tem-se:\\[0.5em]
\phantom \ \ \ -128 \ \ \ 1000 0000 \ \ \ 0111 1111\\[-1em]
\phantom \ \ \ \ \ \ \ \ \ \ \ \ \ \ \ \ \ \ \ \ \ \ \ \ \ \ \ \ \ \ \ \ \ +\\[-1em]
\phantom \ \ \ \ \ \ \ \ \ \ \ \ \ \ \ \ \ \ \ \ \ \ \ \ \ \ \ \ \ \ \ \ 1\\[0.5em]
\phantom \ \ \ \ \ \ \ \ \ \ \ \ \ \ \ \ \ \ \ \ \ \ \ \ 1000 0000 \ \ \ +128?

\begin{picture}(0,0)
\put(47,57){\vector(1,0){16}}
\put(127,57){\vector(1,0){16}}
\put(147,35){\line(1,0){57}}
\put(208,24){\vector(1,0){16}}
\end{picture}\\[-2em]

que está errado, pois é a representação de -128, não de\\
+128.
\item O número zero tem apenas uma representação: 0000 0000.\\
De fato, tem-se:\\
\phantom \ \ \ \ \ \ \ 0000 0000 \ \ \ \ 1111 1111\\[-1em]
\phantom \ \ \ \ \ \ \ \ \ \ \ \ \ \ \ \ \ \ \ \ \ \ \ \ \ \ \ \ \ \ +\\[-1em]
\phantom \ \ \ \ \ \ \ \ \ \ \ \ \ \ \ \ \ \ \ \ \ \ \ \ \ \ \ \ \ 1\\[0.5em]
\phantom \ \ \ \ \ \ \ desprezado \ \ 10000 0000 \ \ \ \ \ 0000 0000

\begin{picture}(0,0)
\put(104,57){\vector(1,0){22}}
\put(128,35){\line(1,0){57}}

\dashline{4}(40,32)(129,32)(129,16)(40,16)(40,32)

\put(108,23){\vector(1,0){13}}
\put(194,23){\vector(1,0){22}}
\end{picture}
\end{enumerate}
\end{enumerate}

\newpage

\phantomsection
\subsection*{6. \uline{Aritmética no Patinho Feio}}
\addcontentsline{toc}{subsection}{\hskip 5em Aritmética no Patinho Feio}

Já foi visto como somar números binários positivos e\\
como representar números negativos em oito bits. Deste \ modo ,\\
pode-se passar à soma (e subtração) de números de oito bits \ ,\\
que é o que o Patinho Feio consegue fazer diretamente \ através\\
de seus circuitos eletrônicos. Ver-se-á também, como operar com\\
números de mais de oito bits e como reconhecer quando o resul-\\
tado de uma soma não pode ser representado em oito bits ( isto\\
é, o número é menor que -128 ou maior que +127).\\

\begin{enumerate}[label=\alph*), align=left, leftmargin=1.5em, labelsep=-0.5em, itemsep=1em, topsep=1.5em]
\item Vai-um:\\
\\[-0.5em]
Denomina-se \uline{vai-um} de uma soma entre dois numeros de\hfill oito\\
bits ao vai-um na última soma realizada (bit mais signific\uline a\\
tivo), ou seja, ao que seria o nono bit da soma (se \ fossem\\
considerados números de nove bits). Exemplo:\\
\end{enumerate}







\newpage

\phantomsection
\section*{\hskip 2.5em g) Exemplo de Programa Absoluto:}
\addcontentsline{toc}{section}{Exemplo de Programa Absoluto}

\setstretch{0.8}

\noindent \\[1em]

\noindent MEM \ E00\\
LER \ E01\\
LEY \ E7C\\
ARM \ E34\\
SAI \ E72\\
LIR \ E23\\
LEX \ E25\\
GUA \ E31\\
PRE \ E4F\\
VRC \ E49\\
CAI \ E4D\\
ENA \ E5F\\
WAT \ E65\\
WFF \ E76\\
IGR \ E7E\\
ACC \ E84\\
LOP \ E86\\
DIS \ EC5\\
ACH \ EB2\\
BRO \ EB5\\
ARA \ EB8\\
\\
\\
/00 SI\\
\\
\ding{122}\hskip 0.05em\hskip 0.05em PASSO2\\
\\
\phantom \ \ \ 1 \ @BLTC\\
\phantom \ \ \ 2 \ E00 \ \ \ \ \ \ \ \ \ \ \ \ \ \ \ ORG \ \ \ \ /E00\\
\phantom \ \ \ 3 \ *\\
\phantom \ \ \ 4 \ * HEXAM - PROGRAMA QUE CARREGA A MEMORIA\\
\phantom \ \ \ 5 \ * \ \ \ \ \ \ \ \ A PARTIR DE DADOS FORNECIDOS\\
\phantom \ \ \ 6 \ * \ \ \ \ \ \ \ \ EM HEXADECIMAL PELA CONSOLE\\
\phantom \ \ \ 7 \ *\\
\phantom \ \ \ 8 \ ********************************************************************\\
\phantom \ \ \ 9 \ *\\
\phantom \ \ 10 \ * \ \ HEXAM - INSTRUCOES DE UTILIZACAO:\\
\phantom \ \ 11 \ *\\
\phantom \ \ 12 \ * 1. ENDERECAR HEXAR\\
\phantom \ \ 13 \ * 2. DAR PARTIDA\\
\phantom \ \ 14 \ * 3. O CANAL B VAI FICAR ESPERANDO ENDERECAMENTO.\\
\phantom \ \ 15 \ * 4. PARA ENDERECAR A QUALQUER MOMENTO, BATER ARROBA (@).\\
\phantom \ \ 16 \ * 5. O COMPUTADOR RESPONDE C/ RETURN, 2 LINEFEEDS.\\
\phantom \ \ 17 \ * 6. ENTRAR C/ ENDERECO EM HEXA, COM 3 DIGITOS\\
\phantom \ \ 18 \ * 7. SE ERRAR, BASTA VOLTAR P/ 4 OU BATER UM BRANCO.\\
\phantom \ \ 19 \ * \ \ \ NESTE CASO, O PROGRAMA IGNORA A ENTRADA ANTERIOR\\
\phantom \ \ 20 \ * \ \ \ E AGUARDA NOVO ENDERECO.\\
\phantom \ \ 21 \ * 8. UMA VEZ ENDERECADO, OS DADOS QUE FOREM FORNECIDOS\\
\phantom \ \ 22 \ * \ \ \ SERAO GUARDADOS EM SEQUENCIA A PARTIR DO ENDERECO\\
\phantom \ \ 23 \ * \ \ \ ESPECIFICADO.\\
\phantom \ \ 24 \ * 9. OS DADOS DEVERAO VIR SEPARADOS POR UM UNICO BRANCO.

\newpage

\noindent \\

\noindent \ \ 25 \ *10. O ULTIMO DADO DA LINHA NAO DEVE SER SEGUIDO DE BRANCO,\\
\phantom \ \ 26 \ * \ \ \ SENDO QUE NESTE CASO UM LINEFEED, RETURN OU VICE\\
\phantom \ \ 27 \ * \ \ \ VERSA O SUBSTITUIRA'.\\
\phantom \ \ 28 \ *11. UM BRANCO OU RETURN DEPOIS DO DADO E' UMA ORDEM P/\\
\phantom \ \ 29 \ * \ \ \ QUE O DADO SEJA ARMAZENADO.\\
\phantom \ \ 30 \ *12. DEPOIS DE CADA BRANCO OU RETURN O BUFFER E' ZERADO,\\
\phantom \ \ 31 \ * \ \ \ E PORTANTO SE FOREM DADOS 2 BRANCOS EM SEQUENCIA\\
\\
\\
\\
\\
\phantom \ \ 32 \ * \ \ \ SERA' GUARDADO UM ZERO NO LUGAR DO SEGUNDO BRANCO.\\
\phantom \ \ 33 \ *13. EM CASO DE ERRO NOS DADOS, SE O CARACTER FORNECIDO\\
\phantom \ \ 34 \ * \ \ \ FOR HEXADECIMAL,BASTA BATER DE NOVO EM SEGUIDA, SEM\\
\phantom \ \ 35 \ * \ \ \ BRANCOS, O DADO CORRETO. SO' SAO GUARDADOS NA MEMORIA\\
\phantom \ \ 36 \ * \ \ \ OS DOIS ULTIMOS DIGITOS.\\
\phantom \ \ 37 \ *14. SE O CARATER NAO FOR HEXADECIMAL, O COMPUTADOR RESPONDE\\
\phantom \ \ 38 \ * \ \ \ COM UMA SETA (\_) E PARA O PROESSAMENTO.\\
\phantom \ \ 39 \ *15. NESTE CASO, DANDO PARTIDA, O PROGRAMA VOLTA A SER\\
\phantom \ \ 40 \ * \ \ \ EXECUTADO COMO NO CASO 14.\\
\phantom \ \ 41 \ *16. ANTES DE DAR ENDERECAMENTO, E' PRECISO NAO ESQUECER\\
\phantom \ \ 42 \ * \ \ \ DE GUARDAR O DADO ANTERIOR. SE NAO FOR DADO UM BRANCO\\
\phantom \ \ 43 \ * \ \ \ OU RETURN, O DADO NAO SERA' ARMAZENADO.\\
\phantom \ \ 44 \ *\\
\phantom \ \ 45 \ ********************************************************************\\
\phantom \ \ 46 \ *\\
\phantom \ \ 47 \ E00 9A \ \ \ \ HEXAM \ \ INIB \ \ \ * \ \ \ \ \ \ INIBE INTERRUPCAO\\
\phantom \ \ 48 \ *\\
\phantom \ \ 49 \ * SECAO DE LEITURA DE ENDERECO\\
\phantom \ \ 50 \ *\\
\phantom \ \ 51 \ E01 FE 7C \ LEENDER PUG \ \ \ \ LECONV \ LE PRIMEIRO CAR. DO END.\\
\phantom \ \ 52 \ E03 AE 01 \ \ \ \ \ \ \ \ \ PLAN \ \ \ LEENDER SE BCO. OU RETURN, VOLTA\\
\phantom \ \ 53 \ E05 D2 20 \ \ \ \ \ \ \ \ \ XOR \ \ \ \ /20 \ \ \ \ NAO| MONTA "ARM"\\
\phantom \ \ 54 \ E07 2E 34 \ \ \ \ \ \ \ \ \ ARM \ \ \ \ ARM \ \ \ \ GUARDA P/EXECUTAR\\
\phantom \ \ 55 \ E09 FE 7C \ \ \ \ \ \ \ \ \ PUG \ \ \ \ LECONV \ LE SEG. CARATER\\
\phantom \ \ 56 \ E0B AE 01 \ \ \ \ \ \ \ \ \ PLAN \ \ \ LEENDER SE BCO, VOLTA A LER ENDERECO\\
\phantom \ \ 57 \ E0D D1 4F \ \ \ \ \ \ \ \ \ DE \ \ \ \ \ 4 \ \ \ \ \ \ AJEITA P/ COMPOR\\
\phantom \ \ 58 \ E0F 2E 35 \ \ \ \ \ \ \ \ \ ARM \ \ \ \ ARM-1 \ \ GUARDA\\
\phantom \ \ 59 \ E11 FE 7C \ \ \ \ \ \ \ \ \ PUG \ \ \ \ LECONV \ LE TERCEIRO CARATER\\
\phantom \ \ 60 \ E13 AE 01 \ \ \ \ \ \ \ \ \ PLAN \ \ \ LEENDER SE BCO, VOLTA A LER END.\\
\phantom \ \ 61 \ E15 6E 35 \ \ \ \ \ \ \ \ \ SOM \ \ \ \ ARM+1 \ \ SE NAO COMPOE COM SEGUNDO\,\,\,DIG.\\
\phantom \ \ 62 \ E17 2E 35 \ \ \ \ \ \ \ \ \ ARM \ \ \ \ ARM+1 \ \ GUARDA P/ EXECUTAR\\
\phantom \ \ 63 \ E19 DA 0D \ \ \ \ \ \ \ \ \ CARI \ \ \ /0D \ \ \ \ SAI RETURN\\
\phantom \ \ 64 \ E1B FE 72 \ \ \ \ \ \ \ \ \ PUG \ \ \ \ SAI \ \ \ \ NA TTY\\
\phantom \ \ 65 \ E1D DA 0A \ \ \ \ \ \ \ \ \ CARI \ \ \ /0A \ \ \ \ SAI LINEFEED\\
\phantom \ \ 66 \ E1F FE 72 \ \ \ \ \ \ \ \ \ PUG \ \ \ \ SAI \ \ \ \ NA TTY\\
\phantom \ \ 67 \ E21 FE 72 \ \ \ \ \ \ \ \ \ PUG \ \ \ \ SAI \ \ \ \ IDEM\\
\phantom \ \ 68 \ *\\
\phantom \ \ 69 \ * LEITURA DE UMA PALAVRA

\newpage

\noindent \\[2em]

\noindent \ \ 70 \ *\\
\phantom \ \ 71 \ E23 80 \ \ \ \ LIMPA \ \ LIMPO \ \ * \ \ \ \ \ \ ZERA\\
\phantom \ \ 72 \ E24 99 \ \ \ \ \ \ \ \ \ \ \ \ TRE \ \ \ \ * \ \ \ \ \ \ EXTENSAO\\
\phantom \ \ 73 \ E25 FE 7C \ LEPROX \ PUG \ \ \ \ LECONV \ LE UM CARATER\\
\phantom \ \ 74 \ E27 AE 31 \ \ \ \ \ \ \ \ \ PLAN \ \ \ GUARDA \ SE BCO OU LINEFEED, STORE.\\
\phantom \ \ 75 \ E29 99 \ \ \ \ \ \ \ \ \ \ \ \ TRE \ \ \ \ * \ \ \ \ \ \ SE NAO, TRAZ EXTENSAO\\
\phantom \ \ 76 \ E2A 01 4F \ \ \ \ \ \ \ \ \ DE \ \ \ \ \ * \ \ \ \ \ \ AJEITA PARA COMPOR\\
\phantom \ \ 77 \ E2C 60 01 \ \ \ \ \ \ \ \ \ SOM \ \ \ \ /001 \ \ \ COMPOE\\
\phantom \ \ 78 \ E2E 99 \ \ \ \ \ \ \ \ \ \ \ \ TRE \ \ \ \ * \ \ \ \ \ \ C/ DIGITO LIDO\\
\phantom \ \ 79 \ E2F 0E 25 \ \ \ \ \ \ \ \ \ PLA \ \ \ \ LEPROX \ CONTINUA LENDO ATE' ACHAR\,\,\,BCO/RET\\
\phantom \ \ 80 \ *\\
\phantom \ \ 81 \ * ARMAZENAMENTO NA MEMORIA\\
\phantom \ \ 82 \ *\\
\phantom \ \ 83 \ E31 FE 4F \ GUARDA \ PUG \ \ \ \ PROTEGE TESTA SE ENDERECO INVADE\\
\phantom \ \ 84 \ E33 99 \ \ \ \ \ \ \ \ \ \ \ \ TRE \ \ \ \ * \ \ \ \ \ \ HEXAM. SE NAO,\\
\phantom \ \ 85 \ E34 20 00 \ ARM \ \ \ \ ARM \ \ \ \ *-* \ \ \ \ GUARDA EXTENSAO NO ENDERECO\,\,\,CONV.\\
\phantom \ \ 86 \ E36 4E 35 \ \ \ \ \ \ \ \ \ CAR \ \ \ \ ARM+1 \ \ INCREMENTA\\
\phantom \ \ 87 \ E38 85 \ \ \ \ \ \ \ \ \ \ \ \ INC \ \ \ \ * \ \ \ \ \ \ SEGUNDA PALAVRA\\
\phantom \ \ 88 \ E39 2E 35 \ \ \ \ \ \ \ \ \ ARM \ \ \ \ ARM+1 \ \ DO ENDERECO.\\
\phantom \ \ 89 \ E3B 96 \ \ \ \ \ \ \ \ \ \ \ \ SV \ \ \ \ \ 1 \ \ \ \ \ \ SE NAO DEU CARRY,\\
\phantom \ \ 90 \ E3C 0E 23 \ \ \ \ \ \ \ \ \ PLA \ \ \ \ LIMPA \ \ VAI LER A PROXIMA PALAVRA\\
\phantom \ \ 91 \ E3E 4E 34 \ \ \ \ \ \ \ \ \ CAR \ \ \ \ ARM \ \ \ \ SE DEU,\\
\phantom \ \ 92 \ E40 85 \ \ \ \ \ \ \ \ \ \ \ \ INC \ \ \ \ * \ \ \ \ \ \ INCREMENTA PRIMEIRA PALAVRA\\
\phantom \ \ 93 \ E41 2E 34 \ \ \ \ \ \ \ \ \ ARM \ \ \ \ ARM \ \ \ \ GUARDA\\
\phantom \ \ 94 \ E43 D1 0F \ \ \ \ \ \ \ \ \ DD \ \ \ \ \ 4 \ \ \ \ \ \ TESTA SE DEU CARRY\\
\phantom \ \ 95 \ E45 D8 FE \ \ \ \ \ \ \ \ \ SOMI \ \ \ -/02 \ \ \ ALEM DO BIT 11\\
\phantom \ \ 96 \ E47 BE 23 \ \ \ \ \ \ \ \ \ PLAZ \ \ \ LIMPA \ \ NAO: VAI LER PROXIMA PAL.\\
\phantom \ \ 97 \ E49 86 \ \ \ \ UNEG \ \ \ UNEG \ \ \ * \ \ \ \ \ \ SIM: PARA EM LOOP\\
\phantom \ \ 98 \ E4A 9D \ \ \ \ \ \ \ \ \ \ \ \ PARE \ \ \ * \ \ \ \ \ \ COM /FF\\
\phantom \ \ 99 \ E4B 0E 49 \ \ \ \ \ \ \ \ \ PLA \ \ \ \ UNEG \ \ \ NO ACUMULADOR\\
\phantom \ 100 \ *\\
\phantom \ 101 \ *\\
\phantom \ 102 \ * PROTECAO DO PROGRAMA PARA QUE NAO SEJA DESTRUIDO PELOS\,\,\,DADOS\\
\phantom \ 103 \ * \ \ \ \ \ \ \ \ \ \ \ \ \ \ \ \ \ \ \ \ \ OU POR ENDERECAMENTO INVALIDO.\\
\phantom \ 104 \ *\\
\phantom \ 105 \ *\\
\phantom \ 106 \ E4D DA 00 \ CARI \ \ \ CARI \ \ \ /00 \ \ \ \ RESTAURA ACUMULADOR\\
\phantom \ 107 \ *\\
\phantom \ 108 \ E4F 00 00 \ PROTEGE PLA \ \ \ \ 0 \ \ \ \ \ \ RETORNA\\
\phantom \ 109 \ E51 2E 4E \ \ \ \ \ \ \ \ \ ARM \ \ \ \ CARI+1 \ SALVA ACUMULADOR\\
\phantom \ 110 \ E53 4E 34 \ \ \ \ \ \ \ \ \ CAR \ \ \ \ ARM \ \ \ \ SEPARA 4 BITS\\
\phantom \ 111 \ E55 D1 4F \ \ \ \ \ \ \ \ \ DE \ \ \ \ \ 4 \ \ \ \ \ \ MAIS SIGNIFICATIVOS\\
\phantom \ 112 \ E57 D1 6F \ \ \ \ \ \ \ \ \ GE \ \ \ \ \ 4 \ \ \ \ \ \ DO ENDERECO\\
\phantom \ 113 \ E59 D8 F2 \ \ \ \ \ \ \ \ \ SOMI \ \ \ -/0E \ \ \ TESTA SE ENDERECO >= /E00\\
\phantom \ 114 \ E5B AE 4D \ \ \ \ \ \ \ \ \ PLAN \ \ \ CARI \ \ \ NAO: VAI RETORNAR\\
\phantom \ 115 \ E5D 0E 49 \ \ \ \ \ \ \ \ \ PLA \ \ \ \ UNEG \ \ \ SIM: VAI PARAR EM LOOP\\
\phantom \ 116 \ *\\
\phantom \ 117 \ * ENTRAS DRIVER DE ENTRADA DE DADOS, SEM BIT DE PARIDADE
 
\newpage

\noindent \\[2em]

\noindent \ 118 \ *\\
\phantom \ 119 \ E5F 00 00 \ ENTRA \ \ PLA \ \ \ \ 0\\
\phantom \ 120 \ E61 CB 11 \ \ \ \ \ \ \ \ \ FNC \ \ \ \ /81 \ \ \ \ CLF\\
\phantom \ 121 \ E63 CB 16 \ \ \ \ \ \ \ \ \ FNC \ \ \ \ /86 \ \ \ \ STC\\
\phantom \ 122 \ E65 CB 21 \ WAIT \ \ \ SAL \ \ \ \ /B1 \ \ \ \ ESPERA\\
\phantom \ 123 \ E67 0E 65 \ \ \ \ \ \ \ \ \ PLA \ \ \ \ WAIT \ \ \ FLAG\\
\phantom \ 124 \ E69 CB 40 \ \ \ \ \ \ \ \ \ ENTR \ \ \ /B0 \ \ \ \ ENTRA DADO COMPLEMENTADO\\
\phantom \ 125 \ E6B 82 \ \ \ \ \ \ \ \ \ \ \ \ CMP1 \ \ \ * \ \ \ \ \ \ DESCOMPLEMENTA\\
\phantom \ 126 \ E6C D1 41 \ \ \ \ \ \ \ \ \ DE \ \ \ \ \ 1 \ \ \ \ \ \ LIMPA PARIDADE\\
\phantom \ 127 \ E6E D1 21 \ \ \ \ \ \ \ \ \ GD \ \ \ \ \ 1 \ \ \ \ \ \ DO DADO\\
\phantom \ 128 \ E70 0E 5F \ \ \ \ \ \ \ \ \ PLA \ \ \ \ ENTRA \ \ RETORNA\\
\phantom \ 129 \ *\\
\phantom \ 130 \ * SAI - DRIVER DE SAIDA\\
\phantom \ 131 \ *\\
\phantom \ 132 \ E72 00 00 \ SAI \ \ \ \ PLA \ \ \ \ 0\\
\phantom \ 133 \ E74 CB 80 \ \ \ \ \ \ \ \ \ SAI \ \ \ \ /B0 \ \ \ \ SAI DADO\\
\phantom \ 134 \ E76 CB 21 \ WFF \ \ \ \ SAL \ \ \ \ /B1 \ \ \ \ ESPERA\\
\phantom \ 135 \ E78 0E 76 \ \ \ \ \ \ \ \ \ PLA \ \ \ \ WFF \ \ \ \ FLAG\\
\phantom \ 136 \ E7A 0E 72 \ \ \ \ \ \ \ \ \ PLA \ \ \ \ SAI \ \ \ \ RETORNA\\
\phantom \ 137 \ *\\
\phantom \ 138 \ * LECONV = ROTINA DE "CONVERSAO" HEXBIN\\
\phantom \ 139 \ *\\
\phantom \ 140 \ E7C 00 00 \ LECONV \ PLA \ \ \ \ 0\\
\phantom \ 141 \ E7E DA 0F \ IGNOR \ \ CARI \ \ \ /0F \ \ \ \ FAZ INDICE\\
\phantom \ 142 \ E80 9E \ \ \ \ \ \ \ \ \ \ \ \ TRI \ \ \ \ * \ \ \ \ \ \ \_/0F(NUMERO DE DIGITOS)\\
\phantom \ 143 \ E81 FE 5F \ \ \ \ \ \ \ \ \ PUG \ \ \ \ ENTRA \ \ OBTEM DADO\\
\phantom \ 144 \ E83 83 \ \ \ \ \ \ \ \ \ \ \ \ CMP2 \ \ \ * \ \ \ \ \ \ TROCA SINAL\\
\phantom \ 145 \ E84 2E C4 \ \ \ \ \ \ \ \ \ ARM \ \ \ \ ACC \ \ \ \ SALVA DADO COMPLEMENTADO\\
\phantom \ 146 \ E86 4E C4 \ LOOP \ \ \ CAR \ \ \ \ ACC \ \ \ \ CARREGA DADO COMPLEMENTADO\\
\phantom \ 147 \ E88 BE 7E \ \ \ \ \ \ \ \ \ PLAZ \ \ \ IGNOR \ \ SE FFRM, IGNORA-O\\
\phantom \ 148 \ E8A 7E C5 \ \ \ \ \ \ \ \ \ SOMX \ \ \ DIGITOS TESTA SE E' DIGITO\\
\phantom \ 149 \ E8C BE 82 \ \ \ \ \ \ \ \ \ PLAZ \ \ \ ACH \ \ \ \ SE FOR, => ACH\\
\phantom \ 150 \ E8E E0 00 \ \ \ \ \ \ \ \ \ S?S \ \ \ \ 0 \ \ \ \ \ \ SE NAO, APONTAR PROXIMO\\
\phantom \ 151 \ E90 0E 86 \ \ \ \ \ \ \ \ \ PLA \ \ \ \ LOOP \ \ \ E VAI TESTAR NOVAMENTE\\
\phantom \ 152 \ E92 4E C4 \ \ \ \ \ \ \ \ \ CAR \ \ \ \ ACC \ \ \ \ SE NAO DIGITO, RECUPERA\,\,\,DADO\\
\phantom \ 153 \ E94 D8 20 \ \ \ \ \ \ \ \ \ SOMI \ \ \ @ \ \ \ \ \ \ TESTA SE E' BRANCO\\
\phantom \ 154 \ E96 BE B5 \ \ \ \ \ \ \ \ \ PLAZ \ \ \ BRANCO \ SIM: => BRANCO\\
\phantom \ 155 \ E98 4E C4 \ \ \ \ \ \ \ \ \ CAR \ \ \ \ ACC \ \ \ \ NAO=> CARREGA DADO\\
\phantom \ 156 \ E9A D8 0D \ \ \ \ \ \ \ \ \ SOMI \ \ \ /0D \ \ \ \ TESTA SE E' RETURN\\
\phantom \ 157 \ E9C BE B5 \ \ \ \ \ \ \ \ \ PLAZ \ \ \ BRANCO \ SIM=> BRANCO\\
\phantom \ 158 \ E9E 4E C4 \ \ \ \ \ \ \ \ \ CAI \ \ \ \ ACC \ \ \ \ NAO=> TESTA SE\\
\phantom \ 159 \ EA0 DB 0A \ \ \ \ \ \ \ \ \ SOMI \ \ \ /0A \ \ \ \ E' LINEFEED

\newpage

\noindent \\[2em]

\noindent \ 160 \ EA2 BE 7E \ \ \ \ \ \ \ \ \ PLAZ \ \ \ IGNOR \ \ SIM=> IGNORA\\
\phantom \ 161 \ EA4 4E C4 \ \ \ \ \ \ \ \ \ CAR \ \ \ \ ACC \ \ \ \ NAO=> TESTA SE\\
\phantom \ 162 \ EA6 D8 40 \ \ \ \ \ \ \ \ \ SOMI \ \ \ @@ \ \ \ \ \ ARROBA\\
\phantom \ 163 \ EA8 BE B8 \ \ \ \ \ \ \ \ \ PLAZ \ \ \ ARROBA \ SIM=> ARROBA\\
\phantom \ 164 \ EAA DA 3F \ \ \ \ \ \ \ \ \ CARI \ \ \ @\_ \ \ \ \ \ NAO=> INVALIDO!\\
\phantom \ 165 \ EAC FE 72 \ \ \ \ \ \ \ \ \ PUG \ \ \ \ SAI \ \ \ \ IMPRIME "\_" NA CONSOLE\\
\phantom \ 166 \ EAE 80 \ \ \ \ \ \ \ \ \ \ \ \ LIMPO \ \ * \ \ \ \ \ \ ZERA ACUMULADOR\\
\phantom \ 167 \ EAF 9D \ \ \ \ \ \ \ \ \ \ \ \ PARE \ \ \ * \ \ \ \ \ \ PARA\\
\phantom \ 168 \ EB0 0E 7E \ \ \ \ \ \ \ \ \ PLA \ \ \ \ IGNOR \ \ E IGNORA O CARATER\\
\phantom \ 169 \ EB2 9E \ \ \ \ ACH \ \ \ \ TRI \ \ \ \ * \ \ \ \ \ \ (INDICE=DIGITO CONVERTIDO)\\
\phantom \ 170 \ EB3 0E 7C \ \ \ \ \ \ \ \ \ PLA \ \ \ \ LECONV \ JOGA NO ACC E VOLTA\\
\phantom \ 171 \ EB5 86 \ \ \ \ BRANCO \ UNEG \ \ \ * \ \ \ \ \ \ RETORNA C/ -1\\
\phantom \ 172 \ EB6 0E 7C \ \ \ \ \ \ \ \ \ PLA \ \ \ \ LECONV \ SE FOR BRANCO OU RETURN\\
\phantom \ 173 \ *\\
\phantom \ 174 \ EB8 DA 0D \ ARROBA \ CARI \ \ \ /0D \ \ \ \ SAI\\
\phantom \ 175 \ EBA FE 72 \ \ \ \ \ \ \ \ \ PUG \ \ \ \ SAI \ \ \ \ RETURN\\
\phantom \ 176 \ EBC DA 0A \ \ \ \ \ \ \ \ \ CARI \ \ \ /0A \ \ \ \ SAI\\
\phantom \ 177 \ EBE FE 72 \ \ \ \ \ \ \ \ \ PUG \ \ \ \ SAI \ \ \ \ DOIS\\
\phantom \ 178 \ EC0 FE 72 \ \ \ \ \ \ \ \ \ PUG \ \ \ \ SAI \ \ \ \ LINEFEEDS\\
\phantom \ 179 \ EC2 0E 01 \ \ \ \ \ \ \ \ \ PLA \ \ \ \ LEENDER VOLTA A LER ENDERECO.\\
\phantom \ 180 \ *\\
\phantom \ 181 \ * BUFFERS E CONSTANTES\\
\phantom \ 182 \ *\\
\phantom \ 183 \ *\\
\phantom \ 184 \ EC4 \ 00 \ \ \ ACC \ \ \ \ DEFC \ \ \ 0 \ \ \ \ \ \ P/ SALVAR ACUMULADOR\\
\phantom \ 185 \ EC5 \ 30 \ \ \ DIGITOS DEFC \ \ \ @0\\
\phantom \ 186 \ EC6 \ 31 \ \ \ \ \ \ \ \ \ \ \ DEFC \ \ \ @1\\
\phantom \ 187 \ EC7 \ 32 \ \ \ \ \ \ \ \ \ \ \ DEFC \ \ \ @2\\
\phantom \ 188 \ EC8 \ 33 \ \ \ \ \ \ \ \ \ \ \ DEFC \ \ \ @3\\
\phantom \ 189 \ EC9 \ 34 \ \ \ \ \ \ \ \ \ \ \ DEFC \ \ \ @4\\
\phantom \ 190 \ ECA \ 35 \ \ \ \ \ \ \ \ \ \ \ DEFC \ \ \ @5\\
\phantom \ 191 \ ECB \ 36 \ \ \ \ \ \ \ \ \ \ \ DEFC \ \ \ @6\\
\phantom \ 192 \ ECC \ 37 \ \ \ \ \ \ \ \ \ \ \ DEFC \ \ \ @7\\
\phantom \ 193 \ ECD \ 38 \ \ \ \ \ \ \ \ \ \ \ DEFC \ \ \ @8\\
\phantom \ 194 \ ECE \ 39 \ \ \ \ \ \ \ \ \ \ \ DEFC \ \ \ @9\\
\phantom \ 195 \ ECF \ 41 \ \ \ \ \ \ \ \ \ \ \ DEFC \ \ \ @A\\
\phantom \ 196 \ ED0 \ 42 \ \ \ \ \ \ \ \ \ \ \ DEFC \ \ \ @B\\
\phantom \ 197 \ ED1 \ 43 \ \ \ \ \ \ \ \ \ \ \ DEFC \ \ \ @C\\
\phantom \ 198 \ ED2 \ 44 \ \ \ \ \ \ \ \ \ \ \ DEFC \ \ \ @D\\
\phantom \ 199 \ ED3 \ 45 \ \ \ \ \ \ \ \ \ \ \ DEFC \ \ \ @E\\
\phantom \ 200 \ ED4 \ 46 \ \ \ \ \ \ \ \ \ \ \ DEFC \ \ \ @F\\
\ding{122}\hskip 0.05em\hskip 0.05em\ding{122}\hskip 0.05em\hskip 0.05em\ding{122}\hskip 0.05em\hskip 0.05em\ding{122}\hskip 0.05em\hskip 0.05em\ding{122}\hskip 0.05em\hskip 0.05em\ding{122}\hskip 0.05em\hskip 0.05em\ding{122}\hskip 0.05em\hskip 0.05em\ding{122}\hskip 0.05em\hskip 0.05em\ding{122}\hskip 0.05em\hskip 0.05em\ding{122}\hskip 0.05em\hskip 0.05em\ding{122}\hskip 0.05em\hskip 0.05em\ding{122}\hskip 0.05em\hskip 0.05em\ding{122}\hskip 0.05em\hskip 0.05em\ding{122}\hskip 0.05em\hskip 0.05em\ding{122}\hskip 0.05em\hskip 0.05em\ding{122}\hskip 0.05em\hskip 0.05em\ding{122}\hskip 0.05em\hskip 0.05em\ding{122}\hskip 0.05em\hskip 0.05em\ding{122}\hskip 0.05em\hskip 0.05em\ding{122}\hskip 0.05em\hskip 0.05em\ding{122}\hskip 0.05em\hskip 0.05em\ding{122}\hskip 0.05em\hskip 0.05em\ding{122}\hskip 0.05em\hskip 0.05em\ding{122}\hskip 0.05em\hskip 0.05em\ding{122}\hskip 0.05em\hskip 0.05em\ding{122}\hskip 0.05em\hskip 0.05em\ding{122}\hskip 0.05em\hskip 0.05em\ding{122}\hskip 0.05em\hskip 0.05em\ding{122}\hskip 0.05em\hskip 0.05em\ding{122}\hskip 0.05em\hskip 0.05em\ding{122}\hskip 0.05em\hskip 0.05em\ding{122}\hskip 0.05em\hskip 0.05em\ding{122} 201 \ 000 \ \ \ \ \ \ \ \ \ \ \ \ \ \ \ FIM \ \ \ \ \ 0

\newpage

\phantomsection
\section*{\hskip 0.5em h) Exemplos de Progamas Relocáveis:}
\addcontentsline{toc}{section}{Exemplos de Progamas Relocáveis}

\noindent \\[6em]

\noindent CON \ 000 \ ENT\\
SUB \ *** \ EXT\\
ARF \ *** \ EXT\\
CAF \ *** \ EXT\\
SEN \ *** \ EXT\\
MAT \ 00E \ ABS\\
PIS \ 012\\
ACF \ 00A \ ABS\\
OFW \ 012 \ ABS\\
\\
\\
/00 SI\\
\\
PASSO2\\
\\
\\
\phantom \ \ \ 1 \ \ \ \ \ \ \ \ \ \ \ \ \ \ @BLT\\
\phantom \ \ \ 2 \ 000 \ \ \ \ \ \ \ \ \ \ \ \ \ \ \ \ \ SUBR \ \ \ COSEN\\
\phantom \ \ \ 3 \ \ \ \ \ \ \ \ \ \ \ \ \ \ * ROTINA QUE CALCULA O COSENO NO PATINHO\\
\phantom \ \ \ 4 \ \ \ \ \ \ \ \ \ \ \ \ \ \ * PELA FORMULA COS(X)= SEN(PI/2 - X)\\
\phantom \ \ \ 5 \ \ \ \ \ \ \ \ \ \ \ \ \ \ *\\
\phantom \ \ \ 6 \ 000 \ \ \ \ \ \ \ \ \ \ \ \ \ \ \ \ \ EXT \ \ \ \ COSEN\\
\phantom \ \ \ 7 \ 000 \ \ \ \ \ \ \ \ \ \ \ \ \ \ \ \ \ EXT \ \ \ \ SUB\\
\phantom \ \ \ 8 \ 000 \ \ \ \ \ \ \ \ \ \ \ \ \ \ \ \ \ EXT \ \ \ \ ARMACF\\
\phantom \ \ \ 9 \ 000 \ \ \ \ \ \ \ \ \ \ \ \ \ \ \ \ \ EXT \ \ \ \ CARACF\\
\phantom \ \ 10 \ 000 \ \ \ \ \ \ \ \ \ \ \ \ \ \ \ \ \ EXT \ \ \ \ SEN\\
\phantom \ \ 11 \ 000 00 00 \ \ \ COSEN \ \ PLA \ \ \ \ 0\\
\phantom \ \ 12 \ 002 F0 00 X \ \ \ \ \ \ \ \ \ PUG \ \ \ \ ARMACF\\
\phantom \ \ 13 \ 004 01 \ \ \ \ \ \ \ \ \ \ \ \ \ \ DEFC \ \ \ 1\\
\phantom \ \ 14 \ 005 00 0E \ \ \ \ \ \ \ \ \ \ \ DEFE \ \ \ MANT\\
\phantom \ \ 15 \ 007 F0 00 X \ \ \ \ \ \ \ \ \ PUG \ \ \ \ CARACF\\
\phantom \ \ 16 \ 009 01 \ \ \ \ \ \ \ \ \ \ \ \ \ \ DEFC \ \ \ 1\\
\phantom \ \ 17 \ 00A 00 12 R \ \ \ \ \ \ \ \ \ DEFE \ \ \ PISDOIS\\
\phantom \ \ 18 \ 00C F0 00 X \ \ \ \ \ \ \ \ \ PUG \ \ \ \ SUB\\
\phantom \ \ 19 \ 00E F0 00 X \ \ \ \ \ \ \ \ \ PUG \ \ \ \ SEN\\
\phantom \ \ 20 \ 010 00 00 R \ \ \ \ \ \ \ \ \ PLA \ \ \ \ COSEN\\
\phantom \ \ 21 \ 00A \ \ \ \ \ \ \ \ \ ACF \ \ \ \ EQU \ \ \ \ /00A\\
\phantom \ \ 22 \ 00E \ \ \ \ \ \ \ \ \ MANT \ \ \ EQU \ \ \ \ /00E\\
\phantom \ \ 23 \ 012 \ \ \ \ \ \ \ \ \ OFLOW \ \ EQU \ \ \ \ /012\\
\phantom \ \ 24 \ 012 64 \ \ \ \ \ \ PISDOIS DEFC \ \ \ /64\\
\phantom \ \ 25 \ 013 87 \ \ \ \ \ \ \ \ \ \ \ \ \ \ DEFC \ \ \ /87\\
\phantom \ \ 26 \ 014 D0 \ \ \ \ \ \ \ \ \ \ \ \ \ \ DEFC \ \ \ /D0\\
\phantom \ \ 27 \ 015 01 \ \ \ \ \ \ \ \ \ \ \ \ \ \ DEFC \ \ \ /01\\
\phantom \ \ 28 \ 000 \ \ \ \ \ \ \ \ \ \ \ \ \ \ \ \ \ PLA

\newpage

\noindent \\[6em]

\noindent DIY \ 000 \ ENT\\
SAA \ *** \ EXT\\
NOM \ *** \ EXT\\
NAM \ *** \ EXT\\
TAB \ *** \ EXT\\
ARF \ *** \ EXT\\
SGL \ *** \ EXT\\
COM \ *** \ EXT\\
SOI \ *** \ EXT\\
CAF \ *** \ EXT\\
SHL \ *** \ EXT\\
TAC \ *** \ EXT\\
SHR \ *** \ EXT\\
PON \ *** \ EXT\\
RET \ *** \ EXT\\
OFW \ 012 \ ABS\\
ZEO \ 017 \ ABS\\
FOO \ 01A \ ABS\\
ACF \ 00A \ ABS\\
MAT \ 00E \ ABS\\
DFT \ 01E \ ABS\\
GUI \ 016\\
GOG \ 078\\
SOS \ 05A\\
MOE \ 030\\
YES \ 045\\
GOL \ 06E\\
\\
\\
/00 SI\\
\\
PASSO2

\newpage

\noindent \ \ \ 1 \ \ \ \ \ \ \ \ \ \ \ \ \ \ @BLT\\ 
\phantom \ \ \ 2 \ 000 \ \ \ \ \ \ \ \ \ \ \ \ \ \ \ \ \ SUBR \ \ \ DIV\\
\phantom \ \ \ 3 \ \ \ \ \ \ \ \ \ \ \ \ \ \ *\\
\phantom \ \ \ 4 \ \ \ \ \ \ \ \ \ \ \ \ \ \ * \ DIV \ - \ ROTINA DE DIVISAO EM PONTO FLUTUANTE\\
\phantom \ \ \ 5 \ \ \ \ \ \ \ \ \ \ \ \ \ \ * \ \ \ \ \ \ \ \ \ \ \ \ \ \ \ \ \ \ ACF = ACF/MANT\\
\phantom \ \ \ 6 \ \ \ \ \ \ \ \ \ \ \ \ \ \ *\\
\phantom \ \ \ 7 \ 000 \ \ \ \ \ \ \ \ \ \ \ \ \ \ \ \ \ EXT \ \ \ \ DIV\\
\phantom \ \ \ 8 \ 000 \ \ \ \ \ \ \ \ \ \ \ \ \ \ \ \ \ EXT \ \ \ \ SALVA\\
\phantom \ \ \ 9 \ 000 \ \ \ \ \ \ \ \ \ \ \ \ \ \ \ \ \ EXT \ \ \ \ NORM\\
\phantom \ \ 10 \ 000 \ \ \ \ \ \ \ \ \ \ \ \ \ \ \ \ \ EXT \ \ \ \ NADABEM\\
\phantom \ \ 11 \ 000 \ \ \ \ \ \ \ \ \ \ \ \ \ \ \ \ \ EXT \ \ \ \ TAB\\
\phantom \ \ 12 \ 000 \ \ \ \ \ \ \ \ \ \ \ \ \ \ \ \ \ EXT \ \ \ \ ARMACF\\
\phantom \ \ 13 \ 000 \ \ \ \ \ \ \ \ \ \ \ \ \ \ \ \ \ EXT \ \ \ \ SGNAL\\
\phantom \ \ 14 \ 000 \ \ \ \ \ \ \ \ \ \ \ \ \ \ \ \ \ EXT \ \ \ \ COMPLEM\\
\phantom \ \ 15 \ 000 \ \ \ \ \ \ \ \ \ \ \ \ \ \ \ \ \ EXT \ \ \ \ SOMATRI\\
\phantom \ \ 16 \ 000 \ \ \ \ \ \ \ \ \ \ \ \ \ \ \ \ \ EXT \ \ \ \ CARACF\\
\phantom \ \ 17 \ 000 \ \ \ \ \ \ \ \ \ \ \ \ \ \ \ \ \ EXT \ \ \ \ SHIFTL\\
\phantom \ \ 18 \ 000 \ \ \ \ \ \ \ \ \ \ \ \ \ \ \ \ \ EXT \ \ \ \ TAC\\
\phantom \ \ 19 \ 000 \ \ \ \ \ \ \ \ \ \ \ \ \ \ \ \ \ EXT \ \ \ \ SHIFTR\\
\phantom \ \ 20 \ 000 \ \ \ \ \ \ \ \ \ \ \ \ \ \ \ \ \ EXT \ \ \ \ POESIN\\
\phantom \ \ 21 \ 000 \ \ \ \ \ \ \ \ \ \ \ \ \ \ \ \ \ EXT \ \ \ \ REST\\
\phantom \ \ 22 \ 012 \ \ \ \ \ \ \ \ \ OFLOW \ \ EQU \ \ \ \ /012\\
\phantom \ \ 23 \ 017 \ \ \ \ \ \ \ \ \ ZERO \ \ \ EQU \ \ \ \ /017\\
\phantom \ \ 24 \ 01A \ \ \ \ \ \ \ \ \ F \ \ \ \ \ \ EQU \ \ \ \ /01A\\
\phantom \ \ 25 \ 00A \ \ \ \ \ \ \ \ \ ACF \ \ \ \ EQU \ \ \ \ /00A\\
\phantom \ \ 26 \ 00E \ \ \ \ \ \ \ \ \ MANT \ \ \ EQU \ \ \ \ /00E\\
\phantom \ \ 27 \ 01E \ \ \ \ \ \ \ \ \ DFLOAT \ EQU \ \ \ \ /01E\\
\phantom \ \ 28 \ \ \ \ \ \ \ \ \ \ \ \ \ \ *\\
\phantom \ \ 29 \ 000 00 00 \ \ \ DIV \ \ \ \ PLA \ \ \ \ 0\\
\phantom \ \ 30 \ 002 F0 00 X \ \ \ \ \ \ \ \ \ PUG \ \ \ \ SALVA \ \ \ SALVA ACC,EXT,INDICE,T,V\\
\phantom \ \ 31 \ TODO: Finish transcribing this...\\

\end{document}