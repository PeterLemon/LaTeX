\documentclass[a4paper,12pt]{article}
\usepackage[a4paper, total={180mm, 272mm}]{geometry}

\usepackage{fontspec}
\setmainfont[Path=fonts/, Extension=.ttf]{ipaexm}

\setlength\parindent{3.5em}
\setlength\parskip{0em}
\renewcommand{\baselinestretch}{1.247}

\begin{document}

\thispagestyle{empty}

\Large
\noindent \\
Max Min Ino\medskip
\par
\normalsize
画像の明るい(暗い)部分を膨らませます。\\
\par
丸く(あるいは多角形に)ふくらみます。\par
また、変化は滑らかです。\\
\par
初めに、指定あれば Alphaチャンネルに対して処理します。\par
次に、Alphaチャンネルがゼロでない Pixel の RGBを処理します。\\
\\
-{-}- \ 入力 \ -{-}-\\
Source\par
処理をする画像を接続します。\\
Reference\par
Pixel 毎に膨らむ効果の強弱をつけるための参照画像を接続します。\\
\\
-{-}- \ 設定 \ -{-}-\\
Max Min Select\par
処理方法を撰択します。\par
\textquotedbl Max\textquotedbl \ -> 明るい部分を膨らませる\par
\textquotedbl Min\textquotedbl \ -> 暗い部分を膨らませる\par
\textquotedbl Min\textquotedbl の場合、セル画像の輪郭の墨線は、その外側の透過領域がゼ\par
ロで塗られているため、透過領域が膨らみ墨線が消えていきます。\par
同様に Alphaも透過領域が増します。\par
初期設定は\textquotedbl Max\textquotedbl です。\\
\\
Radius\par
膨らむ大きさを、円半径で指定します。\par
単位はミリメートルです。\par
ゼロ以上の値を指定します。\par
Smoothingと足して(ピクセル単位で)1より小さい時は膨らみません。\par
よって値が小さいと、細かい画像で効果があるのに、\par
荒い画像ではかからないことがあります。\par
半径は大きくするほど処理に時間がかかります。\\
\\
Polygon Number\par
円に膨らますか、多角形に膨らますかを指定します。\par
整数値で指定します。\par
2の時は丸く膨らみます。\par
3以上を指定すると、その角数の多角形に膨らみます。最大は16です。\par
膨らむ中心の真右から始まる多角形です。\par
初期値は2です。

\newpage

\thispagestyle{empty}

\ \vspace{-0.2em}
\\
\par
\noindent Degree\par
\textquotedbl Polygon Number\textquotedbl が3以上の時の、多角形膨らみの傾きを指定します。\par
\textquotedbl Polygon Number\textquotedbl が2のときは意味ありません。\par
ゼロ以上の Degree単位で指定します。\par
時計回りに回転します。\par
初期値は0です。\\
\\
Alpha Rendering\par
ONで Alphaにも処理をします。\par
OFFのときは RGBにのみ処理します。Alphaチャンネルのない BG 画で使います。\par
初期値は ONです。\\
\\
Reference\par
Pixel 毎に効果の強弱をつけるための参照画像の値の取り方を選択します。\par
入力の\textquotedbl Reference\textquotedbl に画像を接続し、\par
Red/Green/Blue/Alpha/Luminance/Nothingから選びます。\par
この効果をつけたくないときは Nothingを選ぶか、接続を切ります。\par
初期値は Red です。

\end{document}