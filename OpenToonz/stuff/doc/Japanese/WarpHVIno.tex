\documentclass[a4paper,12pt]{article}
\usepackage[a4paper, total={180mm, 272mm}]{geometry}

\usepackage{fontspec}
\setmainfont[Path=fonts/, Extension=.ttf]{ipaexm}

\setlength\parindent{3.5em}
\setlength\parskip{0em}
\renewcommand{\baselinestretch}{1.247}

\begin{document}

\thispagestyle{empty}

\Large
\noindent \\
Warp HV Ino\medskip
\par
\normalsize
参照画を元に絵をゆがませます。\\
\\
-{-}- \ 入力 \ -{-}-\\
Source\par
処理をする画像を接続します。\\
Hori\par
横方向にゆがませるための参照画像を接続します。\\
Vert\par
縦方向にゆがませるための参照画像を接続します。\\
\\
-{-}- \ 設定 \ -{-}-\\
H MaxLen\par
水平方向に絵を歪ませます。\par
歪みの最大値を指定します。\par
単位は mm で、0から100の値を与えます。\par
初期値は0で変化しないので何か値をいれてください。\\
\par
\textquotedbl Hori\textquotedbl 参照画像の赤値から各ピクセルの歪みが決まります。\par
8bits 画像の場合\par
\noindent \hskip 7em 1なら-MaxLen(mm)\par
\noindent \hskip 7em 128なら歪みなし\par
\noindent \hskip 7em 255なら MaxLen(mm)\par
16bits 画像の場合\par
\noindent \hskip 7em 1なら-MaxLen(mm)\par
\noindent \hskip 7em 32768なら歪みなし\par
\noindent \hskip 7em 65535なら MaxLen(mm)\par
の歪みとなります。\\
\\
V MaxLen\par
垂直方向に絵をゆがませます。\par
\textquotedbl Vert\textquotedbl 参照画像の赤値から各ピクセルの歪みが決まります。\par
他は\textquotedbl H MaxLen\textquotedbl と同様です。\\
\\
Alpha Rendering\par
OFFのとき Alphaチャンネルにはなにもしません。\par
ONで Alphaにも処理をします。\par
初期値は ONです。\\
\\
Anti Aliasing\par
OFFのときは絵のジャギーを維持します。

\newpage

\thispagestyle{empty}

\ \vspace{-0.2em}
\par
ON ならジャギーの少ない絵に処理します。\par
初期値は ON です。

\end{document}