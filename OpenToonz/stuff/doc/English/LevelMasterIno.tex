\documentclass[a4paper,12pt]{article}
\usepackage[a4paper, total={180mm, 272mm}]{geometry}

\usepackage{fontspec}
\setmainfont[Path=fonts/, Extension=.ttf]{ipaexm}

\setlength\parindent{3.5em}
\setlength\parskip{0em}
\renewcommand{\baselinestretch}{1.247}

\begin{document}

\thispagestyle{empty}

\Large
\noindent \\
Level Master Ino\medskip
\par
\normalsize
Add correction to the level.\par
Please use \textquotedbl Level RGBA\textquotedbl \ when you want to correct the RGB separately.\\
\\
-{-}- \ Inputs \ -{-}-\\
Source\par
Connect the image to process.\\
Reference\par
Connect the reference image to put the strength of the effect into each Pixel.\\
\\
-{-}- \ Settings \ -{-}-\\
In\par
Specify the Min (Minimum) value and Max (Maximum) value of the input Pixel value.\par
Min value is less than the Min, Max value is greater than the Max limit.\par
Minimum value is 0, maximum value is 1.\par
The default values are, Min is 0, Max is 1.\par
Values can take an input of up to 4 decimal places.\\
\\
Out\par
Specify the range for \textquotedbl In\textquotedbl ,\par
Set to the range of Min (Minimum) value and Max (Maximum) value that you specify.\par
Minimum value is 0, maximum value is 1.\par
The default values are, Min is 0, Max is 1.\par
Values can take an input of up to 4 decimal places.\\
\\
Gamma\par
Used for the gamma correction between \textquotedbl Out Min\textquotedbl \ and \textquotedbl Out Max\textquotedbl .\par
A Value between 0.1 and 1.0, will make the image become darker.\par
It does not compensate when you specify a value of 1.0.\par
A value between 1.0 and 10.0, will make the image become brighter.\par
The default value is 1.\\
\\
Alpha Rendering\par
When ON it will also process the Alpha channel.\par
When OFF, it does not process the Alpha channel.\par
The default setting is ON.\\
\\
Premultiplied\par
When ON, RGB is already set to Premultiply\par
(The value of Alpha channel is multiplied in advance RGB channels)\par
and processes as an image.

\newpage

\thispagestyle{empty}

\ \vspace{-0.2em}
\par
At that time, and added to the Alpha processing, you may not get the correct image.\par
The default setting is ON.\\
\\
Reference\par
Choose how Reference image values put the strength of the effect into each Pixel.\par
An image is connected to the \textquotedbl Reference\textquotedbl \ of the input,\par
Choose from Red/Green/Blue/Alpha/Luminance/Nothing.\par
Choose Nothing when you do not want this effect, it will turn off the connection.\par
The default setting is Red.

\end{document}