\documentclass[a4paper,12pt]{article}
\usepackage[a4paper, total={180mm, 272mm}]{geometry}

\usepackage{fontspec}
\setmainfont{ipaexm.ttf}

\setlength\parindent{3.5em}
\setlength\parskip{0em}
\renewcommand{\baselinestretch}{1.247}

\begin{document}

\thispagestyle{empty}

\Large
\noindent \\
HSV Add Ino\medskip
\par
\normalsize
Add atmosphere (Noise) using values of the specified channel from reference image.\par
Hue, Saturation, Brightness, and Alpha will change.\\
\par
The cell image tone, Hue, Saturation, Brightness, and Alpha will have added noise,\par
use this to be able to adapt the background picture.\par
When painting with the same image, the noise of each dot will be a single tone.\par
Otherwise, the user can provide a reference image as noise,\par
for calculating the noise component of the image, to determine the atmosphere.
\\
\par
Since the processing for a still image is only two-dimensional, if you have the need\par
for a three-dimensional video, another consideration, processing (?) Is required.\\
\par
The Alpha channel will determine the strength of the effect. Thus,\par
masked edges of Alpha will retain their state if they are smooth.\\
\\
-{-}- \ Inputs \ -{-}-\\
Source\par
Connect the image to process.\\
Noise\par
Connect the image for adding the noise by the HSV value for each Pixel.\\
Reference\par
Connect the reference image to put the strength of the effect into each Pixel.\\
\\
-{-}- \ Settings \ -{-}-\\
From RGBA\par
Specify whether to use the channel for the (Noise) Reference image.\par
Choose from one of the \textquotedbl Red\textquotedbl \ \textquotedbl Green\textquotedbl \ \textquotedbl Blue\textquotedbl \ \textquotedbl Alpha\textquotedbl \ channels.\par
The default value is \textquotedbl Red\textquotedbl .\\
\\
Offset\par
The offset value for the Pixel value of the (Noise) Reference image.\par
Choose a Pixel value (8 or 16bits) as a value away from zero, you can specify a value\par
between -1.0 and 1.0.\par
Pixel values do not change when the same value is specified here.\par
Depending on the magnitude of the reference Pixel value, values will change.\par
The default value is 0.5.\\
\\
Hue\par
It specifies the intensity of the color (Hue) change.\par
The value will scale with the Offset value.

\newpage

\thispagestyle{empty}

\ \vspace{-0.2em}
\par
You can use a value from -1.0 to 1.0.\par
The image will not change when the value is 0.\par
The default value is 0.0.\\
\\
Saturation\par
Specify the strength of the chroma (Saturation) change.\par
The default value is 0.0.\par
Other options are the same as \textquotedbl Hue\textquotedbl .\\
\\
Value\par
Specify the strength of the brightness (Value) change.\par
The default value is 0.25.\par
Other options are the same as \textquotedbl Hue\textquotedbl .\\
\\
Alpha\par
Specify the strength of the opacity (Alpha) change.\par
The default value is 0.0.\par
Other options are the same as \textquotedbl Hue\textquotedbl .\\
\\
Premultiply\par
When ON, RGB is already set to Premultiply\par
(The value of Alpha channel is multiplied in advance RGB channels)\par
and processes as an image.\par
At that time, and added to the Alpha processing, you may not get the correct image.\par
The default setting is ON.\\
\\
Reference\par
Choose how Reference image values put the strength of the effect into each Pixel.\par
An image is connected to the \textquotedbl Reference\textquotedbl \ of the input,\par
Choose from Red/Green/Blue/Alpha/Luminance/Nothing.\par
Choose Nothing when you do not want this effect, it will turn off the connection.\par
The default setting is Red.

\end{document}