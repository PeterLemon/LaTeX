\documentclass[a4paper,12pt]{article}
\usepackage[a4paper, total={148.3mm, 245mm}, left=34mm]{geometry}

\usepackage[colorlinks,linkcolor=blue,bookmarks,bookmarksopen,pdfauthor=krom]{hyperref}

\usepackage[T1]{fontenc}
\usepackage[portuguese,brazil]{babel}

\usepackage{soul}

\usepackage[normalem]{ulem}
\renewcommand{\ULthickness}{0.04em}

\usepackage[compact]{titlesec}
\titlespacing{\section}{-0.1em}{0em}{0em}
\titlespacing{\subsection}{2.22em}{0em}{1.5em}
\titleformat{\section}{\ttfamily}{\thesection}{0em}{}
\titleformat{\subsection}{\ttfamily}{\thesection}{0em}{}

\setlength\parindent{5.78em}
\renewcommand{\baselinestretch}{1.2}

\usepackage{enumitem}

\usepackage{fancyhdr}
\pagestyle{fancy}
\fancyhf{}
\renewcommand\headrulewidth{0pt}

\frenchspacing

\usepackage{epic}

\begin{document}

\ttfamily
\sodef\an{}{0.05em}{0.6em}{0em}

\noindent \\

\vspace{2em}
\hspace{5em} \an{UNIVERSIDADE DE SÃO PAULO}\par
\hspace{7.2em} \an{ESCOLA POLITÉCNICA}\par
\hspace{0.05em} \an{DEPARTAMENTO DE ENGENHARIA DE ELETRICIDADE}\par
\hspace{2.8em} \an{LABORATÕRIO DE SISTEMAS DIGITAIS}\par

\vspace{19em}
\hspace{4.45em} \an{MONTADOR DO ``PATINHO FEIO''}\par

\vspace{7em}
\hspace{8.1em} \an{Antonio Marcos de Aguirra Massola}\par
\hspace{8.1em} \an{João José Neto}\par
\hspace{8.1em} \an{Moshe Bain}\par

\vspace{7em}
\hspace{10em} \an{Julho}\par
\hspace{10.3em} \an{1977}

\newpage

\noindent \\

\vspace{21em}
\hspace{6.7em} \an{Em memória de}\par
\hspace{6.7em} \an{Laís Costa Ortenzi}\par

\newpage

\setcounter{page}{1}
\fancyhead[R]{\ttfamily {I\hskip 0.05em .\hskip 0.05em \thepage}}

\renewcommand\contentsname{\ttfamily \hskip 14.5em \uline{I\hskip 0.05em N\hskip 0.05em D\hskip 0.05em I\hskip 0.05em C\hskip 0.05em E}\\
\\
\uline{A\hskip 0.05em s\hskip 0.05em s\hskip 0.05em u\hskip 0.05em n\hskip 0.05em t\hskip 0.05em o} \hskip 26em \uline{P\hskip 0.05em á\hskip 0.05em g\hskip 0.05em i\hskip 0.05em n\hskip 0.05em a}}
\tableofcontents

\newpage

\setcounter{page}{1}
\setcounter{section}{1}
\fancyhead[R]{\ttfamily {\thesection \hskip 0.05em .\hskip 0.05em \thepage}}

\phantomsection
\section*{\an{1 - }\uline{I\hskip 0.05em N\hskip 0.05em T\hskip 0.05em R\hskip 0.05em O\hskip 0.05em D\hskip 0.05em U\hskip 0.05em Ç\hskip 0.05em Ã\hskip 0.05em O}}
\addcontentsline{toc}{section}{C\hskip 0.05em A\hskip 0.05em P\hskip 0.05em Í\hskip 0.05em T\hskip 0.05em U\hskip 0.05em L\hskip 0.05em O \hskip 0.05em  \hskip 0.05em 1 \hskip 0.05em  \hskip 0.05em - \hskip 0.05em  \hskip 0.05em I\hskip 0.05em N\hskip 0.05em T\hskip 0.05em R\hskip 0.05em O\hskip 0.05em D\hskip 0.05em U\hskip 0.05em Ç\hskip 0.05em Ã\hskip 0.05em O}

\noindent \\
\par
\an{O ante-projeto do minicomputador Patinho Feio nasceu\\
de um curso de pós-graduação dado pelo Professor Glen \hfill George\\
Langdon Jr., em 1972. A seguir, os engenheiros e estagiários do\\
Laboratório de Sistemas Digitais (LSD) da EPUSP terminaram \hfill o\\
projeto e montaram o Patinho Feio que, dessa forma, se \hfill tornou\\
o primeiro computador projetado e construído no Brasil.\\
\par
Os circuitos do Patinho Feio são totalmente \hfill consti-\\
tuídos por circuitos integrados da família TTL (``transistor\,\, \hfill -\\
transistor logic''), apresentando uma memória de núcleos de fe{\uline r}\\
rite, e tendo um ciclo de máquina de dois microsegundos.\\
\par
O Patinho Feio foi destinado a pesquisas no LSD, ta{\uline n}\\
to na área de programação (``software'') como dos circuitos ele-\\
trônicos (``hardware'').\\
\par
Cuidou-se do desenvolvimento de um ``software'' que pe{\uline r}\\
mitisse um uso mais eficiente do minicomputador, já que, \hfill de\\
início só se podia programá-lo em linguagem de máquina, manua{\uline l}\\
mente, através do seu painel. Em particular, foi definida \hfill uma\\
linguagem de montador (``assembly language''), que associa a ca-\\
da instrução de máquina um mnemônico, e um programa \hfill montador\\
(``assembler''), cuja função é traduzir programas escritos \hfill em\\
linguagem de montador para linguagem de máquina, os quais \hfill são\\
os assuntos tratados neste manual.\\
\par
Este manual foi escrito de forma a tratar cada tópi-\\
co de forma mais ou menos extensa, na suposição de que o \hfill lei-\\
tor tenha tido previamente apenas um pequeno contato com \hfill a\\
área de computação, e pouco ou nenhum conhecimento de \hfill lingu{\uline a}\\
gens de baixo nível, como um montador. Por causa disso,tentou-\\
se fazer com que o manual fosse o mais auto-explicativo e ind{\uline e}\\
pendente possível de outros textos. Naturalmente é \hfill impossível}

\newpage

\noindent \an{que um texto seja completamente independente de outros; por i{\uline s}\\
so, recomenda-se consultar outros textos, tais como manuais de\\
operação do Patinho Feio e de seus equipamentos periféricos(de\\
entrada/saída), textos sobre números binários, etc.\\
\par
Foi feito um bom esforço para apresentar os \hfill concei-\\
tos com clareza e para padronizar as notações, com o \hfill objetivo\\
de tornar o manual realmente útil. Contudo, certamente \hfill muitas\\
falhas subsistem, de forma que são bem recebidas quaisquer su-\\
gestões e críticas de modo a melhorar o manual em futuras edi-\\
ções.}\\
\\
\\[-0.5em]
\uline{O\hskip 0.05em b\hskip 0.05em s\hskip 0.05em e\hskip 0.05em r\hskip 0.05em v\hskip 0.05em a\hskip 0.05em ç\hskip 0.05em õ\hskip 0.05em e\hskip 0.05em s}\hskip 0.05em :

\begin{enumerate}[label=\alph*\hskip 0.05em ), align=left, leftmargin=1.65em, labelsep=-0.25em, itemsep=1em, topsep=1em]
\item \an{As informações contidas neste manual são as melhores que se\\
pôde obter na época em que o manual foi escrito (setembro de\\
1975). Contudo, devido ao constante desenvolvimento de \hfill no-\\
vos projetos de ``hardware'' e ``software'' para o Patinho Feio,\\
alguns detalhes podem ter sofrido alterações até a presente\\
data.}
\item \an{Os programas e trechos de programa existentes no manual fo-\\
ram aí colocados por estarem sintaticamente corretos, \hfill mas\\
não representam necessariamente exemplos de boa técnica \hfill de\\
programação.}
\end{enumerate}

\newpage

\setcounter{page}{1}
\setcounter{section}{2}

\phantomsection
\section*{\an{2 - }\uline{A\hskip 0.05em R\hskip 0.05em I\hskip 0.05em T\hskip 0.05em M\hskip 0.05em É\hskip 0.05em T\hskip 0.05em I\hskip 0.05em C\hskip 0.05em A \hskip 0.05em \hskip 0.05em B\hskip 0.05em I\hskip 0.05em N\hskip 0.05em Ã\hskip 0.05em R\hskip 0.05em Í\hskip 0.05em A \hskip 0.05em \hskip 0.05em E \hskip 0.05em \hskip 0.05em H\hskip 0.05em E\hskip 0.05em X\hskip 0.05em A\hskip 0.05em D\hskip 0.05em E\hskip 0.05em C\hskip 0.05em I\hskip 0.05em M\hskip 0.05em A\hskip 0.05em L}}
\addcontentsline{toc}{section}{C\hskip 0.05em A\hskip 0.05em P\hskip 0.05em Í\hskip 0.05em T\hskip 0.05em U\hskip 0.05em L\hskip 0.05em O \hskip 0.05em \hskip 0.05em 2 \hskip 0.05em  \hskip 0.05em - \hskip 0.05em  \hskip 0.05em A\hskip 0.05em R\hskip 0.05em I\hskip 0.05em T\hskip 0.05em M\hskip 0.05em É\hskip 0.05em T\hskip 0.05em I\hskip 0.05em C\hskip 0.05em A \hskip 0.05em  \hskip 0.05em B\hskip 0.05em I\hskip 0.05em N\hskip 0.05em Ã\hskip 0.05em R\hskip 0.05em Í\hskip 0.05em A \hskip 0.05em E \hskip 0.05em  \hskip 0.05em H\hskip 0.05em E\hskip 0.05em X\hskip 0.05em A\hskip 0.05em D\hskip 0.05em E\hskip 0.05em C\hskip 0.05em I\hskip 0.05em M\hskip 0.05em A\hskip 0.05em L}

\hskip -3.3em \an{(Com números inteiros)}\\
\\

\phantomsection
\subsection*{\an{1. }\uline{B\hskip 0.05em a\hskip 0.05em s\hskip 0.05em e\hskip 0.05em s \hskip 0.05em \hskip 0.05em d\hskip 0.05em e \hskip 0.05em \hskip 0.05em N\hskip 0.05em u\hskip 0.05em m\hskip 0.05em e\hskip 0.05em r\hskip 0.05em a\hskip 0.05em ç\hskip 0.05em ã\hskip 0.05em o}}
\addcontentsline{toc}{subsection}{\hskip 5.68em B\hskip 0.05em a\hskip 0.05em s\hskip 0.05em e\hskip 0.05em s \hskip 0.05em \hskip 0.05em d\hskip 0.05em e \hskip 0.05em \hskip 0.05em N\hskip 0.05em u\hskip 0.05em m\hskip 0.05em e\hskip 0.05em r\hskip 0.05em a\hskip 0.05em ç\hskip 0.05em ã\hskip 0.05em o}

\an{Utiliza-se, na vida diária, a base decimal de numer{\uline a}\\
ção para representar os números. Isto significa duas coisas:}

\begin{enumerate}[label=\alph*\hskip 0.05em ), align=left, leftmargin=1.65em, labelsep=-0.25em, itemsep=1em, topsep=1em]
\item \an{existem dez algarismos com os quais todos os números são r{\uline e}\\
presentados (pois a base de numeração é dez), a saber: 0, 1,\\
2, 3, 4, 5, 6, 7, 8, 9.}
\item \an{emprega-se uma } \uline{n\hskip 0.05em o\hskip 0.05em t\hskip 0.05em a\hskip 0.05em ç\hskip 0.05em ã\hskip 0.05em o \hskip 0.05em \hskip 0.05em p\hskip 0.05em o\hskip 0.05em s\hskip 0.05em i\hskip 0.05em c\hskip 0.05em i\hskip 0.05em o\hskip 0.05em n\hskip 0.05em a\hskip 0.05em l} \an{ onde está subentendido que,\\
quando um algarismo é deslocado de uma posição para a esque{\uline r}\\
da, seu valor é multiplicado por dez. Por exemplo:}\\[-0.5em]
\phantom \ \ \ \ \ \ \ \ \ \ \ \ \ \ \an{2 \ \ \ \ \ \ \ 1 \ \ \, \, \, \, 0}\\[-1em]
\an{295 = 2 x 10} \ \ \, \an{+ 9 x 10} \ \ \, \an{+ 5 x 10}
\end{enumerate}

\an{Generalizando, quando se escreve o número N = d \ d \ \ ...}\\[-1em]
\phantom \ \ \ \ \ \ \ \ \ \ \ \ \ \ \ \ \ \ \ \ \ \ \ \ \ \ \ \ \ \ \ \ \ \ \ \ \ \ \ \ \ \ \ \ \ \ \ \ \ \ \ \ \ \ \ \ \ \ \ \ \ \, \, \an{n \ n-1}\\[-0.5em]
d \, d \, d \ \an{ (sem sinal), onde os d } \an{ (i= 0, 1, 2,...,n) são os seus}\\[-1em]
\phantom \ 2 \, 1 \, 0 \ \ \ \ \ \ \ \ \ \ \ \ \ \ \ \ \ \ \ \ \ \ \ \, i\\[-0.5em]
\an{algarismos (ou dígitos), está-se querendo dizer que: N = d \hfill x}\\[-1em]
\phantom \ \ \ \ \ \ \ \ \ \ \ \ \ \ \ \ \ \ \ \ \ \ \ \ \ \ \ \ \ \ \ \ \ \ \ \ \ \ \ \ \ \ \ \ \ \ \ \ \ \ \ \ \ \ \ \ \ \ \ \ \ \ \ \ n \\[-1em]
\phantom \ \, n \ \ \ \ \ \ \ \ \ \ \ \ \an{n-1} \ \ \ \ \ \ \ \ \ \ \ \ \ \ \ \, \, 2 \ \ \ \ \ \ \ \ \ \, 1 \ \ \ \ \ \ \ \ \ \ 0\\[-1em]
\an{10} \ \ \, \an{+ d} \ \ \ \, \an{x 10} \ \ \ \, \, \an{+ ....+ d} \ \ \, \an{x 10} \ \ \, \an{+ d} \ \ \, \an{x 10} \ \ \, \an{+ d} \ \ \ \an{x 10} \,\,\,\, .\\[-1em]
\phantom \ \ \ \ \ \ \, \, \an{n-1} \ \ \ \ \ \ \ \ \ \ \ \ \ \ \ \, \, \, 2 \ \ \ \ \ \ \ \ \ \ 1 \ \ \ \ \ \ \ \ \ \ 0\\
\par
\an{Nada obriga a que se use apenas a base dez. Na verda-
de, qualquer base } \uline{b} \an{(inteira) pode ser escolhida para \ represen-
tar um número. Para tanto, escolhem-se } \uline{b} \an{ símbolos distintos (os\\
algarismos da base) que representam os números de zero a (b - 1).\\
Escrevendo-se agora n + 1 algarismos adjacentes d d} \ \ \ \, \an{...d d \ e}\\[-1em]
\phantom \ \ \ \ \ \ \ \ \ \ \ \ \ \ \ \ \ \ \ \ \ \ \ \ \ \ \ \ \ \ \ \ \ \ \ \ \ \ \ \ \ \ \ \ \ \ \ \ \ \ \ \ \ \ \ \an{n n-1 \ \,\, 1 0}\\[-0.5em]
\an{subentendida a notação posicional descrita acima, tem-se o núm{\uline e}\\
ro N representado por essa notação:}\par
\ \ \ \ \ \ \ \ \ \ \ \ \ \, n \ \ \ \ \ \ \ \ \ \ \an{n-1} \ \ \ \ \ \ \ \ \ \ \ \ 1 \ \ \ \ \ \ \ 0\\[-1em]
\phantom \ \ \ \ \ \ \ \ \ \ \ \ \ \ \, \an{N = d} \ \ \, \an{. b} \ \ \, \an{+ d} \ \ \ \ \, \an{b} \ \ \, \, \an{+ ...+ d} \ \ \ \an{b \ + d} \ \ \ \an{b}\\[-1em]
\phantom \ \ \ \ \ \ \ \ \ \ \ \ \ \ \ \ \ \ \ \ n \ \ \ \ \ \ \ \ \ \ \an{n-1} \ \ \ \ \ \ \ \ \ \ \ \ \ \, 1 \ \ \ \ \, \, 0\\
\par
\an{Inversamente, pode-se provar que cada número N tem uma\\
única representação, numa dada base {\uline b}, que satisfaz as condições\\
mencionadas acima.\\
\par
Exemplo: Escolhendo b = 3, têm-se três algarismos;}

\newpage

\noindent \an{convencionalmente usa-se 0, 1, 2. Então tem-se:}\\
\\[-1em]
\phantom \ \ \ \ \ \ \ \ \ \ \ \ \ \ \ \ \ \ \ \ \ \ \ 3 \ \ \ \ \ \ \ \ 2 \ \ \ \ \ \ \ \ 1 \ \ \ \ \ \ \ \ 0\\[-1em]
\phantom \ \ \ \ \ \, \, \an{(1202)} \ \ \, \an{= 1 x 3} \ \ \, \an{+ 2 x 3} \ \ \, \an{+ 0 x 3} \ \ \, \an{+ 2 x 3} \ \ \, \an{= (47)}\\[-1em]
\phantom \ \ \ \ \ \ \ \ \ \ \ \ \, 3 \ \ \ \ \ \ \ \ \ \ \ \ \ \ \ \ \ \ \ \ \ \ \ \ \ \ \ \ \ \ \ \ \ \ \ \ \ \ \ \ \ \ \ \ \ \ \, \an{10}\\
\par
\an{Pode-se começar a perceber a importância do que \hfill foi\\
dito acima quando se considera que os computadores modermos tr{\uline a}\\
balham sempre, em última análise, com a base dois.}\\
\\

\phantomsection
\subsection*{\an{2. }\uline{B\hskip 0.05em a\hskip 0.05em s\hskip 0.05em e\hskip 0.05em s \hskip 0.05em \hskip 0.05em m\hskip 0.05em a\hskip 0.05em i\hskip 0.05em s \hskip 0.05em \hskip 0.05em e\hskip 0.05em m\hskip 0.05em p\hskip 0.05em r\hskip 0.05em e\hskip 0.05em g\hskip 0.05em a\hskip 0.05em d\hskip 0.05em a\hskip 0.05em s \hskip 0.05em e\hskip 0.05em m \hskip 0.05em \hskip 0.05em c\hskip 0.05em o\hskip 0.05em m\hskip 0.05em p\hskip 0.05em u\hskip 0.05em t\hskip 0.05em a\hskip 0.05em ç\hskip 0.05em ã\hskip 0.05em o}}
\addcontentsline{toc}{subsection}{\hskip 5.68em B\hskip 0.05em a\hskip 0.05em s\hskip 0.05em e\hskip 0.05em s \hskip 0.05em \hskip 0.05em m\hskip 0.05em a\hskip 0.05em i\hskip 0.05em s \hskip 0.05em \hskip 0.05em e\hskip 0.05em m\hskip 0.05em p\hskip 0.05em r\hskip 0.05em e\hskip 0.05em g\hskip 0.05em a\hskip 0.05em d\hskip 0.05em a\hskip 0.05em s \hskip 0.05em e\hskip 0.05em m \hskip 0.05em \hskip 0.05em c\hskip 0.05em o\hskip 0.05em m\hskip 0.05em p\hskip 0.05em u\hskip 0.05em t\hskip 0.05em a\hskip 0.05em ç\hskip 0.05em ã\hskip 0.05em o}

\an{Além da base dez, que é de uso geral, empregam-se co- 
mumente as seguintes bases:}

\begin{enumerate}[label=\alph*\hskip 0.05em ), align=left, leftmargin=1.65em, labelsep=-0.25em, itemsep=1em, topsep=1em]
\item \an{base dois (binária) - necessita dois algarismos distintos p{\uline a}\\
ra representar os números zero e um. Por convenção utilizam-
se os símbolos 0 e 1, Um algarismo binário é também \ chamado\\
``bit'' (do inglês ``binary digit'').\\
\\
A base dois é extremamente importante pois, como já foi cit{\uline a}\\
do, os computadores só entendem sequências de zeros e uns \ ,\\ 
que são usadas tanto para representar as instruções dadas \ à\\
máquina quanto números prop iamente ditos.}
\item \an{base oito (octal) - utiliza os algarismos de 0 a 7. Não será\\
aqui tratada com mais detalhes porque não é utilizada no Pa-
tinho Feio, embora o seja em vários outros computadores.}
\item \an{base dezesseis (hexadecimal) - os dígitos hexadecimais \ são:\\
0, 1, 2, 3, 4, 5, 6, 7, 8, 9, A, B, C, D, E, F; usados \ para\\
representar os números de zero a quinze.}
\end{enumerate}

\noindent \\[-2.5em]
 \phantom \ \ \ \ \ \ \ \ \ \ \ \ \ \ \ \ \ \ \ \ \ \ \ \ \ \ \ \ \ \,\, 1 \ \ \ \ \ \ \ \ \ \, 0\\[-1em] 
\phantom \ \ \ \, \an{Exemplo: (AB)} \ \ \ \, \an{= 10 x 16 \ + 11 x 16} \ \ \, \an{= (171)}\\[-1em]
\phantom \ \ \ \ \ \ \ \ \ \ \ \ \ \ \ \, \, 16 \ \ \ \ \ \ \ \ \ \ \ \ \ \ \ \ \ \ \ \ \ \ \ \ \ \ \ \ \ \ \ \ \ \an{10}\\

\an{A correspondência entre os valores binários, decimais\\
e hexadecimais é apresentada na tabela seguinte (note-se que são\\
necessários quatro bits para representar todos os dígitos \ hexa\\
decimais na base dois).}

\newpage

\ \ \ \, \, \uline{D\hskip 0.05em e\hskip 0.05em c\hskip 0.05em i\hskip 0.05em m\hskip 0.05em a\hskip 0.05em l} \ \ \, \uline{H\hskip 0.05em e\hskip 0.05em x\hskip 0.05em a\hskip 0.05em d\hskip 0.05em e\hskip 0.05em c\hskip 0.05em i\hskip 0.05em m\hskip 0.05em a\hskip 0.05em l} \ \ \, \uline{B\hskip 0.05em i\hskip 0.05em n\hskip 0.05em á\hskip 0.05em r\hskip 0.05em i\hskip 0.05em o}

\ \ \ \ \ \ \ \ \ 0 \ \ \ \ \ \ \ \ \ \ \ 0 \ \ \ \ \ \ \ \ \ \ \ \ \an{0000}\par
\ \ \ \ \ \ \ \ \ 1 \ \ \ \ \ \ \ \ \ \ \ 1 \ \ \ \ \ \ \ \ \ \ \ \ \an{0001}\par
\ \ \ \ \ \ \ \ \ 2 \ \ \ \ \ \ \ \ \ \ \ 2 \ \ \ \ \ \ \ \ \ \ \ \ \an{0010}\par
\ \ \ \ \ \ \ \ \ 3 \ \ \ \ \ \ \ \ \ \ \ 3 \ \ \ \ \ \ \ \ \ \ \ \ \an{0011}\par
\ \ \ \ \ \ \ \ \ 4 \ \ \ \ \ \ \ \ \ \ \ 4 \ \ \ \ \ \ \ \ \ \ \ \ \an{0100}\par
\ \ \ \ \ \ \ \ \ 5 \ \ \ \ \ \ \ \ \ \ \ 5 \ \ \ \ \ \ \ \ \ \ \ \ \an{0101}\par
\ \ \ \ \ \ \ \ \ 6 \ \ \ \ \ \ \ \ \ \ \ 6 \ \ \ \ \ \ \ \ \ \ \ \ \an{0110}\par
\ \ \ \ \ \ \ \ \ 7 \ \ \ \ \ \ \ \ \ \ \ 7 \ \ \ \ \ \ \ \ \ \ \ \ \an{0111}\par
\ \ \ \ \ \ \ \ \ 8 \ \ \ \ \ \ \ \ \ \ \ 8 \ \ \ \ \ \ \ \ \ \ \ \ \an{1000}\par
\ \ \ \ \ \ \ \ \ 9 \ \ \ \ \ \ \ \ \ \ \ 9 \ \ \ \ \ \ \ \ \ \ \ \ \an{1001}\par
\ \ \ \ \ \ \ \ 10 \ \ \ \ \ \ \ \ \ \ \ A \ \ \ \ \ \ \ \ \ \ \ \ \an{1010}\par
\ \ \ \ \ \ \ \ 11 \ \ \ \ \ \ \ \ \ \ \ B \ \ \ \ \ \ \ \ \ \ \ \ \an{1011}\par
\ \ \ \ \ \ \ \ 12 \ \ \ \ \ \ \ \ \ \ \ C \ \ \ \ \ \ \ \ \ \ \ \ \an{1100}\par
\ \ \ \ \ \ \ \ 13 \ \ \ \ \ \ \ \ \ \ \ D \ \ \ \ \ \ \ \ \ \ \ \ \an{1101}\par
\ \ \ \ \ \ \ \ 14 \ \ \ \ \ \ \ \ \ \ \ E \ \ \ \ \ \ \ \ \ \ \ \ \an{1110}\par
\ \ \ \ \ \ \ \ 15 \ \ \ \ \ \ \ \ \ \ \ F \ \ \ \ \ \ \ \ \ \ \ \ \an{1111}\\

\phantomsection
\subsection*{\an{3. } \uline{C\hskip 0.05em o\hskip 0.05em n\hskip 0.05em v\hskip 0.05em e\hskip 0.05em r\hskip 0.05em s\hskip 0.05em ã\hskip 0.05em o \hskip 0.05em \hskip 0.05em d\hskip 0.05em e \hskip 0.05em \hskip 0.05em n\hskip 0.05em ú\hskip 0.05em m\hskip 0.05em e\hskip 0.05em r\hskip 0.05em o\hskip 0.05em s \hskip 0.05em \hskip 0.05em e\hskip 0.05em n\hskip 0.05em t\hskip 0.05em r\hskip 0.05em e \hskip 0.05em \hskip 0.05em a\hskip 0.05em s \hskip 0.05em  \hskip 0.05em b\hskip 0.05em a\hskip 0.05em s\hskip 0.05em e\hskip 0.05em s \hskip 0.05em d\hskip 0.05em o\hskip 0.05em i\hskip 0.05em s\hskip 0.05em , \hskip 0.05em d\hskip 0.05em e\hskip 0.05em z \hskip 0.05em e \ \hskip 0.05em \hskip 0.05em d\hskip 0.05em e\hskip 0.05em z\hskip 0.05em e\hskip 0.05em s\hskip 0.05em -}\\
\phantom \ \ \, \uline{s\hskip 0.05em e\hskip 0.05em i\hskip 0.05em s}}
\addcontentsline{toc}{subsection}{\hskip 5.68em C\hskip 0.05em o\hskip 0.05em n\hskip 0.05em v\hskip 0.05em e\hskip 0.05em r\hskip 0.05em s\hskip 0.05em ã\hskip 0.05em o \hskip 0.05em \hskip 0.05em d\hskip 0.05em e \hskip 0.05em \hskip 0.05em n\hskip 0.05em ú\hskip 0.05em m\hskip 0.05em e\hskip 0.05em r\hskip 0.05em o\hskip 0.05em s \hskip 0.05em \hskip 0.05em e\hskip 0.05em n\hskip 0.05em t\hskip 0.05em r\hskip 0.05em e \hskip 0.05em \hskip 0.05em a\hskip 0.05em s \hskip 0.05em  \hskip 0.05em b\hskip 0.05em a\hskip 0.05em s\hskip 0.05em e\hskip 0.05em s \hskip 0.05em d\hskip 0.05em o\hskip 0.05em i\hskip 0.05em s\hskip 0.05em ,\hskip 0.05em d\hskip 0.05em e\hskip 0.05em z \hskip 0.05em e\hskip 0.05em \hskip 0.05em d\hskip 0.05em e\hskip 0.05em z\hskip 0.05em e\hskip 0.05em s\hskip 0.05em s\hskip 0.05em e\hskip 0.05em i\hskip 0.05em s}

\an{Conforme já se deve ter percebido, surge frequente -\\
mente a necessidade de converter números escritos em uma \ base\\
psta outra. Para isso existem algoritmos gerais, dos quais são\\
apresentados abaixo alguns casos particulares:}

\begin{enumerate}[label=\alph*\hskip 0.05em ), align=left, leftmargin=1.65em, labelsep=-0.25em, itemsep=1em, topsep=1em]
\item \an{Conversão para a base dez de números escritos em outra base.}\\[-0.5em]
\phantom \ \ \ \ \ \ \ \ \ \ \ \ \ \ \ \ \ \ \ \ \ \ \ \ \ \ \ \ \ \ \ \ \ \ \ \ \ \ \ \ \ \ \ n\\[-1em]
\an{Basta escrever o número na forma d} \ \ \, \an{. b} \ \ \, \an{+ ...+ d} \ \ \, \an{e efetuar}\\[-1em]
\phantom \ \ \ \ \ \ \ \ \ \ \ \ \ \ \ \ \ \ \ \ \ \ \ \ \ \ \ \ \ \ \ \ \ \ \ \, \, n \ \ \ \ \ \ \ \ \ \ \ \ \ \, 0\\[-0.5em]
\an{as operações indicadas.}\\[0.5em]
\an{Exemplos: 19)} \ \ \, \an{(101111100001)} \ \ \, \an{para a base 10}\\[-1em]
\phantom \ \ \ \ \ \ \ \ \ \ \ \ \ \ \ \ \ \ \ \ \ \ \ \ \ \ \ \ \ \ \ \ 2\\[-0.5em]
\phantom \ \ \ \ \ \ \ \ \ \ \ \ \ \ \ \ \ \ \ \ \ \ \ \an{11} \ \ \ \ \ \, \, \an{10} \ \ \ \ \ \ \ \ \ \ \ \, 1\\[-1em]
\phantom \ \ \ \ \ \ \ \ \ \ \ \ \ \ \ \, \, \an{= 1.2} \ \ \ \, \an{+ 0.2} \ \ \ \, \an{+ ...+ 0.2} \ \ \, \an{+ 1 =} \ \ \, \an{(3041)}\\[-1em]
\phantom \ \ \ \ \ \ \ \ \ \ \ \ \ \ \ \ \ \ \ \ \ \ \ \ \ \ \ \ \ \ \ \ \ \ \ \ \ \ \ \ \ \ \ \ \ \ \ \ \ \ \ \ \ \ \ \ \ \ \ \ \, \, \an{10}\\[1em]
\phantom \ \ \ \ \ \ \ \ \ \ \ \ \an{29)} \ \ \, \an{(BE1)} \ \ \, \, \an{para a base 10}\\[-1em]
\phantom \ \ \ \ \ \ \ \ \ \ \ \ \ \ \ \ \ \ \ \ \ \ \ \an{16}\\
\phantom \ \ \ \ \ \ \ \ \ \ \ \ \ \ \ \ \ \ \ \ \ \ \ \ \, \an{2}\\[-1em]
\phantom \ \ \ \ \ \ \ \ \ \ \ \ \ \ \, \an{= 11 x 16} \ \, \, \an{+ 14 x 16 + 1 =} \ \ \, \an{(3041)}\\[-1em]
\phantom \ \ \ \ \ \ \ \ \ \ \ \ \ \ \ \ \ \ \ \ \ \ \ \ \ \ \ \ \ \ \ \ \ \ \ \ \ \ \ \ \ \ \ \ \ \ \ \ \ \, \, \, \an{10}\\[1em]

\newpage

\noindent \an{Uma forma conveniente de fazer isso é dada \ nos \hfill diagramas \\
abaixo:\\}\\[-0.5em]
\phantom \ \ \ \ \ \ \ \ \ \ \ \ \ \ \ \, \an{(BE1)} \ \ \, \, \an{para a base 10}\\[-1em]
\phantom \ \ \ \ \ \ \ \ \ \ \ \ \ \ \ \ \ \ \ \ \ \ \an{16}\\

\noindent \ \ \ \ \ \ \ \ \ \uline{\ \ \ \ \ \, 1\hskip 0.05em1 \, \, 1\hskip 0.05em4 \ \ \, 1 \ \ \ \, \, }\\[-0.5em]
\an{base}\\[-1em]
\phantom \ \ \ \ \ \ \ \ \ \ \ \ \, x\\[-1em]
\an{original=16} \ \ \ \, \an{11} \ \ \ \ \an{190} \ \ \ \ \an{(3041)}\\[-1em]
\phantom \ \ \ \ \ \ \ \ \ \ \ \ \ \ \ \ \ \ \ \ \ \ \ \ \ \ \ \ \ \ \ \ \ \ \an{10}

\begin{picture}(0,0)
\put(101,46){\vector(0,-1){10}}
\put(92,30){\vector(-1,0){15}}
\end{picture}

\noindent \an{Método usado: 11 x 16 + 14 = 190}\\[1em]
\phantom \ \ \ \ \ \ \ \ \ \ \ \ \ \ \ \ \ \ \ \ \ \ \ \ \ \ \ \ \ \ \ \ \ \an{190 x 16 + 1 = 3041}

\begin{picture}(0,0)
\put(207,48){\vector(0,-1){15}}
\end{picture}

\noindent \\[-1em]
\phantom \ \ \ \ \ \ \ \ \ \ \ \ \ \ \ \, \an{(10110)} \ \, \, \an{para a base 10}\\[-1em]
\phantom \ \ \ \ \ \ \ \ \ \ \ \ \ \ \ \ \ \ \ \ \ \ \ 2\\[-0.5em]

\noindent \ \ \ \ \ \ \ \ \ \uline{\ \ \ \ \ \, 1 \, \, 0 \ \, 1 \, \, 1 \ \, 0 \ \ }\\
\an{base}\\[-0.5em]
\an{original= 2} \ \ \, 1 \, \, 2 \ \, 5 \, \, \an{11 (22)}\\[-1em]
\phantom \ \ \ \ \ \ \ \ \ \ \ \ \ \ \ \ \ \ \ \ \ \ \ \ \ \ \ \ \ \ \ \ \ \ \ \, \an{10}

\begin{picture}(0,0)
\put(97.5,53){\vector(0,-1){15}}
\end{picture}

\noindent \\[-2em]
\an{Método usado: 1\,x\,2 +\,0 = 2} \ \ \ \ \ \ \ \ \ \ \an{5\,x\,2 +\,1\,=\,11}\\
\\[-0.5em]
\phantom \ \ \ \ \ \ \ \ \ \ \ \ \ \ \ \ \ \ \ \ \ \ \ \ \ \, \, \an{2\,x\,2 +\,1\,=\,5} \ \ \ \ \ \ \ \ \ \ \an{11\,x\,2\,+\,0 = 22}

\begin{picture}(0,0)
\put(167,48){\vector(0,-1){15}}
\put(228.5,33){\vector(0,1){15}}
\put(293.5,48){\vector(0,-1){15}}
\end{picture}

\noindent \\[-4em]
\item \an{Conversão de números escritos na base dez para uma outra b{\uline a}\\
se. Divide-se} \ \ \,\an{repetidamente o número dado pela base de de{\uline s}\\
tino até que o quociente seja zero. Os restos obtidos \hfill são\\
a representação desejada, em ordem invertida. Ver os esque-\\
mas abaixo:}\\[-2em]

\phantom \ \ \ \ \ \ \ \ \ \ \ \ \ \ \an{(3041)} \ \ \ \ \ \ \ \ \ \an{base 16}\\[-1em]
\phantom \ \ \ \ \ \ \ \ \ \ \ \ \ \ \ \ \ \, \, \an{10}\\
\phantom \ \ \ \ \ \ \ \ \ \ \ \ \ \ \ \ \ \ \ \ \ \ \ \ \ \ \ \ \ \ \ \ \ \ \ \, \an{resto de 3041 : 16}\\
\phantom \ \ \ \ \ \ \ \ \ \ \ \ \ \ \ \ \ \ \ \ \ \ \ \ \ \ \ \ \ \ \ \ \ \ \ \, \an{resto de} \ \ \, \an{190 : 16}\\[-1em]
\phantom \ \ \ \ \ \ \ \ \ \ \ \ \ \ \ \ \ \, \, \an{3041} \ \ \ \, \, \an{1}\\[-0.5em]
\phantom \ \ \ \ \ \ \ \ \ \ \ \ \ \ \ \ \ \ \ \ \ \ \ \ \ \ \ \ \ \ \ \ \ \ \ \, \an{resto de} \ \ \ \, \an{11 : 16}\\[-0.5em]
\phantom \ \ \, \an{3041 : 16} \ \ \ \ \ \ \, \, \an{190} \ \ \, \, \an{14}\\[0.5em]
\phantom \ \ \ \, \an{190 : 16} \ \ \ \ \ \ \ \, \, \an{11} \ \ \ \ \ \an{11} \ \ \ \an{14} \ \ \, \an{1}\\[1em]
\phantom \ \ \ \, \, \an{11 : 16} \ \ \ \ \ \ \ \ \, \, \an{0} \ \ \ \, \, \an{(BE1)}\\[-1em]
\phantom \ \ \ \ \ \ \ \ \ \ \ \ \ \ \ \ \ \ \ \ \ \ \ \ \ \ \ \ \ \ \ \ \ \, \an{16}\\

\begin{picture}(0,0)
\put(157,28){\line(0,1){110}}
\put(82,101){\vector(1,0){30}}
\put(82,78){\vector(1,0){30}}
\put(82,48){\vector(1,0){30}}

\dashline{4}(180,125)(214,142)
\put(180,125){\vector(-2,-1){2}}

\dashline{4}(180,103)(214,124)
\put(180,103){\vector(-2,-1){2}}

\dashline{4}(180,80)(214,108)
\put(180,80){\vector(-1,-1){2}}

\put(179,120){\vector(1,-1){37}}
\put(179,98){\vector(1,-1){15}}
\put(172,72){\vector(1,-3){6}}
\put(196,72){\vector(-1,-2){9}}
\put(215,72){\vector(-1,-1){18}}
\end{picture}

\newpage







\end{enumerate}






\end{document}