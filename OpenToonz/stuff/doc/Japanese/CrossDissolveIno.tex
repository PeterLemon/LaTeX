\documentclass[a4paper,12pt]{article}
\usepackage[a4paper, total={180mm, 272mm}]{geometry}

\usepackage{fontspec}
\setmainfont[Path=fonts/, Extension=.ttf]{ipaexm}

\setlength\parindent{3.5em}
\setlength\parskip{0em}
\renewcommand{\baselinestretch}{1.247}

\begin{document}

\thispagestyle{empty}

\Large
\noindent \\
Cross Dissolve Ino\medskip
\par
\normalsize
二重写し\par
Back に Fore を重ねて溶け込ませ、Back を Fore に転換する。オーバーラップ\par
式 = Back × (1 - Opacity) + Fore × Opacity\par
式は RGB 各値をゼロから1とした場合の値で示しています。\\
\\
-{-}- \ 入力 \ -{-}-\par
両方とも接続していれば合成処理します。\par
動作スイッチ OFF のとき Back を表示します。\par
片方のみ繋がっていればそれを表示します。\\
Fore\par
上に重ねる画像を接続します。\\
Back\par
下に置く画像を接続します。\\
\\
-{-}- \ 設定 \ -{-}-\\
Opacity\par
上に重ねる画像の不透明度を指定します。\par
0の時は Fore 画像は透明になります。\par
初期値は\textquotedbl 1.0\textquotedbl で Fore 画像は不透明として合成します。\par
0から1.0の間の値を指定します。\\
\\
Clipping Mask\par
ON にすると、\par
素材(Back)の存在しない(Alpha 値がゼロ)場所は、透明のままにします。\par
初期設定は OFF です。

\end{document}